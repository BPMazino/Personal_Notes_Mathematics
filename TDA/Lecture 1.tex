\documentclass{article}
\usepackage{../preamble}


\title{Simplificial Complexes}
\author{Mazino}
\date{18 March 2024}


\begin{document}


\maketitle

\section{Motivation}

\section{Definitions}

\begin{definition}[Simplicies]

Simplciies can be seen as "triangles" in higher dimensions.\\
A 0-simplex is a point, a 1-simplex is a line segment, a 2-simplex is a triangle, a 3-simplex is a tetrahedron, and so on.
A k-simplex is the convex hull of k+1 affinely independent points in $\mathbb{R}^n$.
\end{definition}

\begin{remark}
    The boundary of a k-simplex is the union of its (k-1)-dimensional faces.
\end{remark}


A simplicial complex is a set of simplicies glued by their faces.

\begin{definition}[Simplicial Complex]
    Let $K$ be a set of simplicies in $\mathbb{R}^n$. $K$ is a simplicial complex if:
    \begin{enumerate}
        \item Every face of a simplex in $K$ is also in $K$.
        \item The intersection of any two simplicies in $K$ is a face of each of them.
    \end{enumerate}
\end{definition}

\begin{example}
    
\end{example}

For example can we construct a circle from a simplicial complex?

We can just take 3 three points and connect them with 1-simplicies to form a triangle And this is a 2-simplex. Then we can considere this as homeomorphic to a circle.

We can do the same with a 2-simplex to form a disk.


Let's considerate another definition of a simplicial complex.

\begin{definition}[Abstract simplicial complexes]
    An abstract simplicial complex is a non-empty family of sets (called simplicies) closed under the operation of taking subsets, i.e. if $A$ is a set in the family, and $B$ is a non empty subset of $A$, then $B$ is also in the family.

    A family of sets X is an abstract simplicial complex if and only if:
    \(Y_1 \in X \) and \(Y_2 \subset Y1 \) and  \(Y_2 \neq \emptyset \) implies \(Y_1 \cap Y_2 \in X \)
\end{definition}

More intuitively we can think that we take the simplicial complexes and we decompose it into 0-simplicies, 1-simplicies, 2-simplicies, etc. And we can think of the simplicial complex as a set of vertices, edges, triangles, etc.


\section{Application to data}
Let's say we have a set of points in $\mathbb{R}^n$, and real number $\alpha \geq 0$.

\begin{enumerate}
    \item The \textbf{Vietoris-Rips complex} \( VR_\alpha(X) \) is the simplicial complex whose k-simplicies \([x_1,\dots,x_k]\) such that \( d_X(x_i,x_j) \leq \alpha \) for all \( i,j \).
    So when the balls of radius \(\frac\alpha2 \)  around the points in \(X\) are not disjoint we can connect them with a 1-simplex. And when the balls of radius \(\frac\alpha2 \)  around the points in \(X\) are not disjoint we can connect them with a 2-simplex. And so on.

    \item The \textbf{Cech complex} \( Cech_\alpha(X) \) is the simplicial complex such that the k+1 closed balls of radius \(\alpha \) around the points in \(X\) have a non-empty intersection.
\end{enumerate}

\end{document}