\documentclass[12pt,a4paper]{article}
\usepackage[utf8]{inputenc}
\usepackage{lmodern}
\usepackage[T1]{fontenc}
\usepackage[french]{babel}
\usepackage{microtype}
\usepackage{mathtools}
\usepackage{amsmath,amsthm,amsfonts,amssymb}
\usepackage{dsfont}
\usepackage{stmaryrd}
\usepackage{graphicx}
\usepackage{numprint}
\usepackage{xfrac}
\usepackage{modroman}
\usepackage{enumitem}
\usepackage{titlesec}
\usepackage{hyperref}
\usepackage{tikz}
\usetikzlibrary{matrix,decorations.pathreplacing, calc, positioning,fit}
\usepackage[all]{xy}
\setlength{\parindent}{0pt}
\usepackage{geometry}
\geometry{rmargin=25mm,bmargin=25mm,lmargin=30mm,tmargin=35mm}
\DeclareMathOperator{\Img}{Im}
\DeclareMathOperator{\Ker}{Ker}
\DeclareMathOperator{\rg}{rg}
\DeclareMathOperator{\Sp}{Sp}
\DeclareMathOperator{\Tr}{Tr}
\DeclareMathOperator{\Hom}{Hom}
\DeclareMathOperator{\diag}{diag}
\DeclareMathOperator{\Mor}{Mor}
\DeclareMathOperator{\End}{End}
\DeclareMathOperator{\Aut}{Aut}
\DeclareMathOperator{\Isom}{Isom}
\DeclareMathOperator{\Pol}{Pol}
\DeclareMathOperator{\Id}{Id}
\DeclareMathOperator{\GL}{GL}
\DeclareMathOperator{\SL}{SL}
\DeclareMathOperator{\Or}{O}
\DeclareMathOperator{\Si}{S}
\DeclareMathOperator{\Supp}{Supp}
\DeclareMathOperator{\SO}{SO}
\DeclareMathOperator{\Vect}{Vect}
\DeclareMathOperator{\Alt}{Alt}
\DeclareMathOperator{\Sym}{Sym}
\DeclareMathOperator{\Mat}{Mat}
\DeclareMathOperator{\Com}{Com}
\DeclareMathOperator{\dive}{div}
\DeclareMathOperator{\grad}{grad}
\DeclareMathOperator{\rot}{rot}
\DeclareMathOperator{\Ric}{Ric}
\DeclareMathOperator{\Rscal}{Rscal}
\DeclareMathOperator{\Jac}{Jac}
\usepackage{geometry}
\geometry{rmargin=25mm,bmargin=25mm,lmargin=30mm,tmargin=35mm}
\theoremstyle{definition}
\newtheorem{thm}{Théorème}
\newtheorem{prop}[thm]{Proposition}
\newtheorem{defn}[thm]{Définition}
\newtheorem{ex}[thm]{Exemple}
\newtheorem{rqe}[thm]{Remarque}
\newtheorem*{dem}{Démonstration}
\newtheorem*{just}{Justification}
\newtheorem{cor}[thm]{Corollaire}
\newtheorem{lem}[thm]{Lemme}
\newcommand{\K}{\mathbf{K}}
\newcommand{\MnpK}{\mathcal{M}_{n,p}(\K)}
\newcommand{\MnK}{\mathcal{M}_n(\K)}
\newcommand{\MnR}{\mathcal{M}_n(\mathbf{R})}
\newcommand{\GLnR}{\mathrm{GL}_n(\mathbf{R})}
\newcommand{\GLnK}{\mathrm{GL}_n(\K)}
\newcommand{\SLnK}{\mathrm{SL}_n(\K)}
\newcommand{\OnR}{\mathrm{O}_n(\mathbf{R})}
\newcommand{\SOnR}{\mathrm{SO}_n(\mathbf{R})}
\newcommand{\OdR}{\mathrm{O}_2(\mathbf{R})}
\newcommand{\SOdR}{\mathrm{SO}_2(\mathbf{R})}
\newcommand{\DnK}{\mathrm{D}_n(\K)}
\newcommand{\TnplusK}{\mathrm{T}^+_n(\K)}
\newcommand{\TnplusplusK}{\mathrm{T}^{++}_n(\K)}
\newcommand{\TnmoinsK}{\mathrm{T}^-_n(\K)}
\newcommand{\TnmoinsmoinsK}{\mathrm{T}^{--}_n(\K)}
\newcommand{\SnK}{\mathrm{S}_n(\K)}
\newcommand{\AnK}{\mathrm{A}_n(\K)}
\author{Valentin \bsc{Clarisse} et Nassim \bsc{Arifette}}
\title{Géométrie différentielle}
\begin{document}
\maketitle
\newpage
\tableofcontents
\newpage
\section{Variétés différentielles}
--------------------
Un autre point de vue sur la notion d'atlas
On se donne $\mathcal{M}$ un ensemble.
\begin{defn}
Un \emph{atlas} de classe $\mathcal{C}^k$ et de dimension $n$ sur $\mathcal{M}$ est une collection de couples $(U_i,\phi_i)_{i\in I}$ tels que :
\begin{itemize}
\item $(U_i)_{i\in I}$ est un recouvrement de $\mathcal{M}$
\item Pour tout $i\in I$, $\phi_i:U_i\to \mathbf{R}^n$ est un injection
\item Pour tout $i,j\in I$, $\phi_i\circ\phi_j^{-1}$ est un difféomorphisme de classe $\mathcal{C}^k$
\end{itemize}
\end{defn}
Parler ensuite de la notion d'atlas compatibles, topologie canonique associée à un atlas, simplifier les définitions des fibrés (co)tangents, faire plus en détails les fibrés vectoriels sur des variétés
-------------------
Intuitivement, une variété différentielle est un espace topologique localement homéomorphe à un ouvert de $\mathbf{R}^n$ ($n$ étant fixé).\newline Pour pouvoir se repérer sur un tel objet, on le munit localement d'un système de coordonnées, et on exige que les changements de coordonnées aient une certaine régularité.\newline L'utilisation de systèmes de coordonnées permettra de développer un calcul différentiel sur une variété, nettement plus subtil que que le calcul différentiel sur un espace vectoriel normé.

\medskip

Dans ce paragraphe, on se donne $\mathcal{M}$ un espace topologique séparé et $\sigma$-compact (ie $\mathcal{M}$ est une union dénombrable de compacts), $n$ un entier naturel et $k\in\mathbf{N}^*\cup\{\infty\}$.
\begin{defn}
Une \textbf{carte} $\mathcal{M}$ est la donnée d'un couple $(U,\phi)$ où $U$ est un ouvert de $\mathcal{M}$ et $\phi:U\to\mathbf{R}^n$ est un homéomorphisme de $U$ sur $\phi(U)$.
\end{defn}
\begin{rqe}
On confonds parfois la carte $(U,\phi)$ et l'application $\phi$.
\end{rqe}
\begin{defn}
Deux cartes $(U,\phi)$ et $(V,\psi)$ sur $\mathcal{M}$ sont dites \textbf{$\mathcal{C}^k$-compatibles} si l'application de changement de carte $\phi\circ\psi^{-1}:\psi(U\cap V)\to \phi(U\cap V)$ est un $\mathcal{C}^k$-difféomorphisme.
\end{defn}
\begin{defn}
Un \textbf{atlas de classe $\mathcal{C}^k$ sur $\mathcal{M}$} est un recouvrement de $\mathcal{M}$ par des cartes deux à deux $\mathcal{C}^k$-compatibles.
\end{defn}
\begin{defn}
Deux atlas de classe $\mathcal{C}^k$ sur $\mathcal{M}$ sont dits \textbf{compatibles} si leur réunion est un atlas de classe $\mathcal{C}^k$.
\end{defn}
\begin{rqe}
On vérifie facilement que la relation de compatibilité est une relation d'équivalence sur l'ensemble des atlas de classe $\mathcal{C}^k$ sur $\mathcal{M}$. Chaque classe d'équivalence contient un unique représentant maximal au sens de l'inclusion (il suffit de prendre la réunion des atlas de la classe d'équivalence).
\end{rqe}
\begin{defn}
Une \textbf{variété différentielle de classe $\mathcal{C}^k$ et de dimension $n$} est la donnée d'un couple $(\mathcal{M},\mathcal{A})$ où $\mathcal{M}$ est un espace topologique séparé et $\mathcal{A}$ est un atlas maximal de classe $\mathcal{C}^k$ sur $\mathcal{M}$.
\end{defn}
\begin{rqe}
La différentielle d'un difféomorphisme étant un isomorphisme en tout point où elle est définie, on en déduit, en considérant une application de changement de carte, qu'il y a unicité de la dimension d'une variété différentielle.
\end{rqe}
\begin{rqe}
Par la suite, on arrêtera de préciser l'atlas maximal associé à la variété différentielle.
\end{rqe}
\begin{rqe}
Toute variété différentielle de classe $\mathcal{C}^k$ est de classe $\mathcal{C}^{l}$ pour tout $l\in\llbracket 1,k\rrbracket$.
\end{rqe}
\begin{ex}
Soit $E$ un $\mathbf{R}$-espace vectoriel de dimension $n$. Une atlas lisse de $E$ est donné par la seule carte $(E,\phi)$ où $\phi:E\to\mathbf{R}^n$ est un isomorphisme d'espaces vectoriels. L'atlas maximal associé est l'ensemble des difféomorphismes lisses définis sur des ouverts de $E$. Notons que la dimension de $E$ en tant que variété différentielle et la dimension de $E$ en tant que $\mathbf{R}$-espace vectoriel sont les mêmes.\newline Les $\mathbf{R}$-espaces vectoriels de dimensions finies seront systématiquement munis de cet atlas.
\end{ex}
\begin{ex}
Voyons un exemple moins trivial de variété différentielle.\newline On note $\displaystyle\mathbf{S}^n=\left\lbrace(x^1,\ldots,x^{n+1})\in\mathbf{R}^{n+1},\sum\limits_{i=1}^{n+1}(x^i)^2=1\right\rbrace$ la sphère de $\mathbf{R}^{n+1}$.\newline On note $N=(0,\ldots, 0,1)$ et $S=(0,\ldots, 0,-1)$ les pôles Nord et Sud de $\mathbf{S}^n$, et on pose $U_N=\mathbf{S}^n\backslash \{N\}$ et $U_S=\mathbf{S}^n\backslash\{S\}$. Prenons un point $(x^1,\ldots,x^{n+1})\in U_N$. On trace la droite passant par ce point et $N$ (ce qui est possible puisqu'il s'agit de deux points distincts), et on examine l'intersection de cette droite avec l'hyperplan $\mathbf{R}^n\times\{0\}$.\newline On vérifie que ce point d'intersection est $\dfrac{1}{1-x^{n+1}}(x^1,\ldots,x^n,0)$.\newline On définit ainsi l'application $\phi_N: U_N\to\mathbf{R}^n, (x^1,\ldots,x^{n+1})\mapsto\dfrac{1}{1-x^{n+1}}(x^1,\ldots,x^n)$.\newline De même, on définit $\phi_S: U_S\to\mathbf{R}^n, (x^1,\ldots,x^{n+1})\mapsto\dfrac{1}{1+x^{n+1}}(x^1,\ldots,x^n)$. On vérifie que $((U_N,\phi_N),(U_S,\phi_S))$ est un atlas sur $\mathbf{S}^n$, appelé \textbf{projection stéréographique}.\newline
Notons que, munie de cette structure différentielle, $\mathbf{S}^n$ est une variété différentielle de dimension $n$ (ce qui justifie qu'on la note $\mathbf{S}^n$, et non $\mathbf{S}^{n+1}$...).
\end{ex}
Voyons une première opération entre variétés différentielles : le produit.\newline D'autres opérations, comme le quotient, sont plus subtiles et ne sont pas traitées dans ce polycopié.
\begin{defn}
Soit $\mathcal{M}$ et $\mathcal{N}$ deux variétés différentielles de classe $\mathcal{C}^k$ et de dimensions $n$ et $p$ respectivement.\newline On munit $\mathcal{M}\times\mathcal{N}$ de la topologie produit (qui est séparée puisque $\mathcal{M}$ et $\mathcal{N}$ le sont).\newline Étant données deux cartes $(U,\phi)$ et $(V,\psi)$ de $\mathcal{M}$ et $\mathcal{N}$ respectivement, on définit la carte produit par : $(U,\phi)\times (V,\psi)=(U\times V,\phi\times\psi)$ avec
$$\phi\times\psi:U\times V\to\mathbf{R}^n\times\mathbf{R}^p,(x,y)\mapsto (\phi(x),\psi(y)).$$
L'ensemble des produits de cartes de $\mathcal{M}$ et $\mathcal{N}$ est un atlas de classe $\mathcal{C}^k$ sur $\mathcal{M}\times\mathcal{N}$.\newline On munit ainsi $\mathcal{M}\times\mathcal{N}$ d'une structure de variété différentielle de classe $\mathcal{C}^k$ et de dimension $n+p$.
\end{defn}
Par la suite, nous serons amenés à munir des ensembles de structures de variétés différentielles au moyen d'une famille d'applications dont on voudrait qu'elle soit un atlas.
\begin{lem}
Soit $\mathcal{M}$ un ensemble, $n\in\mathbf{N}$, et $((U_i,\phi_i))_{i\in I}$ une famille telle que, pour tout $i\in I$, $U_i$ est une partie de $\mathcal{M}$ et $\phi_i$ est une injection de $U_i$ dans $\mathbf{R}^n$ vérifiant :
\begin{enumerate}[label=\roman*)]
\item $(U_i)_{i\in I}$ recouvre $\mathcal{M}$ ;
\item Pour tout $i\in I$, $\phi_i(U_i)$ est un ouvert de $\mathbf{R}^n$ ;
\item Pour tout $i,j\in I$, $\phi_i\circ\phi_j^{-1}$ est de classe $\mathcal{C}^k$.
\end{enumerate}
Une partie $U$ de $\mathcal{M}$ est dite ouverte si, pour tout $i\in I$, $\phi_i(U\cap U_i)$ est un ouvert de $\mathbf{R}^n$.\newline

L'ensemble $\mathcal{M}$ étant muni de cette topologie, la famille $((U_i,\phi_i))_{i\in I}$ est un atlas sur $\mathcal{M}$.
\end{lem}
\begin{rqe}
La topologie ainsi définie sur $\mathcal{M}$ est la topologie la moins fine rendant les $\phi_i$ continues.
\end{rqe}
\newpage
\section{Sous-variétés}
Cette section est consacrée aux sous-variétés, notamment aux sous-variétés de $\mathbf{R}^n$, qui fourniront de nombreux exemples usuels de variétés différentielles.\newline On se donne $\mathcal{M}$ une variété différentielle, $k\in\mathbf{N}\cap\{+\infty\}$ et $n,p\in\mathbf{N}$.
\begin{defn}
Une \textbf{sous-variété de $\mathcal{M}$ de dimension $d\in\mathbf{N}$} est une partie $\mathcal{N}$ de $\mathbf{M}$ telle que, pour tout $p\in\mathcal{N}$, il existe une carte $(U,\phi)$ de $\mathcal{M}$ telle que $p\in U$ et $\phi(\mathcal{N}\cap U)=(\mathbf{R}^d\times\{0\})\cap\phi(U)$. Les couples de la forme $(\mathcal{N}\cap U,(\pi_{d}\circ\phi)_{|\mathcal{N}\cap U})$ (où $\pi_d:\mathbf{R}^n\to\mathbf{R}^d,(x^1,\ldots,x^n)\mapsto (x^1,\ldots,x^d)$, avec $n$ la dimension de $\mathcal{M}$) sont des cartes sur $\mathcal{N}$, munissant $\mathcal{N}$ d'une structure de variété différentielle de dimension $d$ et de régularité au moins égale à celle de $\mathcal{M}$.
\end{defn}
\begin{rqe}
Autrement dit, localement, en dehors de $d$ coordonnées, les coordonnées des points de $\mathcal{N}$ sont nulles.
\end{rqe}
\begin{ex}
Les ouverts de $\mathcal{M}$ sont des sous-variétés de $\mathcal{M}$, ayant même dimension que $\mathcal{M}$. Par exemple, $\GL_n(\mathbf{R})$ est une sous-variété de $\mathcal{M}_n(\mathbf{R})$ de dimension $n^2$.
\end{ex}
La suite de cette section est consacrée aux sous-variétés de $\mathbf{R}^n$. Nous donnerons deux conditions suffisantes simples permettant d'identifier des sous-variétés de $\mathbf{R}^n$.\newline Commençons par deux cas particuliers du théorème de forme normale.
\begin{defn}
Soit $U$ un ouvert de $\mathbf{R}^n$ et $f\in\mathcal{C}^1(U,\mathbf{R}^p)$. On dit que $f$ est :
\begin{itemize}
\item une \textbf{immersion (resp. submersion) en $x\in U$} si $\mathrm{d}f(x)$ est injective (resp. surjective)
\item une \textbf{immersion (resp. submersion)} si $f$ est une immersion (resp. submersion) en tout point de $U$
\item un \textbf{plongement} si $f$ est une immersion et un homéomorphisme de $U$ sur $f(U)$.
\end{itemize}
\end{defn}
\begin{lem}[Forme normale des immersions]
Soit $U$ un ouvert de $\mathbf{R}^n$ et $f:U\to \mathbf{R}^p$ une immersion. Pour tout $x\in U$, il existe $U_x$ un voisinage de $x$ inclus dans $U$, $V_{f(x)}$ un voisinage de $f(x)$ et $\psi:V_{f(x)}\to\mathbf{R}^p$ un difféomorphisme sur son image tels que :
$$
\forall (x^1,\ldots,x^n)\in U_x,(\psi\circ f)(x^1,\ldots,x^n)=(x^1,\ldots,x^n,0,\ldots,0).
$$
\end{lem}
\begin{dem}
Soit $x\in U$.\newline Comme $\mathrm{d}f(x)$ est injective, la famille $(v_i)_{i\in\llbracket 1,n\rrbracket}=(\mathrm{d}f(x)\cdot e_i)_{i\in\llbracket 1,n\rrbracket}$ (avec $(e_i)_{i\in\llbracket 1,n\rrbracket}$ la base canonique de $\mathbf{R}^n$) est libre. On la complète en une base $(v_i)_{i\in\llbracket 1,p\rrbracket}$ de $\mathbf{R}^p$.\newline On pose :$$g:U\times\mathbf{R}^{n-p}\to\mathbf{R}^n\times\mathbf{R}^{p-n},(x^1,\ldots,x^p)\mapsto f(x^1,\ldots,x^n)+x^{n+1}v_{n+1}+\cdots+x^{p}v_p.$$
On a $\mathrm{d}g(x,0,\ldots,0)=\Id_{\mathbf{R}^p}$, donc, d'après le théorème d'inversion locale, il existe $\widetilde{U_x}$ un voisinage de $(x,0,\ldots,0)$ inclus dans $U\times\mathbf{R}^{n-p}$ et $V_{f(x)}$ un voisinage de $f(x)$ tel que $g_{|\widetilde{U_x}}^{|V_{f(x)}}$ est un difféomorphisme. On pose $\iota_n:\mathbf{R}^n\to\mathbf{R}^p,(x^1,\ldots,x^n)\mapsto (x^1,\ldots,x^n,0,\ldots,0)$.\newline On vérifie que $\psi:=\left(g_{|\widetilde{U_x}}^{|V_{g(x)}}\right)^{-1}\circ \iota_{n|U_x}$ avec $U_x=U\cap \pi_p^{-1}\left(\widetilde{U_x}\right)$ convient.
\end{dem}
\begin{rqe}
Autrement dit, à un difféomorphisme local à gauche près, $f$ est localement l'injection canonique. En particulier, $f$ est localement injective.
\end{rqe}
\begin{lem}[Forme normale des submersions]
Soit $U$ un ouvert de $\mathbf{R}^n$ et $f:U\to \mathbf{R}^p$ une submersion. Pour tout $x\in U$, il existe $U_x$ un voisinage de $x$ inclus dans $U$ et $\phi:U_x\to\mathbf{R}^n$ un difféomorphisme sur son image tels que :
$$
\forall (x^1,\ldots,x^n)\in \phi(U_x),\left(f\circ\phi^{-1}\right)(x^1,\ldots,x^n)=(x^1,\ldots,x^p).
$$
\end{lem}
\begin{dem}
Comme $(\mathrm{d}f^i(x))_{i\in\llbracket 1,p\rrbracket}$ est une famille libre de $(\mathbf{R}^n)^*$, dont on peut la compléter à l'aide de formes linéaires $\varphi_{p+1},\ldots,\varphi_{n}\in(\mathbf{R}^n)^*$ en une base de $(\mathbf{R}^n)^*$.\newline On pose :$$g:U\to\mathbf{R}^p\times\mathbf{R}^{n-p},(x^1,\ldots,x^n)\mapsto (f(x^1,\ldots,x^n),\varphi_{p+1}(x^1,\ldots,x^n),\ldots,\varphi_n(x^1,\ldots,x^n)).$$
On a : $\mathrm{d}g(x)=(\mathrm{d}f(x),\varphi_{p+1},\ldots,\varphi_n)$, qui est inversible, donc, d'après le théorème d'inversion locale, il existe $U_x$ un voisinage de $x$ inclus dans $U$ et $\widetilde{V_{f(x)}}$ un voisinage de $g(x)$ tels que $g_{|U_x}^{|\widetilde{V_{f(x)}}}$ est un difféomorphisme. On vérifie que $\phi:=g_{|U_x}^{|\widetilde{V_{f(x)}}}$ convient.
\end{dem}
\begin{rqe}
Autrement dit, à un difféomorphisme local à droite près, $f$ est localement la projection canonique. En particulier, $f$ est localement surjective.
\end{rqe}
On peut enfin énoncer les deux résultats annoncés en début de section.
\begin{thm}[Définition d'une sous-variété par paramétrage]
Soit $U$ un ouvert de $\mathbf{R}^n$ et $f:U\to\mathbf{R}^p$ un plongement.\newline L'ensemble $f(U)$ est une sous-variété de $\mathbf{R}^p$ de dimension $n$.
\end{thm}
\begin{dem}
Soit $x\in U$. D'après le théorème de forme normale pour les immersions, il existe $U_x$ un voisinage de $x$ inclus dans $U$, $V_{f(x)}$ un voisinage de $f(x)$ et $\psi:V_{f(x)}\to\mathbf{R}^p$ un difféomorphisme sur son image tels que :
$$\forall (x^1,\ldots,x^n)\in U_x, (\psi\circ f)(x^1,\ldots,x^n)=(x^1,\ldots,x^n,0,\ldots,0).$$
Le couple $(V_{f(x)},\psi)$ est une carte sur $\mathbf{R}^p$ vérifiant $\psi(V_{f(x)})=(\mathbf{R}^n\times\{0\})\cap\psi(V_{f(x)})$.\newline On en déduit que $f(U)$ est une sous-variété de $\mathbf{R}^p$ de dimension $n$.
\end{dem}
\begin{rqe}
Ainsi, le plongement est la bonne façon d'injecter une variété différentielle dans une autre.
\end{rqe}
\newpage
\begin{thm}[Définition implicite d'une sous-variété]
Soit $U$ un ouvert de $\mathbf{R}^n$ et $f:U\to\mathbf{R}^p$ une submersion.\newline Pour tout $y\in f(U)$, $f^{-1}(y)$ est une sous-variété de $\mathbf{R}^n$, de dimension $n-p$.
\end{thm}
\begin{dem}
Soit $x\in f^{-1}(\{y\})$. D'après le théorème de forme normale pour les submersions, il existe $U_x$ un voisinage de $x$ inclus dans $U$ et $\phi:U_x\to\mathbf{R}^n$ un difféomorphisme sur son image tels que :
$$
\forall (x^1,\ldots,x^n)\in\phi(U_x), \left(f\circ\phi^{-1}\right)(x^1,\ldots,x^n)=(x^1,\ldots,x^p).
$$
On a : $\phi(U_x\cap f^{-1}(\{y\}))=(\{y\}\times\mathbf{R}^{n-p})\cap\phi(U_x)$, ce qui conclut.
\end{dem}
\begin{ex}
À l'aide de cette propriété, on vérifie que :
\begin{itemize}
\item $\mathbf{S}^n$ est une sous-variété de $\mathbf{R}^{n+1}$ de dimension $n$ (prendre $f:\mathbf{R}^{n+1}\to\mathbf{R},x\mapsto\Vert x\Vert^2$). La structure différentielle ainsi induite est la même que celle définie par la projection stéréographique.
\item $\SL_n(\mathbf{R})$ est une sous-variété de $\mathcal{M}_n(\mathbf{R})=\mathbf{R}^{n^2}$ de dimension $n^2-1$ (considérer $f:\mathcal{M}_n(\mathbf{R})\to\mathbf{R}, M\mapsto\det(M)$).
\item $\Or_n(\mathbf{R})$ est une sous variété de $\mathcal{M}_n(\mathbf{R})$ de dimension $\dfrac{n(n-1)}{2}$ (considérer\newline $f:\mathcal{M}_n(\mathbf{R})\to\mathcal{S}_n(\mathbf{R}),M\mapsto {}^tMM$).
\end{itemize}
Les deux derniers exemples sont importants : il s'agit de groupes de \bsc{Lie} (ie des groupes topologiques munis d'une structure de variété différentielle lisse pour laquelle la loi de composition interne et le passage à l'inverse sont lisses), particulièrement utilisés en physique.
\end{ex}
\newpage
\section{Applications différentiables}
Dans cette section, on définit une notion fondamentale du calcul différentiel : la différentielle d'une fonction en un point.

\medskip

On se donne $\mathcal{M}$ et $\mathcal{N}$ deux variétés différentielles de classe $\mathcal{C}^k$ avec $k\in\mathbf{N}^*\cup\{\infty\}$. On se donne également $l$ compris entre $1$ et $k$.
\begin{defn}
Une application continue $f:\mathcal{M}\to\mathcal{N}$ est de classe $\mathcal{C}^l$ si, pour tout point $p\in\mathcal{M}$, il existe une carte $(U,\phi)$ contenant $p$ et une carte $(V,\psi)$ contenant $f(p)$ telles que $\psi\circ f\circ\phi^{-1}$ est de classe $\mathcal{C}^l$.\newline On note $\mathcal{C}^l(\mathcal{M},\mathcal{N})$ l'ensemble des applications de classe $\mathcal{C}^l$ de $\mathcal{M}$ dans $\mathcal{N}$.
\end{defn}
\begin{rqe}
La continuité de $f$ est essentielle pour garantir la bonne définition de $\psi\circ f\circ\phi^{-1}$.
\end{rqe}
\begin{rqe}
L'application $\psi\circ f\circ\phi^{-1}$ n'est autre que la \og lecture de $f$ dans les cartes $(U,\phi)$ et $(V,\psi)$ \fg{}. 
\end{rqe}
\begin{ex}
Les cartes de $\mathcal{M}$ sont des applications de classe $\mathcal{C}^k$.
\end{ex}
\begin{prop}
Soit $f:\mathcal{M}\to\mathcal{N}$ de classe $\mathcal{C}^l$. Pour tout point $p$ de $\mathcal{M}$, pour toute carte $(U,\phi)$ contenant $p$ et pour toute carte $(V,\psi)$ contenant $f(p)$, $\psi\circ f\circ\phi^{-1}$ est de classe $\mathcal{C}^l$.
\end{prop}
\begin{dem}
C'est une conséquence directe de la compatibilité des cartes.\newline Soit $(U_0,\phi_0)$ une carte contenant $p$ et $(V_0,\psi_0)$ une carte contenant $f(p)$ telles que $\psi_0\circ f\circ\phi_0^{-1}$ est de classe $\mathcal{C}^l$. Par compatibilité, $\psi\circ\psi_0^{-1}$ et $\phi_0\circ\phi^{-1}$ sont de classe $\mathcal{C}^k$, donc $\psi\circ f\circ\phi^{-1}$ est de classe $\mathcal{C}^l$.
\end{dem}
\begin{prop}
La composée de deux applications de classe $\mathcal{C}^l$ est de classe $\mathcal{C}^l$.
\end{prop}
\begin{dem}
Soit $\mathcal{M},\mathcal{N},\mathcal{P}$ trois variétés différentielles de classe $\mathcal{C}^k$, et\newline $f:\mathcal{M}\to\mathcal{N}$ et $g:\mathcal{N}\to\mathcal{P}$ de classe $\mathcal{C}^l$. Soit $p$ un point de $\mathcal{M}$.\newline On se donne des cartes $(U,\phi),(V,\psi),(W,\varphi)$ contenant $p,f(p),g(f(p))$.\newline Les applications $\varphi\circ g\circ\psi^{-1}$ et $\psi\circ f\circ\phi^{-1}$ sont de classe $\mathcal{C}^l$, donc par composition, $\varphi\circ g\circ f\circ\phi^{-1}$ est de classe $\mathcal{C}^l$.
\end{dem}
\begin{prop}
Si $\mathcal{N}$ est un $\mathbf{R}$-espace vectoriel de dimension finie, alors $\mathcal{C}^l\left(\mathcal{M},\mathcal{N}\right)$ est un $\mathbf{R}$-espace vectoriel.
\end{prop}
\begin{dem}
C'est immédiat en prenant comme carte de $\mathcal{N}$ une base duale.
\end{dem}
\newpage
\section{Fibrés vectoriels}
Dans cette section, on se donne $k\in\mathbf{N}\cup\{\infty\}$.
\begin{defn}
Un \textbf{fibré vectoriel de dimension $n$ et de classe $\mathcal{C}^k$} est la donnée d'un triplet $(E,B,\pi)$ où $E$ et $B$ sont deux variétés différentielles de classe $\mathcal{C}^k$ appelées respectivement \textbf{espace total} et \textbf{base}, et où $\pi: E\to B$ est une application de classe $\mathcal{C}^k$ appelée \textbf{projection} tels que :
\begin{enumerate}[label=\roman*)]
\item Pour tout $b\in B$, la fibre $E_b=\pi^{-1}(\{b\})$ est munie d'une structure de $\mathbf{R}$-espace vectoriel de dimension $n$.
\item Il existe un recouvrement de $B$ par des ouverts $U$ dits \textbf{localement trivialisables} : pour un tel ouvert $U$, il existe $\varphi:U\times\mathbf{R}^n\to \pi^{-1}(U)$ un $\mathcal{C}^k$-difféomorphisme tel que, pour tout $b\in U$, $\varphi(b,\cdot)$ est un isomorphisme de $\mathbf{R}^n$ dans $E_b$.\newline Le couple $(U,\varphi)$ est qualifié de \textbf{trivialisation locale}.
\end{enumerate}
\end{defn}
\begin{rqe}
Un fibré est une sorte de famille régulière d'espaces vectoriels. En effet, pour tout $b_0\in B$, il existe $U$ un ouvert de $B$ et $e_1,\ldots,e_n$ des applications de $B$ dans $E$ de classe $\mathcal{C}^k$ telles que, pour tout $b\in U$, $(e_1(b),\ldots,e_n(b))$ est une base de $E_b$.
\end{rqe}
\begin{rqe}
On confonds souvent le fibré vectoriel et son espace total.
\end{rqe}
\begin{ex}
Étant donnée $\mathcal{M}$ une variété différentielle, on peut définir le fibré trivial d'espace total $\mathcal{M}\times\mathbf{R}^n$, de base $\mathcal{M}$ et de projection $\pi:\mathcal{M}\times\mathbf{R}^n\to\mathbf{M},(p,v)\mapsto p$.\newline On vérifie que $\Id_{\mathcal{M}\times\mathbf{R}^n}$ est une trivialisation globale.
\end{ex}
\begin{defn}
Une \textbf{section} d'un fibré vectoriel de classe $\mathcal{C}^k$ $(E,B,\pi)$ est une application de classe $\mathcal{C}^k$ $f:B\to E$ telle que $\pi\circ f=\Id_B$.\newline On note $\Gamma E$ l'ensemble des sections du fibré $E$.
\end{defn}
\begin{prop}
Soit $(E,B,\pi)$ un fibré vectoriel de classe $\mathcal{C}^k$.\newline L'ensemble $\Gamma E$ est muni des deux lois suivantes est un $\mathcal{C}^k(B,\mathbf{R})$-module :
\begin{itemize}
\item Pour toute section $f$, pour tout $\lambda\in\mathbf{R}$ : $\lambda f: B\to E, b\mapsto \lambda f(b)$ (la multiplication externe étant celle de $E_b$).
\item Pour toutes section $f,g$ : $f+g:B\to E, b\mapsto f(b)+g(b)$ (la somme étant celle de $E_b$).
\end{itemize}
\end{prop}
\begin{dem}
Seule la stabilité de $\Gamma E$ par les deux opérations précédentes est non triviale, les autres propriétés s'obtiennent par héritage.\newline
On se donne $U$ un ouvert de $B$ et $\varphi:U\times\mathbf{R}^n\to \pi^{-1}(U)$ une trivialisation locale. On a : $$\forall b\in U, \varphi^{-1}((\lambda f+g)(b))=(b,\lambda\varphi(b,\cdot)^{-1}(f(b))+\varphi(b,\cdot)^{-1}(g(b))).$$
On remarque que $\varphi(b,\cdot)^{-1}=\pi_{\mathbf{R}^n}\circ\psi_{|E_b}$ (avec $\psi$ la réciproque de $\varphi$ et\newline $\pi_{\mathbf{R}^n}:U\times\mathbf{R}^n\to\mathbf{R}^n,(x,y)\mapsto y$), donc la restriction de $\lambda f+g$ à $U$ est de classe $\mathcal{C}^k$, donc, comme on peut recouvrir $B$ d'ouverts trivialisables, $\lambda f+g$ est de classe $\mathcal{C}^k$.
\end{dem}
\newpage
\section{Fibré tangent}
\subsection{Espace tangent en un point}
L'objectif de ce paragraphe est de développer la notion d'espace tangent, déjà rencontrée au lycée. En effet, étant donnés un intervalle $I$ et $f:I\to\mathbf{R}$ de classe $\mathcal{C}^1$, en notant $\Gamma$ son graphe, on définit l'espace tangent à $\Gamma$ en un point $(p,f(p))$ par : $$T_{(p,f(p))}\Gamma=\{(x,y)\in\mathbf{R}^2, y-f(p)=f'(p)(x-p)\},$$
qui est une droite affine pouvant être munie d'une structure naturelle d'espace vectoriel (en prenant son vectorialisé au point $(p,f(p))$).\newline Plus généralement, étant donnée une partie $A$ de $\mathbf{R}^n$ et $p$ un point de $A$, un vecteur tangent à $A$ en $p$ est un vecteur vitesse d'un observateur se déplaçant sur $A$ passant par $p$. Formellement, un vecteur $v\in\mathbf{R}^n$ est tangent à $A$ en $p$ s'il existe $\varepsilon>0$ et $\gamma:]-\varepsilon,\varepsilon[\to A$ tels que $\gamma(0)=p$ et $\gamma'(0)=v$ (remarque sans importance pour la suite : si $A$ est quelconque, en général, l'ensemble des vecteurs tangents à $A$ en $p$ n'est pas un espace vectoriel).\newline Cette définition a un inconvénient majeur, qui rend a priori toute extension aux variétés abstraite impossible : elle nécessite de plonger $A$ dans un espace vectoriel.\newline Il nous faut donc trouver une parade pour définir la notion de vecteur tangent en un point d'une variété différentielle abstraite.\newline Notons que, si $f:A\to\mathbf{R}$ est une fonction numérique de classe $\mathcal{C}^1$ (pouvant être par exemple une coordonnée dans une carte), on peut poser $v(f):=(f\circ\gamma)'(0)$ (en termes de coordonnées, on regarde la variation de la coordonnées de $\gamma$ définie par $f$).\newline On peut ainsi identifier le vecteur tangent $v$ à l'application $f\mapsto v(f)$, qui est une dérivation.
\newpage
On se donne $\mathcal{M}$ une variété différentielle de classe $\mathcal{C}^k$ (avec $k\geqslant 2$) de dimension $n$ et $p$ un point de $\mathcal{M}$.\newline 
\begin{defn}
On note $T_p\mathcal{M}$ l'espace vectoriel des dérivations sur $\mathcal{C}^1(\mathcal{M},\mathbf{R})$, autrement dit l'ensemble des formes linéaires $D$ sur $\mathcal{C}^1(\mathcal{M},\mathbf{R})$ telles que
$$\forall f,g\in\mathcal{C}^1(\mathcal{M},\mathbf{R}), D(fg)=f(p)D(g)+g(p)D(f).$$
L'espace vectoriel $T_p\mathcal{M}$ est appelé \textbf{espace tangent à $\mathcal{M}$ en $p$}.
\end{defn}
Comme on peut si attendre, les opérateurs de dérivations partielles en $p$ constituent une base de $T_p\mathcal{M}$.
\begin{prop}
Soit $(x^1,\ldots,x^n)$ une carte dont l'ouvert de définition contient $p$.\newline Pour tout $i\in\llbracket 1,n\rrbracket$, on pose $\left.\dfrac{\partial}{\partial x^i}\right|_p:\mathcal{C}^1(\mathcal{M},\mathbf{R})\to\mathbf{R},f\mapsto\partial_i(f\circ x^{-1})(x(p))$.\newline
La famille $\left(\left.\dfrac{\partial}{\partial x^i}\right|_p\right)_{i\in\llbracket 1,n\rrbracket}$ est une base de $T_p\mathcal{M}$.
\end{prop}
\begin{dem}
Soit $D\in T_p\mathcal{M}$ et $f\in\mathcal{C}^1(\mathcal{M},\mathbf{R})$. D'après la formule de \bsc{Taylor}, pour tout $q$ au voisinage de $p$ :
$$
f(q)=f(p)+\sum\limits_{i=1}^n\int_{0}^1(tx^i(q)+(1-t)x^i(p))\partial_i(f\circ x^{-1})(x(p))\mathrm{d}t.
$$
Donc, comme $D$ s'annule en les fonctions constantes :
$$
D(f)=\sum\limits_{i=1}^nD(x^i)\partial_i(f\circ x^{-1})(x(p))=\sum\limits_{i=1}^nD(x^i)\left.\dfrac{\partial f}{\partial x^i}\right|_p.
$$
La règle de \bsc{Leibniz} donne que, pour tout $i\in\llbracket 1,n\rrbracket$, $\left.\dfrac{\partial}{\partial x^i}\right|_p$ appartient à $T_p\mathcal{M}$, ce qui conclut.
\end{dem}
\begin{ex}
Prenons $\mathcal{M}=\mathbf{R}^n$, munie de la carte $(\mathbf{R}^n,\Id_{\mathbf{R}^n})$.\newline Moyennant l'identification $\left.\dfrac{\partial}{\partial x^i}\right|_p\leftrightarrow e_i$ avec $e_i$ le $i$-ième vecteur de la base canonique de $\mathbf{R}^n$, on a : $T_p\mathbf{R}^n=\mathbf{R}^n$.
\end{ex}
\newpage
\begin{rqe}
Les coordonnées d'un vecteur tangent $v\in T_p\mathcal{M}$ dans la base précédente sont $\left(v(x^i)\right)_{i\in\llbracket 1,n\rrbracket}$, et $T_p\mathcal{M}$ est un $\mathbf{R}$-espace vectoriel qui a la même dimension que celle de la variété $\mathcal{M}$.
\end{rqe}
\begin{prop}
Soit $(x^1,\ldots,x^n),(y^1,\ldots,y^n)$ deux cartes définies au point $p$.\newline On a la relation de changement de base suivante :
$$
\forall i\in\llbracket 1,n\rrbracket, \left.\dfrac{\partial}{\partial y^i}\right|_p=\sum\limits_{j=1}^n\left.\dfrac{\partial x^j}{\partial y^i}\right|_p\left.\dfrac{\partial}{\partial x^j}\right|_p.
$$
\end{prop}
\begin{dem}
Immédiat d'après la remarque précédente.
\end{dem}
\begin{prop}
Soit $\mathcal{N}$ une variété différentielle de classe $\mathcal{C}^k$ de dimension $m$, et soit $q$ un point de $\mathcal{N}$. On a :
$$
T_{(p,q)}(\mathcal{M}\times\mathcal{N})\simeq T_p\mathcal{M}\times T_q\mathcal{N},
$$
l'isomorphisme étant pris au sens des $\mathbf{R}$-espaces vectoriels.
\end{prop}
\begin{dem}
L'application
$$T_{(p,q)}(\mathcal{M}\times\mathcal{N})\to T_p\mathcal{M}\times T_q\mathcal{N},v\mapsto (v\circ\pi_p,v\circ\pi_q)$$ avec $$\pi_p:\mathcal{C}^k(\mathcal{M},\mathbf{R})\to\mathcal{C}^k(\mathcal{M}\times\mathcal{N},\mathbf{R}),f\mapsto ((p,q)\mapsto f(q))$$ et $$\pi_q:\mathcal{C}^k(\mathcal{N},\mathbf{R})\to\mathcal{C}^k(\mathcal{M}\times\mathcal{N},\mathbf{R}),f\mapsto ((p,q)\mapsto f(p))$$
est un isomorphisme d'espaces vectoriels.
\end{dem}
On termine cette section par une injection entre espaces tangents.
\begin{prop}
Soit $\mathcal{N}$ une sous-variété de $\mathcal{M}$ et $p\in\mathbf{N}$.\newline L'application $T_p\mathcal{N}\to T_p\mathcal{M},X\mapsto\widetilde{X}$ avec $\widetilde{X}$ défini par :
$$\forall f\in\mathcal{C}^\infty(\mathcal{M},\mathbf{R}), \widetilde{X}(f)=X(f_{|\mathcal{N}})$$
est une injection linéaire, permettant d'identifier $T_p\mathcal{N}$ à une partie de $T_p\mathcal{M}$.
\end{prop}
\begin{rqe}
Si on se donne $(x^1,\ldots,x^n)$ une carte sur $\mathcal{M}$ définie au voisinage de $p$ telle que, au voisinage de $p$ dans $\mathcal{N}$, les $n-d$ dernières coordonnées sont nulles, un vecteur tangent à $\mathcal{M}$ en $p$ est tangent à $\mathcal{N}$ en $p$ si et seulement si ses $n-d$ coordonnées sont nulles.
\end{rqe}
\newpage
\subsection{Intermède : une autre définition de l'espace tangent}
On se replace temporairement dans le cadre du premier paragraphe de la sous-section précédente.\newline 
L'idée de considérer $(f\circ \gamma)'(0)$ peut être exploitée différemment. En effet, on peut procéder à la lecture du chemin $\gamma$ associé à un vecteur tangent dans une carte $\phi$, et examiner $(\phi\circ\gamma)'(0)$.\newline Ce point de vue motive la définition alternative de l'espace tangent qui suit.
\begin{defn}
Deux chemins $\gamma_1,\gamma_2$ définis sur un ouvert de $\mathbf{R}$ contenant $0$, à valeurs dans $\mathcal{M}$, de classe $\mathcal{C}^1$ et tels que $\gamma_1(0)=\gamma_2(0)=p$ sont dit \textbf{équivalents} s'il existe une carte $(U,\phi)$ sur $\mathcal{M}$ contenant $p$ telle que $(\phi\circ\gamma_1)'(0)=(\phi\circ\gamma_2)'(0)$. La relation ainsi définie est une relation d'équivalence. L'ensemble quotient associé est noté $T_p\mathcal{M}$.\newline On notera $[\gamma]$ la classe d'équivalence du chemin $\gamma$.
\end{defn}
\begin{rqe}
Si la relation $(\phi\circ\gamma_1)'(0)=(\phi\circ\gamma_2)'(0)$ est vraie pour une carte $(U,\phi)$ de $\mathcal{M}$ contenant $p$, alors elle est vraie pour toute carte sur $\mathcal{M}$ contenant $p$. 
\end{rqe}
\begin{rqe}
Contrairement à la définition en termes de dérivations, on n'a pas immédiatement une structure d'espace vectoriel sur $T_p\mathcal{M}$.
\end{rqe}
\begin{prop}
Soit $(U,\phi)$ une carte sur $\mathcal{M}$ contenant $p$.\newline L'application $T_p\mathcal{M}\to\mathbf{R}^n,[\gamma]\mapsto(\phi\circ\gamma)'(0)$ définit une bijection. Par transport de structure, on munit $T_p\mathcal{M}$ d'une structure de $\mathbf{R}$-espace vectoriel (de dimension finie égale à $n$, puisque l'application précédente est alors un isomorphisme), indépendante du choix de la carte $(U,\phi)$.
\end{prop}
\begin{dem}
L'application précédente, qu'on notera $f_\phi$ dans la preuve, est bien définie et injective.\newline Passons à la surjectivité. On vérifie que, pour tout $v\in\mathbf{R}^n$, la classe du chemin $t\mapsto \phi^{-1}(\phi(p)+tv)$ a pour image $v$, ce qui conclut la première partie de la preuve. Montrons que la structure ainsi définie ne dépend pas de la carte choisie.\newline Soit $(V,\psi)$ une carte sur $\mathcal{M}$ contenant $p$. Étant donnés un chemin $\gamma$, on a :
$$
(\phi\circ\gamma)'(0)=(\phi\circ\psi^{-1}\circ\psi\circ\gamma)'(0)=\mathrm{d}(\phi\circ\psi^{-1})(\psi(p))\cdot (\psi\circ\gamma)'(0),
$$
donc $f_\phi=\mathrm{d}(\phi\circ\psi^{-1})(\psi(p))\circ f_\psi$ avec $\mathrm{d}(\phi\circ\psi^{-1})(\psi(p))$ une automorphisme d'espaces vectoriels de $\mathbf{R}^n$, ce qui conclut.
\end{dem}
\begin{rqe}
En écrivant $\phi=(x^1,\ldots,x^n)$, l'application linéaire envoyant $\left.\dfrac{\partial}{\partial x^i}\right|_p$ sur la classe de $t\mapsto\phi^{-1}(\phi(p)+te_i)$ pour tout $i\in\llbracket 1,n\rrbracket$ (avec $(e_i)_{i\in\llbracket 1,n\rrbracket}$ la base canonique de $\mathbf{R}^n$) est un isomorphisme d'espaces vectoriels entre les deux définitions de $T_p\mathcal{M}$.
\end{rqe}
Dans le cas d'uns sous-variété de $\mathbf{R}^n$, en prenant la carte triviale $(\mathbf{R}^n,\Id_{\mathbf{R}^n})$, on peut identifier $[\gamma]$ à $\gamma'(0)$, et on retrouve alors une définition précédente : $T_p\mathcal{M}$ est l'ensemble des dérivées en $0$ des chemins de classe $\mathcal{C}^1$ tracés sur $\mathcal{M}$ passant par $p$ à l'instant $0$.
\newpage
\subsection{Fibré tangent}
On se donne $\mathcal{M}$ une variété différentielle de classe $\mathcal{C}^k$ (avec $k\geqslant 2$) de dimension $n$.
\begin{defn}
On pose $\displaystyle T\mathcal{M}=\bigcup_{p\in\mathcal{M}}\{p\}\times T_p\mathcal{M}$ le \textbf{fibré tangent de $\mathcal{M}$}.\newline On note $(U_i,\phi_i)_{i\in I}$ l'atlas de $\mathcal{M}$.\newline D'après le lemme concluant la première section de ce polycopié, $T\mathcal{M}$ est naturellement muni d'une topologie pour laquelle $(\mathcal{U}_i,\Phi_i)_{i\in I}$ avec $\mathcal{U}_i=\displaystyle\bigcup_{p\in U_i}\{p\}\times T_p\mathcal{M}$ et $$\begin{array}{ccccc}
\Phi_i & : & \mathcal{U}_i & \to & \mathbf{R}^{n}\times\mathbf{R}^n \\
 & & (p,v) & \mapsto & \left(\phi_i(p),\left(\left(\left.\dfrac{\partial}{\partial x^k}\right|_p\right)^*(v)\right)_{k\in\llbracket 1,n\rrbracket}\right) \\
\end{array}$$
(où on a écrit $\phi_i=(x_i^1,\ldots,x_i^n)$) est un atlas faisant de $T\mathcal{M}$ une variété différentielle de classe $\mathcal{C}^{k-1}$ et de dimension $2n$.
\end{defn}
\begin{just}
Il est évident que $(\mathcal{U}_i)_{i\in I}$ recouvre $T\mathcal{M}$.\newline De plus, pour tout $i\in I$, $\Phi_i(\mathcal{U}_i)=\phi_i(U_i)\times\mathbf{R}^n$, qui est un ouvert de $\mathbf{R}^{2n}$.\newline Pour tout $i,j\in I$ :
$$\begin{array}{ccccc}
\Phi_i\circ\Phi_j^{-1} & : & \mathbf{R}^n\times\mathbf{R}^n & \to & \mathbf{R}^n\times\mathbf{R}^n \\
 & & (x,y) & \mapsto & \left(\left(\phi_i\circ\phi_j^{-1}\right)(x),\left(\displaystyle\sum\limits_{l=1}^ny^l\left.\dfrac{\partial x_i^k}{\partial x_j^l}\right|_{\phi_j^{-1}(x)}\right)_{k\in\llbracket 1,n\rrbracket}\right) \\
\end{array},$$
qui est de classe $\mathcal{C}^{k-1}$ (où on a écrit $\phi_i=(x_i^1,\ldots,x_i^n)$).
\end{just}
\begin{prop}
La projection $\pi: T\mathcal{M}\to\mathcal{M},(p,v)\mapsto p$ est de classe $\mathcal{C}^{k-1}$.
\end{prop}
\begin{dem}
Soit $U$ un ouvert de $\mathcal{M}$. On a : $\displaystyle\pi^{-1}(U)=\bigcup_{p\in U}\{p\}\times T_p\mathcal{M}$.\newline En recouvrant $\mathcal{M}$ par des ouverts associés à des cartes, on se ramène au cas où $U$ est associé à une carte, ce qui donne que $\pi^{-1}(U)$ est ouvert par définition de la topologie de $T\mathcal{M}$.

\medskip

Soit $(U,\phi)$ une carte sur $\mathcal{M}$.\newline
On a : $\phi\circ\pi\circ\Phi^{-1}:\phi(U)\times\mathbf{R}^n\to\mathbf{R}^n,(x,v)\mapsto x$, donc $\phi\circ\pi\circ\Phi^{-1}$ est de classe $\mathcal{C}^{k-1}$, donc $\pi$ est de classe $\mathcal{C}^{k-1}$.
\end{dem}
\newpage
\begin{prop}
Le triplet $(T\mathcal{M},\mathcal{M},\pi)$ est un fibré vectoriel de classe $\mathcal{C}^{k-1}$ et de dimension $n$.
\end{prop}
\begin{dem}
Les fibres de $\pi$ sont les espaces tangents, qui sont des espaces vectoriels de dimension $n$.\newline Les trivialisations locales sont données par les applications $\Phi^{-1}$.
\end{dem}
\begin{defn}
Une section du fibré tangent est appelée \textbf{champ vectoriel}.
\end{defn}
\begin{rqe}
Soit $(U,\phi)$ une carte sur $\mathcal{M}$, dont on note $(x^1,\ldots,x^n)$ les coordonnées, et $X$ un champ vectoriel de classe $\mathcal{C}^{k-1}$, qu'on écrit localement $\displaystyle X=\sum\limits_{i=1}^nX^i\dfrac{\partial}{\partial x^i}$.\newline Montrons que les applications $X^i$ sont de classe $\mathcal{C}^{k-1}$. On a, pour tout $p\in U$ :
$$
\Phi(p,X_p)=\left(x^1(p),\ldots,x^n(p),X^1(p),\ldots,X^n(p)\right)
,$$
donc, en prenant la carte $(\mathbf{R}^n,\Id_{\mathbf{R}^n})$ de $\mathbf{R}^n$, $X^1,\ldots,X^n$ sont de classe $\mathcal{C}^{k-1}$.
\end{rqe}
\newpage
\subsection{Application linéaire tangente}
Soit $\mathcal{M}$ et $\mathcal{N}$ deux variétés différentielles de classe $\mathcal{C}^k$ avec $k\geqslant 2$.
\begin{defn}
Soit $f:\mathcal{M}\to\mathcal{N}$ de classe $\mathcal{C}^1$.\newline Pour tout $p\in\mathcal{M}$, on pose $T_pf:T_p\mathcal{M}\to T_{f(p)}\mathcal{N},v\mapsto \left(\varphi\mapsto v(\varphi\circ f)\right)$ l'\textbf{application linéaire tangente de $f$ en $p$}.
\end{defn}
\begin{rqe}
Si $\mathcal{N}$ est un $\mathbf{R}$-espace vectoriel de dimension finie, alors, pour toutes applications $f:\mathcal{M}\to\mathcal{N}$ et $g:\mathcal{M}\to\mathcal{N}$ de classe $\mathcal{C}^1$, pour tout $\lambda\in\mathbf{R}$ :
$$T_p(\lambda f+g)=\lambda T_pf+T_pg.$$
\end{rqe}
\begin{rqe}
Lorsque $\mathcal{N}=\mathbf{R}$, on note plutôt $\mathrm{d}f(p)$ au lieu de $T_pf$.\newline
On a alors, en identifiant $T_{f(p)}\mathbf{R}$ et $\mathbf{R}$ : $\forall v\in T_p\mathcal{M}, \mathrm{d}f(p)(v)=v(f)$.\newline Notons que, dans une carte $(x^1,\ldots,x^n)$ définie en $p$ : $\mathrm{d}f(p)=\mathrm{d}x^j(p)\left.\dfrac{\partial f}{\partial x^j}\right|_{p}$.
\end{rqe}
\begin{rqe}
On se donne une carte $(x^1,\ldots,x^n)$ définie au voisinage du point $p$. et $(y^1,\ldots,y^p)$ une carte définie au voisinage du point $f(p)$. On a :
$$\forall v\in T_p\mathcal{M},\forall i\in\llbracket 1,p\rrbracket,T_pf(v)^i=v^j\left.\dfrac{\partial (y^i\circ f)}{\partial  x^j}\right|_{p}.$$
\end{rqe}
\begin{rqe}
Pour une application $\gamma:\mathbf{R}\to\mathcal{M}$ de classe $\mathcal{C}^1$, on note plutôt $\gamma'(t_0)$ au lieu de $T_{t_0}\gamma$.\newline
On a alors : $\forall\varphi\in\mathcal{C}^1(\mathcal{M},\mathbf{R}),\gamma'(t_0)(\varphi)=(\varphi\circ \gamma)'(t_0)$.\newline On remarque que $\gamma'(t_0)\in T_{\gamma(t_0)}\mathcal{M}$, et que ses coordonnées dans une carte $(x^1,\ldots,x^n)$ contenant $\gamma(t_0)$ sont $\gamma'(t_0)=(x^i\circ \gamma)'(t_0)\left.\dfrac{\partial}{\partial x^i}\right|_{\gamma(t_0)}$.\newline Étant donnés $p\in \mathcal{M}$ et $v\in T_{p}\mathcal{M}$, qu'on écrit $v=v^i\left.\dfrac{\partial}{\partial x^i}\right|_{p}$, en posant\newline $\gamma:t\mapsto y^{-1}((t-t_0)(v^1,\ldots,v^n)+x(p))$, on a : $\gamma(t_0)=p$ et $\gamma'(t_0)=v$.

Ainsi, $T_p\mathcal{M}$ est l'ensemble des vecteurs dérivés en $p$ des courbes tracées sur $\mathcal{M}$ passant par $p$.
\end{rqe}
\begin{prop}
Soit $\mathcal{P}$ une variété différentielle de classe $\mathcal{C}^k$ et deux applications $f:\mathcal{M}\to\mathcal{N}$ et $g:\mathcal{N}\to\mathcal{P}$ de classe $\mathcal{C}^1$. On a :
$$
\forall p\in\mathcal{M},T_p(g\circ f)=T_{f(p)}g\circ T_pf.
$$
\end{prop}
\begin{dem}
Soit $p$ un point de $\mathcal{M}$ et $v$ un vecteur tangent à $\mathcal{M}$ en $p$. On a :
$$
\forall \varphi\in\mathcal{C}^1(\mathcal{M},\mathbf{R}),T_p(g\circ f)(v)(\varphi)=v(\varphi(g\circ f))=T_{f(p)}g(T_pf(v))(\varphi).
$$
\end{dem}
\begin{defn}
Soit $\mathcal{M}$ et $\mathcal{N}$ deux variétés différentielles et $f:\mathcal{M}\to\mathcal{N}$ de classe $\mathcal{C}^1$. On définit l'\textbf{application tangente de $f$} par :
$$Tf:T\mathcal{M}\to T\mathcal{N},(p,v)\mapsto (f(p),T_pf(v)).$$
\end{defn}
\newpage
\section{Fibré cotangent}
\subsection{Espace cotangent en un point}
On se donne $\mathcal{M}$ une variété différentielle de classe $\mathcal{C}^k$ (avec $k\geqslant 2$) de dimension $n$ et $p$ un point de $\mathcal{M}$.
\begin{defn}
L'\textbf{espace cotangent à $\mathcal{M}$ en $p$}, noté $T_p^*\mathcal{M}$, est le dual de $T_p\mathcal{M}$.
\end{defn}
\begin{prop}
Soit $(x^1,\ldots,x^n)$ une carte sur $\mathcal{M}$ définie en $p$.\newline La famille $\left(\mathrm{d}x^i(p)\right)_{i\in\llbracket 1,n\rrbracket}$ est la base duale de $\left(\left.\dfrac{\partial}{\partial x^i}\right|_p\right)_{i\in\llbracket 1,n\rrbracket}$.
\end{prop}
\begin{dem}
Pour tout $i\in\llbracket 1,n\rrbracket$, $\forall v\in T_p\mathcal{M},\mathrm{d}x^i(p)v=v(x_i)$. On conclut en utilisant une remarque précédente.
\end{dem}
\begin{prop}
Soit $(x^1,\ldots,x^n),(y^1,\ldots,y^n)$ deux cartes définies au point $p$.\newline On a la relation de changement de base suivante :
$$
\forall i\in\llbracket 1,n\rrbracket, \mathrm{d}y^i(p)=\sum\limits_{j=1}^n\left.\dfrac{\partial y^i}{\partial x^j}\right|_p\mathrm{d}x^j(p).
$$
\end{prop}
\begin{dem}
Immédiat d'après la proposition précédente.
\end{dem}
\newpage
\subsection{Fibré cotangent}
Soit $\mathcal{M}$ une variété différentielle de classe $\mathcal{C}^k$ (avec $k\geqslant 2$).
\begin{defn}
On pose $\displaystyle T^*\mathcal{M}=\bigcup_{p\in\mathcal{M}}\{p\}\times T^*_p\mathcal{M}$ le \textbf{fibré cotangent de $\mathcal{M}$}.\newline On note $(U_i,\phi_i)_{i\in I}$ l'atlas de $\mathcal{M}$.\newline D'après le lemme concluant la première section de ce polycopié, $T^*\mathcal{M}$ est naturellement muni d'une topologie pour laquelle $(\mathcal{U}^*_i,\Phi^*_i)_{i\in I}$ avec $\mathcal{U}^*_i=\displaystyle\bigcup_{p\in U_i}\{p\}\times T^*_p\mathcal{M}$ et $$\begin{array}{ccccc}
\Phi^*_i & : & \mathcal{U}^*_i & \to & \mathbf{R}^{n}\times\mathbf{R}^n \\
 & & (p,\omega) & \mapsto & \left(\phi_i(p),\left(\mathrm{d}x_i^k(p)^*(\omega)\right)_{k\in\llbracket 1,n\rrbracket}\right) \\
\end{array}$$
(où on a écrit $\phi_i=(x_i^1,\ldots,x_i^n)$) est un atlas faisant de $T^*\mathcal{M}$ une variété différentielle de classe $\mathcal{C}^{k-1}$ et de dimension $2n$.
\end{defn}
\begin{just}
La justification est quasiment identique à celle écrite pour le fibré tangent.
\end{just}
\begin{prop}
La projection $\pi^*: T^*\mathcal{M}\to\mathcal{M},(p,\omega)\mapsto p$ est de classe $\mathcal{C}^{k-1}$.
\end{prop}
\begin{prop}
Le triplet $(T^*\mathcal{M},\mathcal{M},\pi^*)$ est un fibré vectoriel de classe $\mathcal{C}^{k-1}$ et de dimension $n$.
\end{prop}
\begin{defn}
Une section du fibré cotangent est appelée \textbf{$1$-forme différentielle}.
\end{defn}
\begin{rqe}
Soit $(x^1,\ldots,x^n)$ une carte sur $\mathcal{M}$ et $\omega$ une $1$-forme différentielle de classe $\mathcal{C}^{k-1}$, qu'on écrit localement $\omega=\omega_i\mathrm{d}x^i$. Les applications $\omega_i$ sont de classe $\mathcal{C}^{k-1}$.
\end{rqe}
\newpage
\section{Intermède : calcul tensoriel}
Le but de cette section est de présenter les outils usuels du calcul tensoriel, qui sera abondamment utilisé par la suite dans le cadre de la géométrie différentielle.

Dans cette section, on se donne $\mathbf{K}$ un corps.
\subsection{Applications multilinéaires}
\subsubsection{Généralités}
\begin{defn}
Soit $E_1,\ldots,E_p,F$ des $\mathbf{K}$-espaces vectoriels.

Une application $f:E_1\times\cdots\times E_p\to F$ est dite \textbf{$p$-linéaire} lorsque, pour tout $k\in\llbracket 1,p\rrbracket$, pour tout $(x_1,\ldots,x_{k-1},x_{k+1},\ldots,x_p)\in E_1\times\cdots\times E_{k-1}\times E_{k+1}\times\cdots\times E_p$, l'application
$$E_k\to F, x_k\mapsto f(x_1,\ldots,x_p)$$
est linéaire.

On note $\mathcal{L}^p(E_1\times\cdots\times E_p,F)$ l'ensemble des applications $p$-linéaires de $E_1\times\cdots\times E_p$ dans $F$.

Il s'agit d'un sous-espace vectoriel de $F^{E_1\times\cdots\times E_p}$.

Lorsque $E_1=\cdots=E_p=E$, on le note plutôt $\mathcal{L}^p(E,F)$.

Lorsque $F=\mathbf{K}$, on le note plutôt $\mathcal{L}^p(E_1\times\cdots\times E_p)$.

Lorsque $E_1=\cdots=E_p=E$ et $F=\mathbf{K}$, on le note plutôt $\mathcal{L}^p(E)$.
\end{defn}
\begin{prop}
Soit $E_1,\ldots, E_p,F$ des $\mathbf{K}$-espaces vectoriels.

On note $(e_{i_1})_{i_1\in I_1},\ldots,(e_{i_p})_{i_p\in I_p}$ des bases respectives de $E_1,\ldots,E_p$. L'application 
$$
\mathcal{L}^p(E_1\times\cdots\times E_p,F)\to F^{I_1\times\cdots\times I_p},f\mapsto \left(f\left(e_{i_1},\ldots,e_{i_p}\right)\right)_{(i_1,\ldots,i_p)\in I_1\times\cdots\times I_p}
$$
est un isomorphisme.
\end{prop}
\begin{rqe}
En particulier, si $E_1,\ldots,E_n$ sont de dimension finie, alors :
$$\dim\left(\mathcal{L}^p(E_1\times\cdots\times E_p,F)\right)=\dim(F)\prod\limits_{k=1}^p\dim(E_k).$$
\end{rqe}
\newpage
\subsubsection{Applications multilinéaires symétriques}
On se donne $E$ et $F$ deux $\mathbf{K}$-espaces vectoriels et $p\in\mathbf{N}$.
\begin{defn}
Une application $f\in\mathcal{L}^p(E,F)$ est dite \textbf{symétrique} si, pour tout $(x_1,\ldots,x_p)\in E^p$, pour toute permutation $\sigma\in\mathfrak{S}_p$ :
$$
f\left(x_{\sigma (1)},\ldots,x_{\sigma(p)}\right)=f(x_1,\ldots,x_p).
$$
On note $\mathcal{S}^p(E,F)$ l'ensemble des applications $p$-linéaires symétriques de $E^p$ dans $F$.

Il s'agit d'un sous-espace vectoriel de $\mathcal{L}^p(E,F)$.

Lorsque $F=\mathbf{K}$, on le note plus simplement $\mathcal{S}^p(E)$.
\end{defn}
\begin{prop}
On suppose $E$ de dimension finie.

On se donne $(e_1,\ldots,e_n)$ une base de $E$.

L'application 
$$
\mathcal{S}^p(E,F)\to F^{\mathcal{C}},f\mapsto \left(f\left(e_{i_1},\ldots,e_{i_p}\right)\right)_{(i_1,\ldots,i_p)\in\mathcal{C}}
$$ est un isomorphisme (où $\mathcal{C}$ désigne l'ensemble des $p$-uplets croissants à valeurs dans $\llbracket 1,n\rrbracket$).
\end{prop}
\begin{rqe}
Dans le cas où $E$ et $F$ sont de dimension finie, on a :
$$\dim(\mathcal{S}^p(E,F))=\dim(F)\binom{\dim(E)+p-1}{p}.$$
\end{rqe}
\newpage
\subsubsection{Applications multilinéaires alternées et antisymétriques}
On se donne $E$ et $F$ deux $\mathbf{K}$-espaces vectoriels et $p\in\mathbf{N}$.
\begin{defn}
Une application $f\in\mathcal{L}^p(E_F)$ est dite \textbf{alternée} si, pour tout

$x_1,\ldots,x_p\in E$ tels qu'il existe $k,l\in\llbracket 1,p\rrbracket$ distincts de sorte que $x_k=x_l$,
$$f(x_1,\ldots,x_p)=0_F.$$
On note $\mathcal{A}^p(E,F)$ l'ensemble des applications $p$-linéaires alternées de $E^p$ dans $F$.

Il s'agit d'un sous-espace vectoriel de $\mathcal{L}^p(E,F)$.

Lorsque $F=\mathbf{K}$, on le note plus simplement $\mathcal{A}^p(E)$.
\end{defn}
\begin{defn}
Une application $f\in\mathcal{L}^p(E,F)$ est dite \textbf{antisymétrique} si, pour tout $(x_1,\ldots,x_p)\in E^p$, pour toute permutation $\sigma\in\mathfrak{S}_p$ :
$$
f\left(x_{\sigma(1)},\ldots,x_{\sigma(p)}\right)=\varepsilon(\sigma)f(x_1,\ldots,x_p).
$$
L'ensemble des applications $p$-linéaires antisymétriques de $E^p$ dans $F$ est un sous-espace vectoriel de $\mathcal{L}^p(E,F)$.
\end{defn}
\begin{prop}
Toute application $p$-linéaire alternée est antisymétrique.

Si $2\cdot 1_\mathbf{K}\neq 0_\mathbf{K}$, alors toute application $p$-linéaire antisymétrique est alternée.
\end{prop}
\begin{prop}
On suppose $E$ de dimension finie.

On se donne $(e_1,\ldots,e_n)$ une base de $E$.

L'application
$$
\mathcal{A}^p(E,F)\to F^{\mathcal{C}_+},f\mapsto\left(f\left(e_{i_1},\ldots,e_{i_p}\right)\right)_{(i_1,\ldots,i_p)\in\mathcal{C}_+}
$$
est un isomorphisme (où $\mathcal{C}_+$ désigne l'ensemble des $p$-uplets strictement croissants à valeurs dans $\llbracket 1,n\rrbracket$).
\end{prop}
\begin{rqe}
En particulier, si $E$ et $F$ sont de dimension finie, alors :
$$
\dim\left(\mathcal{A}^p(E,F)\right)=\dim(F)\binom{\dim(E)}{p}.
$$
\end{rqe}
\newpage
\subsection{Produit tensoriel d'espaces vectoriels de dimensions finies}
\subsubsection{Généralités}
\begin{defn}
Soit $E_1,\ldots,E_p$ des $\mathbf{K}$-espaces vectoriels de dimensions finies.

On note $E_1\otimes\cdots\otimes E_p=\mathcal{L}^p\left(E_1^*\times\cdots\times E_p^*\right)$
\end{defn}
\begin{rqe}
Pour $p=1$, on confond $E$ et $E^{**}$ en identifiant $x\in E$ à
$$E^*\to\mathbf{K},\varphi\mapsto\varphi(x).$$
\end{rqe}
\begin{defn}
Soit $E_1,\ldots,E_p,F_1,\ldots,F_q$ des $\mathbf{K}$-espaces vectoriels.

Étant donnés $x\in E_1\otimes\cdots\otimes E_p$ et $y\in F_1\otimes\cdots\otimes F_q$, on définit
$$x\otimes y:E_1\times \cdots\times E_p\times F_1\times\cdots\times F_q\to\mathbf{K}$$
par : pour tout $(\varphi_1,\ldots,\varphi_p,\psi_1,\ldots,\psi_q)\in E_1^*\times\cdots\times E_p^*\times F_1^*\times\cdots\times F_q^*$,
$$
x\otimes y(\varphi_1,\ldots,\varphi_p,\psi_1,\ldots,\psi_q)= x(\varphi_1,\ldots,\varphi_p)y(\psi_1,\ldots,\psi_q).
$$
On vérifie facilement que $x\otimes y\in E_1\otimes\cdots\otimes E_p\otimes F_1\otimes\cdots\otimes F_q$.
\end{defn}
\begin{ex}
Soit $E_1$ et $E_2$ deux $\mathbf{K}$-espaces vectoriels de dimensions finies.

Soit $x\in E_1$ et $y\in E_2$. On a:
$$\forall (\varphi,\psi)\in E_1^*\times E_2^*, x\otimes y(\varphi,\psi)=\varphi(x)\psi(y).$$

Soit $\varphi\in E_1^*$ et $\psi\in E_2^*$. En identifiant $E$ et $E^{**}$, on a :
$$\forall (x,y)\in E_1\times E_2,\varphi\otimes\psi(x,y)=\varphi(x)\psi(y).$$

Soit $x\in E_1$ et $\psi\in E_2$. On a :
$$\forall (\varphi,y)\in E_1^*\times E_2,x\otimes\psi(\varphi,y)=\varphi(x)\psi(y).$$
\end{ex}
\begin{prop}
Soit $E_1,\ldots,E_p,F_1,\ldots,F_q$ des $\mathbf{K}$-espaces vectoriels de dimensions finies. L'application
$$E_1\otimes\cdots\otimes E_p\times F_1\otimes\cdots\otimes F_q\to E_1\otimes E_p\otimes F_1\otimes\cdots\otimes F_q,(x,y)\mapsto x\otimes y$$
est bilinéaire.
\end{prop}
\newpage
\begin{prop}
Soit $E_1,\ldots,E_p,F_1,\ldots,F_q,G_1,\ldots,G_r$ des $\mathbf{K}$-espaces vectoriels de dimension finie.

Soit $x\in E_1\otimes\cdots\otimes E_p,y\in F_1\otimes\cdots\otimes F_q,z\in G_1\otimes \cdots\otimes G_r$. On a :
$$x\otimes (y\otimes z)=(x\otimes y)\otimes z.$$
\end{prop}
\begin{prop}
Soit $E_1,\ldots,E_p$ des $\mathbf{K}$-espaces vectoriels de dimension finie.

Pour tout $k\in\llbracket 1,p\rrbracket$, on se donne $(e_{i_k})_{i_k\in\llbracket 1,n_k\rrbracket}$ une base de $E_k$.

La famille $\left(e_{i_1}\otimes\cdots\otimes e_{i_p}\right)_{(i_1,\ldots,i_p)\in\llbracket 1,n_1\rrbracket\times\cdots\times\llbracket 1,n_p\rrbracket}$ est une base de $E_1\otimes\cdots\otimes E_p$.
\end{prop}
\begin{prop}
Pour tous $\mathbf{K}$-espaces vectoriels de dimension finie $E_1,\ldots,E_p,F$, pour toute application $f\in\mathcal{L}^p\left(E_1\times\cdots\times E_p,F\right)$, il existe une unique application $\overline{f}\in\mathcal{L}\left(E_1\otimes\cdots\otimes E_p,F\right)$ telle que :
$$\forall (x_1,\ldots,x_p)\in E_1\times\cdots\times E_p, f(x_1,\ldots,x_p)=\overline{f}(x_1\otimes\cdots\otimes x_p).$$
\end{prop}
\subsubsection{Tenseurs covariants et contravariants}
Dans cette section, on se donne $E$ un $\mathbf{K}$-espace vectoriel de dimension finie.
\begin{defn}
Soit $p,q\in\mathbf{N}$. On pose $E^{\otimes p}\otimes (E^*)^{\otimes q}=E\otimes\cdots\otimes E\otimes E^*\otimes\cdots\otimes E^*$ (avec $p$ occurrences de $E$ et $q$ occurrences de $E^*$). Ses éléments sont appelés \textbf{tenseurs d'ordre $(p,q)$} ou \textbf{tenseurs $p$-contravariants et $q$-covariants}.
\end{defn}
Tous les tenseurs suivants seront pris sur $E$.
\begin{rqe}
Le produit tensoriel d'un tenseur de type $(p,q)$ et d'un tenseur de type $(r,s)$ est un tenseur de type $(p+r,q+s)$.
\end{rqe}
Étant donnée une base $(e_1,\ldots,e_n)$ de $E$, on notera $\left(e^1,\ldots,e^n\right)$ sa base duale.\newline Pour plus de lisibilité, on utilisera la convention de sommation sur les indices répétés : si une lettre apparaît en indice et en exposant, on somme implicitement sur elle.\newline Par exemple, $x^ie_i$ désigne $\displaystyle\sum\limits_{i=1}^nx^ie_i$.
\begin{defn}
Soit $p,q\in\mathbf{N}$, $T$ un tenseur de type $(p,q)$ et $(e_1,\ldots,e_n)$ une base de $E$. On note $T_{\;\;\qquad j_1,\ldots,j_q}^{i_1,\ldots,i_p}$ les composantes de $T$ dans
la base de $E^{\otimes p}\otimes (E^*)^{\otimes q}$

$\left(e_{i_1}\otimes\cdots\otimes e_{i_p}\otimes e^{j_1}\otimes\cdots\otimes e^{j_q}\right)_{(i_1,\ldots,i_p,j_1,\ldots,j_q)\in\llbracket 1,n\rrbracket^{p+q}}$, de sorte que, avec la convention d'\bsc{Einstein} :
$$T=T_{\;\;\qquad j_1,\ldots,j_q}^{i_1,\ldots,i_p}e_{i_1}\otimes\cdots\otimes e_{i_p}\otimes e^{j_1}\otimes\cdots\otimes e^{j_q}.$$
Dans l'expression $T_{\;\;\qquad j_1,\ldots,j_q}^{i_1,\ldots,i_p}$, les indices supérieurs $i_1,\ldots,i_p$ sont appelés \textbf{indices de contravariance}, tandis que les indices inférieurs $j_1,\ldots,j_q$ sont appelés \textbf{indices de covariance}.
\end{defn}
\begin{prop}
Soit $p,q,r,s\in\mathbf{N}$, $T$ un tenseur de type $(p,q)$, $S$ un tenseur de type $(r,s)$ et $(e_1,\ldots,e_n)$ une base de $E$ (dans laquelle seront prises les coordonnées tensorielles suivantes). On a, pour tout $i_1,\ldots,i_p,i'_1,\ldots,i'_r,j_1,\ldots,j_q,j'_1,\ldots,j'_s\in\llbracket 1,n\rrbracket$ :
$$
(T\otimes S)^{i_1,\ldots,i_p,i'_1,\ldots,i'_r}_{\;\;\qquad\;\;\qquad j_1,\ldots,j_q,j'_1,\ldots,j'_s}=T^{i_1,\ldots,i_p}_{\;\;\qquad j_1,\ldots,j_q}S^{i_1',\ldots,i_r'}_{\;\;\qquad j'_1,\ldots,j'_s}
$$
\end{prop}
\begin{prop}
Soit $p,q\in\mathbf{N}$, $T$ un tenseur de type $(p,q)$, $\mathcal{B}$ et $\mathcal{B}'$ deux bases de $E$. On note $T^{i_1,\ldots,i_p}_{\;\;\qquad j_1,\ldots,j_q}$ et $T'^{i_1,\ldots,i_p}_{\;\;\qquad j_1,\ldots,j_q}$ les coordonnées de $T$ dans les bases tensorielles associées à $\mathcal{B}$ et $\mathcal{B}'$ respectivement.

On note $\left(P_{j}^i\right)_{(i,j)\in\llbracket 1,n\rrbracket^2}$ la matrice de passage de $\mathcal{B}$ vers $\mathcal{B}'$.

On a :
$$
T'^{i'_1,\ldots,i'_p}_{\;\;\qquad j'_1,\ldots,j'_q}=\left(P^{-1}\right)^{i'_1}_{i_1}\cdots\left(P^{-1}\right)^{i'_p}_{i_p}P^{j_1}_{j'_1}\cdots P^{j_q}_{j'_q}T^{i_1,\ldots,i_p}_{\;\;\qquad j_1,\ldots,j_q}.
$$
\end{prop}
\begin{rqe}
Cette formule justifie les expression \og covariant \fg{} et \og contravariant \fg{}.

En effet, lors du changement de base $\mathcal{B}\to\mathcal{B}'$, les indices covariants sont associés à $P$ tandis que les indices contravariants sont associés à $P^{-1}$.

Plus généralement, une quantité qui est transformée en $P$ est dite \textbf{covariante}.

C'est le cas des vecteurs de base car $\forall i\in\llbracket 1,n\rrbracket,e'_i=P_i^je_j$, mais aussi des coordonnées dans la base duale car $\forall\varphi\in E^*,\forall i\in\llbracket 1,n\rrbracket,\varphi_i=P_j^ie^j$.

Une quantité qui est transformée en $P^{-1}$ est dite \textbf{contravariante}.

C'est le cas des vecteurs de le base duale car $\forall i\in\llbracket 1,n\rrbracket,e^i=\left(P^{-1}\right)^i_je^j$ et des coordonnées dans la base car $\forall x\in E, \forall i\in\llbracket 1,n\rrbracket, x^i=\left(P^{-1}\right)^i_jx^j$.
\end{rqe}
On termine par la généralisation de la notion de trace : la contraction.
\begin{defn}
Soit $p,q\in\mathbf{N}^*$, $T$ un tenseur de type $(p,q)$ et $(e_1,\ldots,e_n)$ une base de $E$. Étant donnés $(k,l)\in\llbracket 1,p\rrbracket\times\llbracket 1,q\rrbracket$, on définit le tenseur $C^k_lT$ de type $(p-1,q-1)$ par, pour tous $\varphi_1,\ldots,\varphi_{k-1},\varphi_{k+1},\ldots,\varphi_p\in E^*$, pour tous $x_1,\ldots,x_{l-1},x_{l+1},\ldots,x_q\in E$,
\begin{align*}
&C^k_lT(\varphi_1,\ldots,\varphi_{k-1},\varphi_{k+1},\ldots,\varphi_p,x_1,\ldots,x_{l-1},x_{l+1},\ldots,x_q)\\&=\sum\limits_{m=1}^nT(\varphi_1,\ldots,\varphi_{k-1},e^m,\varphi_{k+1},\ldots,\varphi_p,x_1,\ldots,x_{l-1},e_m,x_{l+1},\ldots,x_q).
\end{align*}
Cette définition est indépendante du choix de la base $(e_1,\ldots,e_n)$ de $E$.
\end{defn}
\begin{rqe}
Dans le cas d'un tenseur de type $(1,1)$, on retrouve la trace d'une application linéaire.
\end{rqe}
\begin{rqe}
Si $T_{\;\;\qquad j_1,\ldots,j_q}^{i_1,\ldots,i_p}$ sont les coordonnées de $T$, les coordonnées de $C^k_lT$ sont $T_{\;\;\quad\qquad\qquad\qquad j_1,\ldots,j_{l-1},m,j_{l+1},\ldots,j_q}^{i_1,\ldots,i_{k-1},m,i_{k+1},\ldots,i_p}$
\end{rqe}
\begin{prop}
La contraction est linéaire : pour tous tenseurs $T,S$ de type $(p,q)$, pour tout $(k,l)\in\llbracket 1,p\rrbracket\times\llbracket 1,q\rrbracket$, pour tout $\lambda\in\mathbf{K}$ :
$$
C^k_l(\lambda T+S)=\lambda C^k_lT+C^k_lS.
$$
\end{prop}
\newpage
\subsubsection{Tenseur métrique}
Dans ce paragraphe, on se donne $E$ un $\mathbf{K}$-espace vectoriel de dimension finie.
\begin{defn}
Un \textbf{tenseur métrique} sur $E$ est une forme bilinéaire symétrique non dégénérée sur $E$.
\end{defn}
Dans toute la suite de ce paragraphe, on se donne $g$ un tenseur métrique défini sur $E^2$. Notons qu'il s'agit d'un tenseur de type $(0,2)$.

On rappelle que $E$ et $E^*$ peuvent être confondus en identifiant $x\in E$ à $g(x,\cdot)\in E^*$.

On munit ainsi $E^*$ d'une structure d'espace euclidien, de produit scalaire noté $g^{-1}$.

\medskip

Dans toute la suite, on se donne $(e_1,\ldots,e_n)$ une base de $E$.

\medskip

Si $\left(g_{i,j}\right)_{(i,j)\in\llbracket 1,n\rrbracket^2}$ désigne la famille des coordonnées de $g$ dans la base tensorielle associée $(e_1,\ldots,e_n)$, on notera $\left(g^{i,j}\right)_{(i,j)\in\llbracket 1,n\rrbracket^2}$ les coordonnées de $g^{-1}$ dans la base tensorielle associée à $\left(e^1,\ldots,e^n\right)$.

Notons que $\forall i,j\in\llbracket 1,n\rrbracket, g_{i,k}g^{k,j}=g^{i,k}g_{k,j}=\delta_{i,j}$.

Pour cette raison, $g^{-1}$ est appelé \textbf{tenseur métrique inverse}.
\begin{defn}
Soit $x\in E$.

Les \textbf{coordonnées contravariantes} de $x$, notées $\left(x^i\right)_{i\in\llbracket 1,n\rrbracket}$, sont les coordonnées de $x$ dans $(e_1,\ldots,e_n)$.

Les \textbf{coordonnées covariantes} de $x$, notée $(x_j)_{j\in\llbracket 1,n\rrbracket}$, sont les coordonnées de $g(x,\cdot)$ dans $\left(e^1,\ldots,e^n\right)$.
\end{defn}
\begin{rqe}
Si $(e_1,\ldots,e_n)$ est une base orthonormée de $E$, alors :$$\forall i\in\llbracket 1,n\rrbracket, x^i=x_i.$$
\end{rqe}
\begin{prop}
Soit $(e_1,\ldots,e_n)$ une base de $E$ et $x\in E$. On a :
$$
\forall j\in\llbracket 1,n\rrbracket, x_j=g_{i,j}x^i\text{ et }\forall i\in\llbracket 1,n\rrbracket, x^i=g^{i,j}x_j.
$$
\end{prop}
\begin{prop}
Soit $p,q\in\mathbf{N}$ et $T$ un tenseur de type $(p,q)$.

On peut identifier $T$ avec le tenseur $\widetilde{T}$ de type $(0,p+q)$ défini par : pour tous $x_1,\ldots,x_p,y_1,\ldots,y_q\in E$, $$
\widetilde{T}(x_1,\ldots,x_p,y_1,\ldots,y_q)=T\left(g(x_1,\cdot),\ldots,g(x_p,\cdot),y_1,\ldots,y_q\right).
$$
On a ainsi :
$$
T_{i_1,\ldots,i_p,j_1,\ldots,j_q}=g_{i_1,i_1}\cdots g_{i_p,i_p'}T^{i'_1,\ldots,i'_p}_{\;\;\qquad j_1,\ldots,j_q}.
$$
\end{prop}
\begin{rqe}
Ainsi, les coordonnées du tenseur métrique permettent de \og baisser \fg{} les indices de contravariance pour en faire des indices de covariance. 
\end{rqe}
\begin{prop}
Soit $p,q\in\mathbf{N}$ et $T$ un tenseur de type $(p,q)$.

On peut identifier $T$ avec le tenseur $\widetilde{T}$ de type $(p+q,0)$ défini par :

pour tout $\varphi_1,\ldots,\varphi_p\in E^*$, pour tout $x_1,\ldots,x_q\in E$ :
$$\widetilde{T}(\varphi_1,\ldots,\varphi_p,g(x_1,\cdot),\ldots,g(x_q,\cdot))=T\left(\varphi_1,\ldots,\varphi_p,x_1,\ldots,x_q\right).
$$
On a ainsi :
$$
T^{i_1,\ldots,i_p,j_1,\ldots,j_q}=g^{j_1,j_1'}\cdots g^{j_q,j_q'}T_{\;\;\qquad j_1',\ldots,j_q'}^{i_1,\ldots,i_p}.
$$
\end{prop}
\begin{rqe}
Ainsi, les coordonnées du tenseur métrique inverse permettent de \og monter \fg{} les indices de covariance pour en faire des indices de contravariance.
\end{rqe}
\begin{rqe}
Il est possible de mélanger les deux résultats ci-dessus, par exemple :
$$T^{i\,\;k}_{\,\;j}=g^{k,k'}g_{j,j'}T^{i,j'}_{\quad k'}$$
\end{rqe}
\begin{rqe}
Si $(e_1,\ldots,e_n)$ est orthonormée, alors \og baisser \fg{} ou \og monter \fg{} les indices de contravariance ou de covariance ne change pas les coordonnées du tenseur.  
\end{rqe}
\begin{rqe}
Comme le tenseur métrique et son inverse permettent de \og monter \fg{} ou \og baisser \fg{} les indices, on peut contracter un tenseur selon n'importe quelle paire d'indices, même s'ils sont de même nature. Par exemple, si $T$ est un tenseur de type $(4,3)$, on peut définir le tenseur $C^{1,3}T$ de type $(2,3)$ par : $\left(C^{1,3}T\right)_{\;\;\quad j_1,j_2,j_3}^{i_1,i_2}=T_{\quad\;\;k\;\;\; j_1,j_2,j_3}^{k,i_1\;\;i_2}$.\newline On peut aussi définir le tenseur $C_{2,3}T$ de type $(4,1)$ par : $\left(C_{2,3}T\right)_{\quad\qquad j_1}^{i_1,i_2,i_3,i_4}=T_{\quad\qquad j_1,k}^{i_1,i_2,i_3,i_4\quad k}$.\newline Notons que, au sein de la paire d'indices contractés, mettre le premier en haut et le second en bas ou faire l'inverse ne change pas le résultat.
\end{rqe}
\newpage
\subsection{Puissances extérieures d'un espace vectoriel de dimension finie}
Dans cette section, on se donne $E$ un $\mathbf{K}$-espace vectoriel de dimension finie et on suppose que $\mathbf{K}$ est de caractéristique nulle.
\begin{defn}
Soit $p\in\mathbf{N}$.

On appelle \textbf{$p$-ième puissance extérieure de $E$} l'espace vectoriel $\mathcal{A}^p(E^*)$.

On la note $\Lambda^pE$.
\end{defn}
\begin{rqe}
On a $\Lambda^0E=\mathbf{K}$ et, en identifiant $E$ et $E^{**}$ : $\Lambda^1E=E$.
\end{rqe}
\begin{defn}
Soit $p,q\in\mathbf{N}$, $\alpha\in\Lambda^pE$ et $\beta\in\Lambda^qE$.

On définit $\alpha\wedge\beta:(E^*)^{p+q}\to\mathbf{K}$ par : pour tout $\varphi_1,\ldots,\varphi_{p+q}\in E^*$,
$$\alpha\wedge\beta(\varphi_1,\ldots,\varphi_{p+q})=\dfrac{1}{p!q!}\sum\limits_{\sigma\in\mathfrak{S}_{p+q}}\varepsilon(\sigma)\alpha\left(\varphi_{\sigma(1)},\ldots,\varphi_{\sigma(p)}\right)\beta\left(\varphi_{\sigma(p+1)},\ldots,\varphi_{\sigma(p+q)}\right).$$
\end{defn}
\begin{rqe}
Il s'agit de l'antisymétrisé de $\alpha\otimes\beta$.
\end{rqe}
\begin{ex}
Soit $x,y\in E$. On a : $\forall\varphi,\psi\in E^*, x\wedge y(\varphi,\psi)=\varphi(x)\psi(y)-\varphi(y)\psi(x)$.
\end{ex}
\begin{prop}
Soit $p,q\in\mathbf{N}$.

L'application $\Lambda^pE\times\Lambda^qE\to\Lambda^{p+q}E,(\alpha,\beta)\mapsto\alpha\wedge\beta$ est bilinéaire.
\end{prop}
\begin{prop}
Soit $p,q,r\in\mathbf{N}$. On a :
\begin{enumerate}[label=\roman*)]
\item $\forall \alpha\in\Lambda^pE, \forall \beta\in\Lambda^qE,\forall\gamma\in\Lambda^rE,\alpha\wedge(\beta\wedge\gamma)=(\alpha\wedge\beta)\wedge\gamma$.
\item $\forall \alpha\in\Lambda^pE,\forall \beta\in\Lambda^qE, \alpha\wedge\beta=(-1)^{pq}\beta\wedge\alpha$.
\end{enumerate}
\end{prop}
\begin{prop}
Soit $x_1,\ldots,x_p\in E$.

La famille $(x_1,\ldots,x_p)$ est libre si et seulement si $x_1\wedge\cdots\wedge x_p\neq 0_{\Lambda^pE}$.
\end{prop}
\begin{prop}
Soit $p\in\mathbf{N}$ et $(e_1,\ldots,e_n)$ une base de $E$.

La famille $\left(e_{i_1}\wedge\cdots\wedge e_{i_p}\right)_{1\leqslant i_1<\cdots<i_p\leqslant n}$ est une base de $\Lambda^pE$.
\end{prop}
\begin{prop}
Soit $p\in\mathbf{N}$. Pour tout $\mathbf{K}$-espace vectoriel de dimension finie $F$, pour toute application $f\in\mathcal{A}^p(E,F)$, il existe une unique application $\overline{f}\in\mathcal{L}(\Lambda^pE,F)$ telle que :
$$\forall x_1,\ldots,x_p\in E,f(x_1,\ldots,x_p)=\overline{f}(x_1\wedge\cdots\wedge x_p).$$
\end{prop}
\newpage
\section{Fibrés tensoriels}
\subsection{Espaces tensoriels en un point}
On se donne $\mathcal{M}$ une variété différentielle de classe $\mathcal{C}^k$ (avec $k\geqslant 2$) de dimension $n$ et $p$ un point de $\mathcal{M}$.
\begin{defn}
La \textbf{fibre tensorielle de type $(r,s)$ de $\mathcal{M}$ au-dessus de $p$} est définie par :
$$
(T^r_s)_p\mathcal{M}=\left(T_p\mathcal{M}^{\otimes r}\otimes T_p^*\mathcal{M}^{\otimes s}\right).
$$
\end{defn}
\begin{rqe}
Soit $(x^1,\ldots,x^n)$ une carte sur $\mathcal{M}$ définie en $p$. La famille $$\left(\left.\dfrac{\partial}{\partial x^{i_1}}\right|_p\otimes\cdots\otimes\left.\dfrac{\partial}{\partial x^{i_r}}\right|_p\otimes\mathrm{d}x^{j_1}(p)\otimes\cdots\otimes\mathrm{d}x^{j_s}(p)\right)_{(i_1,\ldots,i_r,j_1,\ldots,j_s)\in\llbracket 1,n\rrbracket^{r+s}}$$ est une base de $(T^r_s)_p\mathcal{M}$.
\end{rqe}
\begin{prop}
Soit $(x^1,\ldots,x^n),(y^1,\ldots,y^n)$ deux cartes définies au point $p$.\newline On a la relation de changement de base suivante :
\begin{align*}
&\left.\dfrac{\partial}{\partial y^{i'_1}}\right|_p\otimes\cdots\otimes\left.\dfrac{\partial}{\partial y^{i'_r}}\right|_p\otimes\mathrm{d}y^{j'_1}(p)\otimes\cdots\otimes\mathrm{d}y^{j'_s}(p)\\&=\left.\dfrac{\partial y^{i_1}}{\partial x^{i'_1}}\right|_p\cdots\left.\dfrac{\partial y^{i_r}}{\partial x^{i'_r}}\right|_p\left.\dfrac{\partial y^{j'_1}}{\partial x^{j_1}}\right|_p\cdots\left.\dfrac{\partial y^{j'_s}}{\partial x^{j_s}}\right|_p\left.\dfrac{\partial}{\partial x^{i_1}}\right|_p\otimes\cdots\otimes\left.\dfrac{\partial}{\partial x^{i_r}}\right|_p\mathrm{d}x^{j_1}(p)\otimes\cdots\otimes\mathrm{d}x^{j_s}(p).
\end{align*}
\end{prop}
\newpage
\subsection{Fibrés tensoriels}
Soit $\mathcal{M}$ une variété différentielle de classe $\mathcal{C}^k$ (avec $k\geqslant 2$) et $r,s\in\mathbf{N}$.
\begin{defn}
On pose $\displaystyle T^r_s\mathcal{M}=\bigcup_{p\in\mathcal{M}}\{p\}\times (T^r_s)_p\mathcal{M}$ le \textbf{fibré tensoriel de type $(r,s)$ de $\mathcal{M}$}.\newline
Étant donnée une carte $(U,(x^1,\ldots,x^n))$ de la variété $\mathcal{M}$, on définit sur $\displaystyle\bigcup_{p\in U}\{p\}\times (T^r_s)_p\mathcal{M}$ la carte renvoyant le $n+(r+s)n$-uplet des coordonnées d'un point $p$ et d'un tenseur de $(T^r_s)_p\mathcal{M}$ dans la base données précédemment.\newline

On munit ainsi $T^r_s\mathcal{M}$ d'une structure de variété différentielle de classe $\mathcal{C}^{k-1}$ et de dimension $n+(r+s)n$.
\end{defn}
\begin{just}
La justification est quasiment identique à celle écrite pour le fibré tangent.
\end{just}
\begin{prop}
Le triplet $(T^r_s\mathcal{M},\mathcal{M},\pi)$ est un fibré vectoriel de classe $\mathcal{C}^{k-1}$ et de dimension $(r+s)n$.
\end{prop}
\begin{defn}
Une section du fibré $T^r_s\mathcal{M}$ est appelée \textbf{champ tensoriel de type $(r,s)$}.
\end{defn}
\begin{rqe}
Soit $(x^1,\ldots,x^n)$ une carte sur $\mathcal{M}$ et $T$ un champ tensoriel de type $(r,s)$ de classe $\mathcal{C}^{k-1}$, qu'on écrit localement
$$T=T_{\;\;\qquad j_1,\ldots,j_s}^{i_1,\ldots,i_r}\dfrac{\partial}{\partial x^{i_1}}\otimes\cdots\otimes\dfrac{\partial}{\partial x^{i_r}}\mathrm{d}x^{j_1}\otimes\cdots\otimes\mathrm{d}x^{j_s}.$$
Les applications $T_{\;\;\qquad j_1,\ldots,j_s}^{i_1,\ldots,i_r}$ sont de classe $\mathcal{C}^{k-1}$.
\end{rqe}
\subsection{Formes différentielles}
Soit $\mathcal{M}$ une variété différentielle de classe $\mathcal{C}^k$ (avec $k\geqslant 2$) et $m\in\mathbf{N}$.
\begin{defn}
On note $\Omega^m_p\mathcal{M}$ la $m$-ième puissance extérieure de $T_p\mathcal{M}$.
\end{defn}
\begin{rqe}
Étant donnée $(x^1,\ldots,x^n)$ une carte définie au voisinage de $p$, la famille $\left(\mathrm{d}x^{i_1}(p)\wedge\cdots\wedge\mathrm{d}x^{i_m}(p)\right)_{1\leqslant i_1<\cdots<i_m\leqslant n}$ est une base de $\Omega^m_p\mathcal{M}$.
\end{rqe}
\begin{rqe}
On munit naturellement $\displaystyle\Omega^m\mathcal{M}=\bigcup_{p\in\mathcal{M}}\{p\}\times\Omega^m_p\mathcal{M}$ d'une structure de variété différentielle et de fibré vectoriel.
\end{rqe}
\begin{defn}
Une \textbf{$m$-forme différentielle} est une section du fibré $\Omega^m\mathcal{M}$.\newline On note $\Omega^m\mathcal{M}$ le $\mathbf{R}$-espace vectoriel des $m$-formes différentielles sur $\mathcal{M}$.
\end{defn}
\begin{defn}
On note $\displaystyle\Omega\mathcal{M}=\bigoplus_{m\in\mathbf{N}}\Omega^m\mathcal{M}$ le $\mathbf{R}$-espace vectoriel des formes différentielles définies sur $\mathcal{M}$. Muni de $\wedge$, il s'agit d'une algèbre graduée.
\end{defn}
\newpage
\section{Champs et dérivée de Lie}
On se donne $\mathcal{M}$ une variété différentielle lisse.
\subsection{Flot d'un champ vectoriel}
\begin{defn}
Soit $p$ un point de $\mathcal{M}$ et $X:U\to T\mathcal{M}$ un champ vectoriel lisse.\newline Il existe un unique chemin maximal $(I_p,\gamma)$ lisse tel que :
$$\left \{
\begin{array}{rcl}
\forall t\in I_p,\dot{\gamma}(t)&=&X_{\gamma(t)} \\
\gamma(0)&=&p
\end{array}
\right..$$
Le \textbf{flot de $X$} est l'application lisse définie sur $\displaystyle\bigcup_{p\in U}I_p\times\{p\}$ (qui est un ouvert de $\mathcal{M}$), qui à $(t,p)$ associe la valeur en $t$ du chemin $\gamma$ vérifiant les deux équations précédentes.
\end{defn}
\begin{dem}
Il suffit d'utiliser un système de coordonnées et d'appliquer le théorème de \bsc{Cauchy}-\bsc{Lipschitz}.
\end{dem}
\begin{prop}
Soit $X:U\to\mathcal{M}$ un champ vectoriel lisse. Le flot $\phi$ est lisse.\newline De plus, si on note, pour tout $t$, $\phi_t:p\mapsto\phi(t,p)$, pour tout $s$, pour tout $p\in U\cap\phi_{-s}(U)$ : $$\phi_t\circ\phi_s(p)=\phi_{s+t}(p).$$
\end{prop}
\begin{dem}
C'est une conséquence du théorème de \bsc{Cauchy}-\bsc{Lipschitz}.
\end{dem}
\begin{rqe}
La dernière égalité fait qu'on appelle parfois $\phi$ le \textbf{groupe à un paramètre} associé à $X$.
\end{rqe}
\newpage
\subsection{Tiré en arrière, poussé en avant}
Commençons par une digression d'algèbre linéaire. Étant donnés deux $\mathbf{R}$-espaces vectoriels $E,F$ isomorphes, $u:E\to F$ un isomorphisme et $\varphi\in F^*$, on souhaite interpréter $\varphi$ comme une forme linéaire sur $E$. Il n'y a en général pas de façon canonique (ie dépendant seulement de $E,F$ et $\varphi$) de procéder.\newline En utilisant $u$, on peut \og tirer en arrière \fg{} $\varphi$ pour obtenir la forme linéaire $\varphi\circ u$ définie sur $E$. Notons qu'on retrouve la notion d'application transposée : $u^*\varphi=\varphi\circ u$.\newline On peut de même définir le \og poussé en avant \fg{} d'une forme linéaire $\psi\in E^*$ par $u$ par $u_*\psi=\psi\circ u^{-1}$ . Le but de la section qui suit est de donner des notions de \og tiré en arrière \fg{} et de \og poussé en avant \fg{} pour les variétés différentielles.  

\medskip

Soit $\mathcal{M}$ et $\mathcal{N}$ deux variétés différentielles et $\phi:\mathcal{M}\to\mathcal{N}$ un difféomorphisme.
\begin{defn}[Tiré en arrière]
Soit $T:\mathcal{N}\to T^r_s\mathcal{N}$ un champ tensoriel.\newline Le \textbf{tiré en arrière de $T$ par $\phi$} est le champ tensoriel $\phi^*T:\mathcal{M}\to T^r_s\mathcal{M}$ défini par, pour tout $p\in\mathcal{M}$, pour tous $\omega_1,\ldots,\omega_r\in T_p^*\mathcal{M}$, pour tous $v_1,\ldots,v_s\in T_p\mathcal{M}$ :
\begin{align*}
&(\phi^*T)_p(\omega_1,\ldots,\omega_r,v_1,\ldots,v_s)\\&=T_{\phi(p)}(\omega_1\circ \left(T_p\phi\right)^{-1},\ldots,\omega_r\circ\left(T_p\phi\right)^{-1}, T_p\phi(v_1),\ldots, T_p\phi(v_s)).
\end{align*}
\end{defn}
\begin{ex}
Soit $f:\mathcal{N}\to\mathbf{R}$ une fonction numérique. On a :
$$\phi^*f:\mathcal{M}\to \mathbf{R},p\mapsto f(\phi(p)).$$
\end{ex}
\begin{ex}
Soit $X:\mathcal{N}\to T\mathcal{N}$ un champ vectoriel. On a :
$$
\phi^*X:\mathcal{M}\to T\mathcal{M},p\mapsto (T_{\phi^{-1}(p)}\phi)^{-1}(X_{\phi(p)}).
$$
\end{ex}
\begin{ex}
Soit $\omega:\mathcal{N}\to T^*\mathcal{N}$ une $1$-forme différentielle. On a :
$$\phi^*\omega:\mathcal{M}\to T^*\mathcal{M},p\mapsto \omega_{\phi(p)}\circ T_p\phi.$$
\end{ex}
\begin{defn}[Poussé en avant]
Soit $T:\mathcal{M}\to T^r_s\mathcal{M}$ un champ tensoriel.\newline Le \textbf{poussée en avant de $T$ par $\phi$} est le champ tensoriel $\phi_*T:\mathcal{N}\to T^r_s\mathcal{N}$ défini par, pour tout $p\in\mathcal{N}$, pour tous $\omega_1,\ldots,\omega_r\in T_p^*\mathcal{N}$, pour tous $v_1,\ldots,v_s\in T_p\mathcal{M}$ :
\begin{align*}
&(\phi_*T)_p(\omega_1,\ldots,\omega_r,v_1,\ldots,v_s)\\&=T_{\phi^{-1}(p)}(\omega_1\circ T_p\phi,\ldots,\omega_r\circ T_p\phi,(T_p\phi)^{-1}(v_1),\ldots,(T_p\phi)^{-1}(v_s)).
\end{align*}
\end{defn}
\begin{ex}
Soit $f:\mathcal{M}\to\mathbf{R}$ une fonction numérique. On a :
$$
\phi_*f:\mathcal{N}\to\mathbf{R},p\mapsto f(\phi^{-1}(p)).
$$
\end{ex}
\begin{ex}
Soit $X:\mathcal{M}\to T\mathcal{M}$ un champ vectoriel. On a :
$$
\phi_*X:\mathcal{N}\to T\mathcal{N},p\mapsto \left(f\mapsto X_{\phi^{-1}(p)}(f\circ\phi)\right).
$$
\end{ex}
\subsection{Dérivée de Lie}
\begin{defn}
Soit $X$ un champ vectoriel sur $\mathcal{M}$ et $f$ une fonction numérique sur $\mathcal{M}$. On définit la \textbf{dérivée directionnelle de $f$ suivant $X$} par :
$$X(f):\mathcal{M}\to\mathbf{R}, p\mapsto X_p(f).$$
\end{defn}
\begin{rqe}
On note que $X(f)$ ne dépend pas des variations de $X$, ce qui justifie l'expression \og dérivée directionnelle \fg{} .
\end{rqe}
\begin{ex}
On se donne une carte $(x^1,\ldots,x^n)$. On écrit $X=X^i\dfrac{\partial}{\partial x^i}$.\newline On a : $X(f)(p)=X^i(p)\left.\dfrac{\partial f}{\partial x^i}\right|_p$, ce qu'on écrit un peu abusivement $X(f)=X^i\dfrac{\partial f}{\partial x^i}$.
\end{ex}
\begin{rqe}
On a : $X(f)=\left.\dfrac{\mathrm{d}}{\mathrm{d}t}\phi_t^*f\right|_{t=0}$ où $\phi$ est le flot associé à $X$.
\end{rqe}
Le but de cette section est de généraliser cette notion de dérivée directionnelle aux champs de tenseurs.
Commençons par introduire un objet pour l'instant purement algébrique, le crochet de \bsc{Lie}.
\begin{defn}
Soit $X$ et $Y$ deux champs vectoriels sur $\mathcal{M}$.\newline
On définit le \textbf{crochet de Lie} de $X$ et $Y$, noté $[X,Y]$, par :
$$\forall p\in\mathcal{M},\forall f\in\mathcal{C}^1(\mathcal{M},\mathbf{R}), [X,Y]_p(f)=X_p(Y(f))-Y_p(X(f)).$$
\end{defn}
\begin{ex}
On se donne $(x^1,\ldots,x^n)$ une carte de $\mathcal{M}$.

On écrit $X=X^i\dfrac{\partial}{\partial x^i}$ et $Y=Y^i\dfrac{\partial}{\partial x^i}$. On a :
\begin{align*}
[X,Y]&=X^i\dfrac{\partial}{\partial x^i}\left(Y^j\dfrac{\partial}{\partial x^j}\right)-Y^i\dfrac{\partial}{\partial x^i}\left(X^j\dfrac{\partial}{\partial x^j}\right)\\&=X^i\left(\dfrac{\partial Y^j}{\partial x^i}\dfrac{\partial}{\partial x^j}+Y^j\dfrac{\partial ^2}{\partial x^i\partial x^j}\right)-Y^i\left(\dfrac{\partial X^j}{\partial x^i}\dfrac{\partial}{\partial x^j}+X^j\dfrac{\partial ^2}{\partial x^i\partial x^j}\right)\\&=\left(X^j\dfrac{\partial Y^i}{\partial x^j}-Y^j\dfrac{\partial X^i}{\partial x^j}\right)\dfrac{\partial}{\partial x^i}.
\end{align*}
\end{ex}
\begin{rqe}
Il est possible d'interpréter le crochet de \bsc{Lie} comme le défaut de commutation de deux flots au sens suivant : si on transporte un point $p$ pendant un temps $t$ à l'aide d'un champ vectoriel $X$, puis pendant un temps $s$ à l'aide d'un champ vectoriel $Y$, on n'obtient pas le même point si on inverse les deux transports. On note $\phi_t$ la flot de $X$ et $\psi_s$ la flot de $Y$, et on se donne $x\in\mathcal{M}$. On raisonne sur une carte définie au voisinage de $x$. On pose, pour $t$ au voisinage de $0$, $\gamma(t)=(\psi_{-t}\circ\phi_{-t}\circ\psi_t\circ\phi_t)(x)$.\newline On a, après calcul :
$$
\gamma(t)=(\psi_{-t}\circ\phi_{-t}\circ\psi_t)(x+tX_{x}+o(t))=x+t[X,Y]+o(t).
$$
\end{rqe}
\begin{prop}
Muni de $[\cdot,\cdot]$, $\Gamma T\mathcal{M}$ est une $\mathbf{R}$-algèbre de \bsc{Lie}.
\end{prop}
\newpage
Jusqu'à la fin du paragraphe, on se donne $X$ un champ vectoriel dont on note $\phi$ le flot.
\begin{defn}[Dérivée de \bsc{Lie} d'un champ vectoriel]
On définit la dérivée de Lie d'un champ de vecteurs $Y$ le long de $X$ par : $\mathcal{L}_XY=[X,Y]$.
\end{defn}
La proposition suivante donne un point de vue géométrique sur la dérivée de \bsc{Lie}, ainsi qu'une écriture permettant de la généraliser à d'autres objets.
\begin{prop}
Soit $Y$ un champ vectoriel sur $\mathcal{M}$. On a :
$$
\forall p\in\mathcal{M},\mathcal{L}_XY_p=\left.\dfrac{\mathrm{d}}{\mathrm{d}t}\left(\phi_t^*Y\right)_p\right|_{t=0}.
$$
\end{prop}
\begin{dem}
Soit $p\in\mathcal{M}$ et $f\in\mathcal{C}^\infty(\mathcal{M},\mathbf{R})$.\newline On a, en notant $\psi$ le flot associé à $Y$ :
\begin{align*}
\left.\dfrac{\mathrm{d}}{\mathrm{d}t}(\phi_t^*Y)_p(f)\right|_{t=0}&=\left.\dfrac{\mathrm{d}}{\mathrm{d}t}\left(T_{\phi_t}\phi_{-t}\right)(Y_{\phi_t(p)})(f)\right|_{t=0}=\left.\dfrac{\mathrm{d}}{\mathrm{d}t}Y_{\phi_t(p)}(f\circ\phi_{-t})\right|_{t=0}\\&=\left.\dfrac{\mathrm{d}}{\mathrm{d}t}\left(\left.\dfrac{\mathrm{d}}{\mathrm{d}s}\psi_s(\phi_t(p))\right|_{s=0}(f\circ\phi_{-t})\right)\right|_{t=0}\\&=\left.\dfrac{\mathrm{d}}{\mathrm{d}t}\left.\dfrac{\mathrm{d}}{\mathrm{d}s}f((\phi_{-t}\circ\psi_s\circ\phi_t)(p))\right|_{s=0}\right|_{t=0}\\&=\left.\dfrac{\mathrm{d}}{\mathrm{d}s}\left.\dfrac{\mathrm{d}}{\mathrm{d}t}f((\phi_{-t}\circ\psi_s\circ\phi_t)(p))\right|_{t=0}\right|_{s=0}\\&=\left.\dfrac{\mathrm{d}}{\mathrm{d}s}\left(\left.\dfrac{\mathrm{d}}{\mathrm{d}t}f(\phi_{-t}(\psi_s(p)))\right|_{t=0}+\left.\dfrac{\mathrm{d}}{\mathrm{d}t}f(\psi_s(\phi_t(p)))\right|_{t=0}\right)\right|_{s=0}\\&=\left.\dfrac{\mathrm{d}}{\mathrm{d}s}\left(-X_{\psi_s(p)}(f)+X_p(f\circ\psi_s)\right)\right|_{s=0}\\&=-Y_p(X(f))+X_p(Y(f))\\&=[X,Y]_p(f)=(\mathcal{L}_XY)_p(f).
\end{align*}
\end{dem}
\begin{defn}
Étant donné un champ tensoriel $T$ sur $\mathcal{M}$, sa dérivée de \bsc{Lie} $\mathcal{L}_XT$ est le champ tensoriel du même type défini par : $\forall p\in\mathcal{M}, (\mathcal{L}_XT)_p=\left.\dfrac{\mathrm{d}}{\mathrm{d}t}\left(\phi_t^*T\right)_p\right|_{t=0}$.
\end{defn}
\begin{rqe}
Cette définition coïncide avec la dérivée de \bsc{Lie} d'un champ scalaire ou d'un champ vectoriel.
\end{rqe}
\begin{rqe}
La dérivée de \bsc{Lie} est compatible avec les opérations linéaires. Par exemple :
\begin{itemize}
\item Pour tous champs tensoriels $T,S$ : $$\mathcal{L}_X(T\otimes S)=(\mathcal{L}_XT)\otimes S+T\otimes \mathcal{L}_X S.$$
\item Pour tout champ vectoriel $Y$, pour toute $1$-forme $\omega$ : $$\mathcal{L}_X(\omega(Y))=\mathcal{L}_X\omega(Y)+\omega(\mathcal{L}_XY).$$
\end{itemize}
\end{rqe}
Pour les exemples suivants, on se donne $(x^1,\ldots,x^n)$ une carte sur $\mathcal{M}$, et on écrit localement $X=X^i\dfrac{\partial}{\partial x^i}$.
\begin{ex}
Soit $\omega$ une $1$-forme différentielle et $Y$ un champ vectoriel, qu'on écrit localement $\omega=\omega_i\mathrm{d}x^i$ et $Y=Y^i\dfrac{\partial}{\partial x^i}$. On a :
$$
X^j\dfrac{\partial(\omega_iY^i)}{\partial x^j}=\left(\mathcal{L}_X\omega\right)_iY^i+\omega_i\left(X^j\dfrac{\partial Y^i}{\partial x^j}-Y^j\dfrac{\partial X^i}{\partial x^j}\right),
$$
donc :
$$
\left(\mathcal{L}_X\omega\right)_i=X^j\dfrac{\partial\omega_i}{\partial x^j}+\omega_j\dfrac{\partial X^j}{\partial x^i}.
$$
\end{ex}
\begin{ex}
Soit $g$ un champ de tenseurs deux fois covariant, qu'on écrit :
$$g=g_{ij}\mathrm{d}x^i\otimes\mathrm{d}x^j.$$
On a :
$$
\mathcal{L}_Xg=\left(X^k\dfrac{\partial g_{ij}}{\partial x^k}+g_{ij}\dfrac{\partial X^i}{\partial x^j}+g_{ij}\dfrac{\partial X^j}{\partial x^i}\right)\mathrm{d}x^i\otimes\mathrm{d}x^j
$$
\end{ex}
\begin{ex}
Soit $T$ un champ de tenseurs de type $(r,s)$, qu'on écrit :
$$T=T^{i_1,\ldots,i_r}_{j_1,\ldots,j_s}\dfrac{\partial}{\partial x^{i_1}}\otimes\cdots\otimes\dfrac{\partial}{\partial x^{i_r}}\otimes\mathrm{d}x^{j_1}\otimes\cdots\otimes\mathrm{d}x^{j_s}.$$
On a :
$$
\left(\mathcal{L}_XT\right)^{i_1,\ldots,i_r}_{j_1,\ldots,j_s}=X^k\dfrac{\partial T^{i_1,\ldots,i_r}_{j_1,\ldots,j_s}}{\partial x^k}+T^{i'_1,\ldots,i'_r}_{j'_1,\ldots,j'_s}\left(-\dfrac{\partial X^{i_1}}{\partial x^{i_1'}}-\cdots-\dfrac{\partial X^{i_r}}{\partial x^{i_r'}}+\dfrac{\partial X^{j'_1}}{\partial x^{j_1}}+\cdots+\dfrac{\partial X^{j'_s}}{\partial x^{j_s}}\right).
$$
\end{ex}
\begin{rqe}
Comme le montrent les exemples ci-dessous, à l'exception des champs scalaires, les dérivées de \bsc{Lie} ne définissent pas à proprement parler des dérivées directionnelles, puisque les dérivées de $X$ interviennent dans le résultat du calcul.

C'est dans l'optique de remédier à ce problème que nous aborderons la notion de connexion dans la prochaine section.
\end{rqe}
\newpage
\section{Connexions}
On se donne $\mathcal{M}$ une variété différentielle lisse.
\subsection{Définition}
\begin{defn}
Une \textbf{connexion} sur un fibré  lisse $(E,\mathcal{M},\pi)$ est une application
$$\nabla: \Gamma T\mathcal{M}\times \Gamma E\to\Gamma E,(X,Y)\mapsto\nabla_XY$$
vérifiant les propriétés suivantes :
\begin{enumerate}[label=\roman*)]
\item Pour toute section lisse $\sigma\in\Gamma E$, $\Gamma T\mathcal{M}\to\Gamma E,X\mapsto\nabla_X\sigma$ est $\mathcal{C}^\infty(\mathcal{M},\mathbf{R})$-linéaire.
\item Pour tous champs vectoriels lisse $X\in\Gamma T\mathcal{M}$, $\Gamma E\to\Gamma E,\sigma\mapsto\nabla_X\sigma$ est $\mathbf{R}$-linéaire.
\item Pour toute fonction numérique $f\in\mathcal{C}^\infty(\mathcal{M},\mathbf{R})$, pour tout champ vectoriel lisse $X\in\Gamma T\mathcal{M}$, pour toute section lisse $\sigma\in \Gamma E$ :
$$\nabla_X(f\sigma)=\left(\mathcal{L}_Xf\right)\nabla_X\sigma+f\nabla_X\sigma.$$
\end{enumerate}
\end{defn}
\begin{rqe}
La première condition permet de pallier au défaut constaté pour la dérivée de \bsc{Lie} : $(\nabla_X\cdot)_p$ ne dépend que de la valeur de $X$ en $p$ (et non des dérivées partielles de ses composantes). Le troisième condition exprime la compatibilité de la connexion avec la dérivée directionnelle des champs scalaires.
\end{rqe}
\begin{ex}
On vérifie que $\nabla_X\sigma=(\mathcal{L}_X\sigma^1,\ldots,\mathcal{L}_X\sigma^n)$ est une connexion sur le fibré trivial $\mathcal{M}\times\mathbf{R}^n$.
\end{ex}
Par compatibilité, on peut étendre une connexion  à tous les fibrés lisses construits à partir de $(E,\mathcal{M},\pi)$, comme le fibré dual ou les fibrés tensoriels.
\begin{ex}
Étant donnée une connexion $\nabla$, on peut l'étendre aux sections du fibré dual $\Gamma E^*$ par :
\begin{center}
Pour tout $X\in \Gamma T\mathcal{M}$, pour tout $\sigma\in\Gamma E$, pour tout $\omega\in\Gamma E^*$ : $\mathcal{L}_X(\omega(\sigma))=\left(\nabla_X\omega\right)(\sigma)+\omega(\nabla_X\sigma)$.
\end{center}
\end{ex}
\begin{ex}
Étant donnée une connexion $\nabla$, on peut l'étendre aux fibrés tensoriels avec l'identité :
$$
\nabla_X(T\otimes S)=\nabla_X(T)\otimes S+T\otimes\nabla_X(S).
$$
où $X$ désigne un champ vectoriel et où $T,S$ sont des champs tensoriels.
\end{ex}
\newpage
Jusqu'à la fin de cette section, on se donne $\nabla$ une connexion sur un fibré lisse $(E,\mathcal{M},\pi)$.
\begin{defn}
Étant donnée une carte $(x^1,\ldots,x^n)$ sur $\mathcal{M}$ et $(e_1,\ldots,e_p)$ une trivialisation locale sur $E$, on définit les \textbf{symboles de Christoffel} de $\nabla$ par :
$$\Gamma_{ij}^k=\nabla_{\dfrac{\partial}{\partial x^i}}\left(e_j\right)^k.$$
\end{defn}
\begin{rqe}
Notons que $\Gamma_{ij}^k$ ne définit pas un champ tensoriel.
\end{rqe}
\begin{rqe}
Dans le cas où $E$ est le fibré tangent, la trivialisation locale de $T\mathcal{M}$ sera toujours celle associée à la carte $(x^1,\ldots,x^n)$.
\end{rqe}
Pour les exemples qui vont suivre, on se donne $X$ un champ vectoriel, qu'on écrit localement $X=X^i\dfrac{\partial}{\partial x^i}$.
\begin{ex}
Soit $\sigma$ une section de $E$ qu'on écrit localement $\sigma=\sigma^ie_i$. On a :
\begin{align*}
\nabla_X\sigma&=X^i\nabla_{\dfrac{\partial}{\partial x^i}}\left(\sigma^je_j\right)=X^i\dfrac{\partial \sigma^j}{\partial x^i}e_j+X^i\sigma^j\nabla_{\dfrac{\partial}{\partial x^i}}\left(e_j\right)\\&=X^i\dfrac{\partial \sigma^j}{\partial x^i}e_j+X^i\sigma^j\Gamma_{ij}^ke_k=\left(X^i\dfrac{\partial \sigma^k}{\partial x^i}+X^i\sigma^j\Gamma_{ij}^k\right)e_k.
\end{align*}
Notons que, contrairement à la dérivée de \bsc{Lie}, $(\nabla_X\cdot)_p$ ne dépend que de la valeur de $X$ en $p$ (et non de ses dérivées). En particulier, si $X_p=0$, alors $(\nabla_X\cdot)_p=0$.
\end{ex}
\begin{ex}
Soit $\omega$ une section du fibré dual, qu'on écrit localement $\omega=\omega_ie^i$.\newline On a : $\omega(\sigma)=\omega_i\sigma^i$, donc 
$$
X^j\dfrac{\partial (\omega_i\sigma^i)}{\partial x^j}=(\nabla_X\omega)_i\sigma^i+\omega_k\left(X^i\dfrac{\partial \sigma^k}{\partial x^i}+X^i\sigma^j\Gamma_{ij}^k\right)
$$
donc
$$
X^j\omega_i\dfrac{\partial \sigma^i}{\partial x^j}+X^j\sigma^i\dfrac{\partial\omega_i}{\partial x^j}=(\nabla_X\omega)_i\sigma^i+\omega_kX^i\dfrac{\partial \sigma^k}{\partial x^i}+\omega_kX^i\sigma^j\Gamma_{ij}^k
$$
donc
$$
\left(\nabla_X\omega\right)_i=X^j\dfrac{\partial\omega_i}{\partial x^j}-\omega_kX^j\Gamma_{ji}^k.
$$
En particulier :
$$\forall i\in\llbracket 1,n\rrbracket,\nabla_Xe^i=-X^j\Gamma_{jk}^ie^k$$
\end{ex}
puis :
$$
\forall i,j\in\llbracket 1,n\rrbracket, \nabla_{\dfrac{\partial}{\partial x^j}}e^i=-\Gamma_{jk}^ie^k.
$$
\newpage
\begin{ex}
Soit $g$ un champ de tenseurs de type $(0,2)$, qu'on écrit localement $g=g_{ij}e^i\otimes e^j$. On a : $\nabla_Xg=X^i\nabla_{\dfrac{\partial}{\partial x^i}}g$. De plus :
\begin{align*}
\forall i\in\llbracket 1,n\rrbracket, \nabla_{\dfrac{\partial}{\partial x^i}}g&=\dfrac{\partial g_{kl}}{\partial x^i} e^k\otimes e^l+g_{kl}\left(\nabla_{\dfrac{\partial}{\partial x^i}} e^k\right)\otimes e^l+g_{kl} e^k\otimes\left(\nabla_{\dfrac{\partial}{\partial x^i}} e^l\right)\\&=\dfrac{\partial g_{kl}}{\partial x^i} e^k\otimes e^l-g_{kl}\Gamma_{im}^k e^m\otimes e^l-g_{kl}\Gamma_{im}^l e^k\otimes e^m\\&=\left(\dfrac{\partial g_{kl}}{\partial x^i}-g_{ml}\Gamma_{ik}^m-g_{km}\Gamma_{il}^m\right) e^k\otimes e^l,
\end{align*}
donc :
$$
\left(\nabla_Xg\right)_{kl}=X^i\left(\dfrac{\partial g_{kl}}{\partial x^i}-g_{ml}\Gamma_{ik}^m-g_{km}\Gamma_{il}^m\right)
$$
\end{ex}
\begin{ex}
Soit $T$ un champ de tenseurs de type $(r,s)$, qu'on écrit :
$$
T=T^{i_1,\ldots,i_r}_{j_1,\ldots,j_s}e_{i_1}\otimes\cdots\otimes e_{i_r}\otimes e^{j_1}\otimes\cdots\otimes e^{j_s}.
$$
On a :
\begin{align*}
\left(\nabla_XT\right)^{i_1,\ldots,i_r}_{j_1,\ldots,j_s}&=X^k\left(\dfrac{\partial T_{j_1,\ldots,j_s}^{i_1,\ldots,i_r}}{\partial x^i}+T_{j_1,\ldots,j_s}^{i_1',i_2,\ldots,i_r}\Gamma_{ki_1'}^{i_1}+\cdots+T_{j_1,\ldots,j_s}^{i_1,\ldots,i_{r-1},i_r'}\Gamma_{ki'_r}^{i_n}\right)\\&-X^k\left(T_{j_1',j_2,\ldots,j_s}^{i_1,\ldots,i_r}\Gamma_{kj_1}^{j_1'}+\cdots+T_{j_1,\ldots,j_{s-1},j_s'}^{i_1,\ldots,i_r}\Gamma_{kj_s}^{j_s'}\right).
\end{align*}
\end{ex}
\begin{ex}
Établissons les relations entre les coefficients de \bsc{Christoffel} (d'une connexion sur le fibré tangent) pris dans deux cartes.\newline
On se donne $(y^1,\ldots,y^n)$ une carte sur $\mathcal{M}$ dont le domaine de définition intersecte celui de $(x^1,\ldots,x^n)$.\newline On note $\Gamma_{~i'j'}^{'k'}$ les coefficients de \bsc{Christoffel} de $\nabla$ dans $(y^1,\ldots,y^n)$.
On a :
\begin{align*}
\nabla_{\dfrac{\partial}{\partial y^{i'}}}\left(\dfrac{\partial}{\partial y^{j'}}\right)&=\nabla_{\dfrac{\partial x^i}{\partial y^{i'}}\dfrac{\partial}{\partial x^i}}\left(\dfrac{\partial x^j}{\partial y^{j'}}\dfrac{\partial}{\partial x^j}\right)\\&=\dfrac{\partial x^i}{\partial y^{i'}}\left(\dfrac{\partial^2x^j}{\partial x^i\partial y^{j'}}\dfrac{\partial}{\partial x^j}+\dfrac{\partial x^j}{\partial y^{j'}}\nabla_{\dfrac{\partial}{\partial x^i}}\left(\dfrac{\partial}{\partial x^j}\right)\right)\\&=\left(\dfrac{\partial x^i}{\partial y^{i'}}\dfrac{\partial^2x^k}{\partial x^i\partial y^{j'}}+\dfrac{\partial x^i}{\partial y^{i'}}\dfrac{\partial x^j}{\partial y^{j'}}\Gamma_{ij}^k\right)\dfrac{\partial}{\partial x^k},
\end{align*}
donc :
$$
\Gamma_{~i'j'}^{'k'}=\dfrac{\partial x^i}{\partial y^{i'}}\dfrac{\partial x^j}{\partial y^{j'}}\dfrac{\partial y^{k'}}{\partial x^k}\Gamma_{ij}^k+\dfrac{\partial y^{k'}}{\partial x^k}\dfrac{\partial^2x^k}{\partial y^{i'}\partial y^{j'}}.
$$
Notons que, si un symbole de \bsc{Christoffel} ne définit pas un tenseur, la différence de deux symboles de \bsc{Christoffel} en définit un, dans le sens où le changement de coordonnées vérifie les règles du calcul tensoriel.
\end{ex}
\newpage
\subsection{Transport parallèle}
On se donne $E$ un fibré lisse sur $\mathcal{M}$ et $\nabla$ une connexion sur $E$. En plus de définir une bonne notion de dérivée directionnelle, une connexion permet d'établir un lien entre les fibres $E_p$ de $E$, en y définissant une notion de transport.\newline

Commençons par définir une notion de dérivée covariante le long du \og champ \fg{} de vecteurs tangents à une courbe (on ne parle pas de champ car il n'est pas défini sur $\mathcal{M}$ tout entier, ni même sur une partie de $\mathcal{M}$). 
\begin{defn}
Étant donné $I$ un intervalle ouvert de $\mathbf{R}$ et $\gamma:I\to\mathcal{M}$ une courbe lisse, on pose $\Gamma_\gamma(E)$ l'ensemble des applications lisses $X:I\to E$ telles que :$$\forall t\in I, X(t)\in E_{\gamma(t)}.$$
Étant donné $X\in\Gamma_{\gamma}(E)$, on définit $\nabla_{\dot{\gamma}}X$ dans un système de coordonnées par :
$$
\forall t\in I, (\nabla_{\dot{\gamma}}X(t))^k=\dot{X}^k(t)+\gamma^i(t)X^j(t)\Gamma_{ij}^k(\gamma(t)).
$$
Notons que $\nabla_{\dot{\gamma}}X\in\Gamma_\gamma(t)$, et qu'on dispose d'une règle de \bsc{Leibniz} semblable à celle de la connexion $\nabla$.
\end{defn}
\begin{defn}
Soit $I$ un intervalle ouvert de $\mathbf{R}$ et $\gamma:I\to\mathcal{M}$ une courbe lisse.\newline Pour tout $s\in I$ et $X\in E_{\gamma (s)}$, le problème de \bsc{Cauchy} $\left \{
\begin{array}{rcl}
\nabla_{\dot{\gamma}}Y &=&0 \\
Y(s)&=&X\\
\end{array}
\right.$ admet une unique solution globale $(I,Y)$. On note, pour tout $t\in I$, $P_s^t(\gamma)X=Y(t)$.\newline

L'application $P_s^t(\gamma):E_{\gamma(s)}\to E_{\gamma(t)},X\mapsto P_s^t(\gamma)X$ est un isomorphisme d'espaces vectoriels. La famille $(P_{s}^t(\gamma))_{(t,s)\in I^2}$ est le \textbf{transport parallèle de $\gamma$}
\end{defn}
\begin{just}
En coordonnées locales, on se ramène à un problème de \bsc{Cauchy} linéaire associé à un opérateur borné sur tout segment, qui admet une unique solution globale, ce qui conclut l'existence d'une solution globale. Le deuxième point vient d'une remarque qui suit.
\end{just}
\begin{rqe}
Autrement dit, $P_t^s(\gamma)$ prend un vecteur $X$ tangent à $\mathcal{M}$ en $\gamma(t)$, et le transporte parallèlement (c'est-à-dire sans modifier la position relative entre $X$ et le champ de vecteurs dérivés de $\gamma$) le long de $\gamma$ jusqu'à arriver en $\gamma(s)$.
\end{rqe}
\begin{rqe}
Pour tous $r,s,t\in I$, on a la relation de \bsc{Chasles} suivante :
$$P_r^s(\gamma)\circ P_s^t(\gamma)=P_r^t(\gamma).$$
En prenant $t=r$, on obtient que $P_r^s(\gamma)$ est un isomorphisme d'espaces vectoriel.
\end{rqe}
\newpage
\subsection{Transport parallèle et géodésiques}
Soit $\nabla$ une connexion sur $T\mathcal{M}$. Le but de cette section est de définir la notion de géodésique, qui en physique correspond à la courbe décrite par un mobile en chute libre (ie dont l'accélération est nulle).
\begin{defn}
Une courbe lisse $\gamma:I\to\mathcal{M}$ est une \textbf{géodésique} si :
$$\forall t\in I, \nabla_{\dot{\gamma}(t)}\dot{\gamma}(t)=0.$$
\end{defn}
\begin{rqe}
Les géodésiques sont parfois appelées \textbf{courbes auto-parallèles}.\newline En effet, c'est une courbe pour laquelle le transport de ses vecteurs tangents est trivial.
\end{rqe}
\begin{ex}
Soit $\gamma$ une géodésique sur $\mathcal{M}$ et $(x^1,\ldots,x^n)$ une carte sur $\mathcal{M}$ définie en $\gamma(0)$. On note $(\gamma^1,\ldots,\gamma^n)$ les coordonnées locales de $\gamma$ dans la carte précédente.\newline On a :
$$
\forall k\in\llbracket 1,n\rrbracket, \dfrac{\mathrm{d}^2\gamma^k}{\mathrm{d}t^2}+\Gamma_{ij}^k\dfrac{\mathrm{d}\gamma^i}{\mathrm{d}t}\dfrac{\mathrm{d}\gamma^j}{\mathrm{d}t}=0.
$$
où $\Gamma_{ij}^k$ sont les coefficients de \bsc{Christoffel} de $\nabla$ dans la carte $(x^1,\ldots,x^n)$.\newline Il s'agit de \textbf{l'équation des géodésiques}.
\end{ex}
\begin{prop}
Soit $p$ un point de $\mathcal{M}$ et $v$ un vecteur tangent à $\mathcal{M}$ en $p$.\newline Il existe une unique géodésique maximale $(I,\gamma)$ telle que $\gamma(0)=p$ et $\dot{\gamma}(0)=v$.
\end{prop}
\begin{dem}
Il suffit d'utiliser un système de coordonnées et d'appliquer le théorème de \bsc{Cauchy}-\bsc{Lipschitz}.
\end{dem}
\begin{rqe}
Attention, la géodésique n'est pas nécessairement définie sur $\mathbf{R}$ tout entier. Voir le théorème de \bsc{Hopf}-\bsc{Rinow} pour plus de détails.
\end{rqe}
\newpage
\subsection{Coordonnées normales}
Soit $p$ un point de $\mathcal{M}$.  L'objectif de cette section est de donner une carte contenant $p$ dans laquelle les calculs sont particulièrement aisés.
\begin{defn}
Pour tout vecteur $v\in T_p\mathcal{M}$, il existe une unique géodésique maximale $(I_v,\gamma_v)$ telle que $\gamma(0)=p$ et $\dot{\gamma}(0)=v$.\newline Notons que $\forall\lambda\in\mathbf{R}^*, I_v=\lambda I_{\lambda v}$ et $\forall t\in I_{\lambda v}, \gamma_{ v}(t)=\gamma_{\lambda v}\left(\dfrac{t}{\lambda}\right)$, donc il existe un voisinage $U$ de $0_{T_p\mathcal{M}}$ tel que $\forall v\in U, [0,1]\subset I_v$.\newline On définit ainsi l'application $\exp_p:U\to\mathcal{M},v\mapsto\gamma_v(1)$.
\end{defn}
\begin{prop}
L'application $\exp_p$ est différentiable en $0$ et sa différentielle vaut $\Id$.
\end{prop}
\begin{dem}
Le théorème de \bsc{Cauchy}-\bsc{Lipschitz} garantit que $\exp_p$ est lisse.\newline
On a, dans un système de coordonnées locales :
$$
\dfrac{\exp_p(tv)-p}{t}=\dfrac{\gamma_{tv}(1)-p}{t}=\dfrac{\gamma_v(t)-\gamma_v(0)}{t}=\dot{\gamma_v}(0)+o(1)=v+o(1).
$$
donc la différentielle de $\exp_p$ en $0$ est l'identité de $T_p\mathcal{M}$.
\end{dem}
\begin{cor}
L'application $\exp_p$ induit un $\mathcal{C}^k$-difféomorphisme local d'un voisinage de $0_{T_p\mathcal{M}}$ vers un voisinage de $p$ dans $\mathcal{M}$.
\end{cor}
\begin{dem}
Il s'agit d'une conséquence du théorème d'inversion locale.
\end{dem}
\begin{defn}
Soit $(x^1,\ldots,x^n)$ un système de coordonnées sur $\mathcal{M}$ défini au voisinage de $p$.
\textbf{Le système de coordonnées normales associé à $(x^1,\ldots,x^n)$} est défini par la carte de la forme $U\to\mathbf{R}^n,q\mapsto \varphi\circ\exp_p^{-1}(q)$ avec $\varphi:T_p\mathcal{M}\to\mathbf{R}^n$ l'isomorphisme défini par $\forall i\in\llbracket 1,n\rrbracket,\varphi\left(\left.\dfrac{\partial}{\partial x^i}\right|_{p}\right)=e_i$ (avec $(e_1,\ldots,e_n)$ la base canonique de $\mathbf{R}^n$).
\end{defn}
La proposition suivante montre un des avantages de se placer en coordonnées normales.
\begin{prop}
Dans un système de coordonnées normales au point $p$, les coefficients de \bsc{Christoffel} au point $p$ sont nuls.
\end{prop}
\begin{dem}
On se donne $(x^1,\ldots,x^n)$ un système des coordonnées défini au voisinage de $p$, de coordonnées normales associées $(y^1,\ldots,y^n)$.\newline Soit $v\in T_p\mathcal{M}$. On note $(y^1(t),\ldots,y^n(t))$ les coordonnées normales de $\gamma_v(t)=\gamma_{tv}(1)$.\newline On a, par définition : $(y^1(t),\ldots,y^n(t))=(tv^1,\ldots,tv^n)$, donc, en écrivant l'équation des géodésiques en coordonnées normales : $\Gamma_{ij}^k(p)v^iv^j=0$, donc $\Gamma_{ij}^k(p)=0$
\end{dem}
\newpage
\subsection{Tenseur de torsion}
On se donne $\nabla$ une connexion sur $T\mathcal{M}$.
\begin{defn}
Le tenseur de \textbf{torsion} de $\nabla$ est l'application
$$\begin{array}{ccccc}
T & : & \Gamma T\mathcal{M}\times\Gamma T \mathcal{M} & \to & \Gamma  T\mathcal{M} \\
 & & (X,Y) & \mapsto & \nabla_XY-\nabla_YX-[X,Y]\\
\end{array}.$$
\end{defn}
\begin{rqe}
L'application $T$ est $\mathbf{R}$-bilinéaire antisymétrique.
\end{rqe}
\begin{rqe}
La torsion d'une connexion représente le défaut de commutativité de la connexion à l'ordre $1$.
\end{rqe}
\begin{ex}
Soit $(x^1,\ldots,x^n)$ une carte sur $\mathcal{M}$. Soit $X,Y$ deux champs vectoriels, qu'on écrit localement : $X=X^i\dfrac{\partial}{\partial x^i}$ et $Y=Y^i\dfrac{\partial}{\partial x^i}$. On a :
\begin{align*}
T(X,Y)&=\left(X^i\dfrac{\partial Y^k}{\partial x^i}+X^iY^j\Gamma_{ij}^k-Y^i\dfrac{\partial X^k}{\partial x^i}-Y^iX^j\Gamma_{ij}^k\right)\dfrac{\partial}{\partial x^k}\\&-\left(X^j\dfrac{\partial Y^k}{\partial x^j}-Y^j\dfrac{\partial X^k}{\partial x^j}\right)\dfrac{\partial}{\partial x^k}\\&=X^iY^j\left(\Gamma_{ij}^k-\Gamma_{ji}^k\right)\dfrac{\partial}{\partial x^k}.
\end{align*}
On pose $T^k_{~ij}=\Gamma_{ij}^k-\Gamma_{ji}^k$, de sorte que :
$$
T(X,Y)=X^iY^jT^k_{~ij}\dfrac{\partial}{\partial x^k}.
$$
\end{ex}
\begin{defn}
Une connexion est une \textbf{dérivée covariante} si sa torsion est nulle.
\end{defn}
\begin{rqe}
Une connexion est une dérivée covariante si et seulement si ses coefficients de \bsc{Christoffel} sont symétriques en les indices : $\Gamma_{ij}^k=\Gamma_{ji}^k$.
\end{rqe}
\newpage
\subsection{Tenseur de Riemann}
On se donne $\nabla$ une connexion sur $T\mathcal{M}$.
\begin{defn}
Le \textbf{tenseur de Riemann} de $\nabla$, noté $R$, est défini par :
$$
R:\Gamma T\mathcal{M}\times \Gamma T\mathcal{M}\to\mathcal{L}_{\mathcal{C}^\infty(\mathcal{M},\mathbf{R})}(\Gamma T\mathcal{M}),(X,Y)\mapsto \nabla_X\nabla_Y-\nabla_Y\nabla_X-\nabla_{[X,Y]}
$$
\end{defn}
\begin{rqe}
À l'instar du tenseur de torsion, le tenseur de \bsc{Riemann} représente le défaut de commutation d'une connexion à l'ordre $2$. On peut en donner une interprétation géométrique aisée dans le cas d'une connexion sans torsion. On se donne trois champs de vecteurs $X,Y,Z$ avec $[X,Y]=0$. On transporte $Z$ selon $X$, puis selon $Y$.\newline On obtient un vecteur $v_1$. En échangeant les rôles de $X$ et $Y$, on obtient un vecteur $v_2$. La différence $v_2-v_1$ vaut $R(X,Y)Z$.
\end{rqe}
\begin{ex}
On se donne une carte $(x^1,\ldots,x^n)$ sur $\mathcal{M}$, et $X,Y,Z$ trois champs vectoriels qu'on écrit respectivement localement $X=X^i\dfrac{\partial}{\partial x^i}, Y=Y^i\dfrac{\partial}{\partial x^i}, Z=Z^i\dfrac{\partial}{\partial x^i}$.

On a :
$$
\nabla_X\nabla_Y=X(Y^i)\nabla_{\dfrac{\partial}{\partial x^i}}+Y^iX^j\nabla_{\dfrac{\partial}{\partial x^j}}\nabla_{\dfrac{\partial}{\partial x^i}}=X^j\dfrac{\partial Y^i}{\partial x^j}\nabla_{\dfrac{\partial}{\partial x^j}}+Y^iX^j\nabla_{\dfrac{\partial}{\partial x^j}}\nabla_{\dfrac{\partial}{\partial x^i}},
$$
et : $\nabla_{[X,Y]}=[X,Y]^i\nabla_{\dfrac{\partial}{\partial x^i}}$, donc :
$$
R(X,Y)=Y^iX^j\nabla_{\dfrac{\partial}{\partial x^j}}\nabla_{\dfrac{\partial}{\partial x^i}}-X^iY^j\nabla_{\dfrac{\partial}{\partial x^j}}\nabla_{\dfrac{\partial}{\partial x^i}}=X^iY^jR\left(\dfrac{\partial}{\partial x^i},\dfrac{\partial}{\partial x^j}\right).$$

Soit $i,j,k\in\llbracket 1,n\rrbracket$. On a :
\begin{align*}
\nabla_{\dfrac{\partial}{\partial x^i}}\nabla_{\dfrac{\partial}{\partial x^j}}\dfrac{\partial}{\partial x^k}&=\nabla_{\dfrac{\partial}{\partial x^i}}\left(\Gamma_{jk}^l\dfrac{\partial}{\partial x^l}\right)=\dfrac{\partial\Gamma^l_{jk}}{\partial x^i}\dfrac{\partial}{\partial x^l}+\Gamma_{jk}^l\nabla_{\dfrac{\partial}{\partial x^i}}\dfrac{\partial}{\partial x^l}\\&=\dfrac{\partial\Gamma^l_{jk}}{\partial x^i}\dfrac{\partial}{\partial x^l}+\Gamma_{jk}^l\Gamma_{il}^m\dfrac{\partial}{\partial x^m}=\left(\dfrac{\partial\Gamma^l_{jk}}{\partial x^i}+\Gamma_{jk}^m\Gamma_{im}^l\right)\dfrac{\partial}{\partial x^l},
\end{align*}
et :
$$
\nabla_{\left[\dfrac{\partial}{\partial x^i},\dfrac{\partial}{\partial x^j}\right]}=\nabla_0=0,
$$
donc :
$$
R\left(\dfrac{\partial}{\partial x^i},\dfrac{\partial}{\partial x^j}\right)\dfrac{\partial}{\partial x^k}=\left(\dfrac{\partial\Gamma^l_{jk}}{\partial x^i}+\Gamma_{jk}^m\Gamma_{im}^l-\dfrac{\partial\Gamma^l_{ik}}{\partial x^j}-\Gamma_{ik}^m\Gamma_{jm}^l\right)\dfrac{\partial}{\partial x^l}
$$
On pose $R_{~ijk}^l=\dfrac{\partial\Gamma^l_{jk}}{\partial x^i}+\Gamma_{jk}^m\Gamma_{im}^l-\dfrac{\partial\Gamma^l_{ik}}{\partial x^j}-\Gamma_{ik}^m\Gamma_{jm}^l$, de sorte que :
$$
R(X,Y)Z=X^iY^jZ^kR_{~ijk}^l\dfrac{\partial}{\partial x^l}.
$$
Ainsi, le tenseur de \bsc{Riemann} est un tenseur $1$-contravariant et $3$-covariant.\newline Notons que, comme $\forall X,Y\in\Gamma T\mathcal{M},R(X,Y)=-R(Y,X)$, le tenseur de \bsc{Riemann} est antisymétrique par rapport à ses deux premières coordonnées covariantes.
\begin{rqe}
L'application $(X,Y,Z)\mapsto R(X,Y)Z$ est $\mathcal{C}^\infty(\mathcal{M},\mathbf{R})$-trilinéaire.
\end{rqe}
Par la suite, on suppose que $\nabla$ est sans torsion.\newline Dans ce cas, le tenseur de \bsc{Riemann} présente de nombreuses symétries, dont les identités de \bsc{Bianchi}.
\begin{prop}[Première identité de \bsc{Bianchi}]
On a, pour tous champs vectoriels $X,Y,Z\in\Gamma T\mathcal{M}$ :
$$
R(X,Y)Z+R(Z,X)Y+R(Y,Z)X=0.
$$
\end{prop}
\begin{dem}
Comme $\nabla$ est sans torsion, on a :
\begin{align*}
R(X,Y)Z+R(Z,X)Y+R(Y,Z)X=&\nabla_X\nabla_YZ-\nabla_Y\nabla_XZ-\nabla_{[X,Y]}Z\\&+\nabla_Z\nabla_XY-\nabla_X\nabla_ZY+\nabla_{[Z,X]}Y\\&+\nabla_{Y}\nabla_ZX-\nabla_{Z}\nabla_Y\nabla_{[Y,Z]}X\\&=\nabla_X[Y,Z]-\nabla_Y[X,Z]+\nabla_Z[X,Y]\\&-\nabla_{[X,Y]}Z-\nabla_{[Z,Y]}X-\nabla_{[X,Z]}Y\\&=[X,[Y,Z]]+[Y,[X,Z]]+[Z,[X,Y]]\\&=0
\end{align*}
La dernière ligne provient de l'identité de \bsc{Jacobi}.
\end{dem}
\begin{rqe}
Dans un système de coordonnées, cela donne : $R_{lijk}+R_{lkij}+R_{ljki}=0$.
\end{rqe}
\begin{prop}[Seconde identité de \bsc{Bianchi}]
On a, pour tous champs vectoriels $X,Y,Z\in\Gamma T\mathcal{M}$ :
$$
(\nabla_X R)(Y,Z)+(\nabla_Z R)(X,Y)+(\nabla_Y R)(Z,X)=0.
$$
\end{prop}
\begin{dem}
Soit $p\in\mathcal{M}$. On se donne $(x^1,\ldots,x^n)$ un système de coordonnées normales définies au voisinage de $p$.\newline La dépendance en $X,Y,Z$ étant locale et $\mathcal{C}^\infty(\mathcal{M},\mathbf{R})$-linéaire, il suffit de vérifier l'identité ci-dessus pour $X=\dfrac{\partial}{\partial x^i}$, $Y=\dfrac{\partial}{\partial x^j}$ et $Z=\dfrac{\partial}{\partial x^k}$. Les calculs suivants sont laissés en exercice au lecteur.
\end{dem}
\begin{rqe}
La seconde identité de \bsc{Bianchi} admet une interprétation géométrique. Imaginons un pavé de côtés $(\delta x,\delta y,\delta z)$ dont un sommet est le point $P(x,y,z)$.\newline On prend un vecteur $V$, astreint à se déplacer sur les arêtes du pavé selon un transport parallèle. En le transportant autour de la face d'abscisse $x+\delta x$, on obtient une variation à l'ordre $1$ : $-\delta V^l=R^l_{~iyz}(x+\delta x,y,z)V^i\delta y\delta z$. En sommant les contributions pour chacune des faces du pavé, on a : $\delta_{\text{tot}} V^l=\left(\partial_x R^l_{iyz}V^i+\partial_y R^l_{xjz}V^j+\partial_z R^l_{xyk}V^k\right)\delta x\delta y\delta z$. Or chaque arête du pavé est parcourue exactement deux fois, dans deux sens opposés, donc $\delta_{\text{tot}} V^l=0$, ce qui donne la seconde identité de \bsc{Bianchi}.
\end{rqe}
\newpage
\section{Métriques}
On se donne $\mathcal{M}$ une variété différentielle lisse.
\subsection{Connexion de Levi-Civita}
\begin{defn}
Un champ tensoriel $g$ tel que, pour tout point $p$ de $\mathcal{M}$, $g_p$ est une forme bilinéaire symétrique non dégénérée est appelé \textbf{tenseur métrique}.
\end{defn}
\begin{thm}
Soit $g$ un tenseur métrique sur $\mathcal{M}$. Il existe une unique dérivée covariante $\nabla$ telle que, pour tout champ vectoriel $X$, $\nabla_Xg=0$.\newline On l'appelle \textbf{connexion de Levi-Civita} de $g$.
\end{thm}
\begin{dem}
On procède par analyse-synthèse. Soit $\nabla$ une telle dérivée covariante.

Soit $X,Y,Z$ trois champs vectoriels. On a :
\begin{align*}
X (g(Y,Z))&=(\nabla_Xg)(Y,Z)+g(\nabla_XY,Z)+g(Y,\nabla_XZ)\\&=g(\nabla_XY,Z)+g(Y,\nabla_XZ).
\end{align*}
On trouve de même :
$$
Z(g(X,Y))=g(\nabla_ZX,Y)+g(X,\nabla_ZY)\text{ et }Y(g(Z,X))=g(\nabla_YZ,Y)+g(Z,\nabla_YX).
$$
Donc, comme $g$ est symétrique et que $\nabla$ est sans torsion :
\begin{align*}
&X(g(Y,Z))+Y(g(Z,X))-Z(g(X,Y))\\&=2g(\nabla_XY,Z)+g(Y,[X,Z])+g(X,[Y,Z]),
\end{align*}
donc :
\begin{align*}
&g(Z,\nabla_XY)\\&=\dfrac{1}{2}\left(X(g(Y,Z))+Y(g(Z,X))-Z(g(X,Y))-g(Y,[X,Z])-g(X,[Y,Z])\right).
\end{align*}
Cette relation caractérise $\nabla_XY$ d'après le lemme de \bsc{Riesz}.

On vérifie au terme d'un long calcul que $\nabla$ ainsi définie est une dérivée covariante telle que, pour tout champ vectoriel $X$, $\nabla_Xg=0$.
\end{dem}
Jusqu'à la fin de cette section, $\nabla$ désigne la connexion de \bsc{Levi}-\bsc{Civita} associée à $g$.
\newpage
\begin{ex}
Calculons les symboles de \bsc{Christoffel} de la connexion de \bsc{Levi}-\bsc{Civita} $\nabla$ associée à $g$. Soit $(x^1,\ldots,x^n)$ une carte sur $\mathcal{M}$. Soit $X$ un champ vectoriel qu'on écrit localement $X=X^i\dfrac{\partial}{\partial x^i}$. On a : $\nabla_Xg=0$, donc :
$$
\forall k,l\in\llbracket 1,n\rrbracket, X^i\left(\dfrac{\partial g_{kl}}{\partial x^i}-g_{ml}\Gamma_{ik}^m-g_{km}\Gamma_{il}^m\right)=0,
$$
donc :
$$
\forall i,k,l\in\llbracket 1,n\rrbracket, \dfrac{\partial g_{kl}}{\partial x^i}-g_{ml}\Gamma_{ik}^m-g_{km}\Gamma_{il}^m=0.
$$
En permutant les indices :
$$
\forall i,k,l\in\llbracket 1,n\rrbracket,\dfrac{\partial g_{ik}}{\partial x^l}-g_{mk}\Gamma_{li}^m-g_{im}\Gamma_{lk}^m=0\text{ et }\dfrac{\partial g_{li}}{\partial x^k}-g_{mi}\Gamma^m_{kl}-g_{lm}\Gamma_{ki}^m=0
$$
On additionne les deux premières relations, et on retranche la dernière :
$$
\forall i,k,l\in\llbracket 1,n\rrbracket,\dfrac{\partial g_{kl}}{\partial x^i}+\dfrac{\partial g_{ik}}{\partial x^l}-\dfrac{\partial g_{li}}{\partial x^k}=2 g_{km}\Gamma_{il}^m.
$$
Donc, en contractant à l'aide de $g^{jk}$ :
$$
\forall i,j,l\in\llbracket 1,n\rrbracket, \Gamma_{il}^j=\dfrac{1}{2}g^{jk}\left(\dfrac{\partial g_{kl}}{\partial x^i}+\dfrac{\partial g_{ik}}{\partial x^l}-\dfrac{\partial g_{li}}{\partial x^k}\right).
$$
\end{ex}
Terminons ce paragraphe par une propriété importante du transport parallèle associé à la connexion de \bsc{Levi}-\bsc{Civita}, à savoir qu'il s'agit d'une isométrie.
\begin{prop}
Soit $I$ un intervalle ouvert de $\mathbf{R}$ et $\gamma:I\to\mathcal{M}$ une courbe lisse. Pour tous $s,t\in I$, $P_s^t(\gamma)$ est une isométrie :
$$
\forall X,Y\in T_{\gamma(s)}\mathcal{M}, g_{\gamma(s)}(X,Y)=g_{\gamma(t)}(P_s^t(\gamma)X,P_s^t(\gamma)Y).
$$
\end{prop}
\begin{dem}
On a :
\begin{align*}
\dfrac{\mathrm{d}}{\mathrm{d}t}g_{\gamma(t)}(P_s^t(\gamma)X,P_s^t(\gamma)Y)&=\left(\nabla_{\dot{\gamma}} g_{\gamma(t)}\right)(P_s^t(\gamma)X,P_s^t(\gamma)Y)\\&+g_{\gamma(t)}(\nabla_{\dot{\gamma}}P_s^t(\gamma)X,P_s^t(\gamma)Y)\\&+g_{\gamma(t)}(P_s^t(\gamma)X,\nabla_{\dot{\gamma}}P_s^t(\gamma)Y)=0.
\end{align*}
\end{dem}
\begin{rqe}
En affinant la preuve précédente, on remarque que le caractère isométrique du transport parallèle caractérise la connexion de \bsc{Levi}-\bsc{Civita} parmi les dérivées covariantes sur $\mathcal{M}$.
\end{rqe}
\newpage
\subsection{Tenseur de Riemann d'une connexion de Levi-Civita}
Commençons par un lemme concernant les coordonnées normales.
\begin{lem}
Soit $p\in\mathcal{M}$. En coordonnées normales, les dérivées premières des coordonnées de $g$ sont nulles en $p$.
\end{lem}
\begin{dem}
On se place en coordonnées normales. Les calculs effectués auparavant pour exprimer les symboles de \bsc{Christoffel} en fonction des coordonnées de $g$ donnent, pour tout champ de vecteurs $X$ :
$$
\forall k,l\in\llbracket 1,n\rrbracket, X^i\left(\dfrac{\partial g_{kl}}{\partial x^i}-g_{ml}\Gamma_{ik}^m-g_{km}\Gamma_{il}^m\right)=0.
$$
On obtient le résultat voulu en évaluant en $p$.
\end{dem}
Voyons un exemple d'application de cette simplification.
\begin{ex}
Calculons les coordonnées du tenseur de \bsc{Riemann} en fonction de la métrique $g$. Soit $p\in\mathcal{M}$. En coordonnées normales, on a :
$$
\left(R_{~ijk}^l\right)_p=\left.\dfrac{\partial\Gamma_{ik}^l}{\partial x^j}\right|_p-\left.\dfrac{\partial\Gamma_{ij}^l}{\partial x^k}\right|_p
$$
avec, d'après le lemme :
$$
\left.\dfrac{\partial\Gamma_{ik}^l}{\partial x^j}\right|_p=\dfrac{1}{2}\left(g^{lm}\right)_p\left(\left.\dfrac{\partial^2 g_{mk}}{\partial x^j\partial x^i}\right|_p+\left.\dfrac{\partial^2 g_{im}}{\partial x^j\partial x^k}\right|_p-\left.\dfrac{\partial^2 g_{ki}}{\partial x^j\partial x^m}\right|_p\right).
$$
On en déduit une expression du tenseur de \bsc{Riemann} au point $p$ en fonction des coordonnées de la métrique.
\end{ex}
\begin{rqe}
L'expression précédente donne la symétrie dite \og par paires \fg{} du tenseur de \bsc{Riemann} : $R_{lijk}=R_{jli}$. 
\end{rqe}
\begin{defn}
Le \textbf{tenseur de Ricci}, noté $\Ric$, est une contraction du tenseur de \bsc{Riemann}, par rapport à son indice contravariant et son troisième indice covariant :
$$\displaystyle\Ric(X,Y)=\sum\limits_{i=1}^n\left(R(X,Y)\dfrac{\partial}{\partial x^i}\right)^i$$ dans un système de coordonnées locales. Autrement dit : $\Ric_{ij}=R^l_{~ijl}$.
\end{defn}
\end{ex}
\begin{rqe}
La symétrie par paire du tenseur de \bsc{Riemann} donne la symétrie du tenseur de \bsc{Ricci}.
\end{rqe}
\begin{rqe}
Une interprétation géométrique du tenseur de \bsc{Ricci} est délicate.\newline Il s'agit d'une sorte de moyenne directionnelle de la courbure.\newline Notons que le tenseur de \bsc{Ricci} peut être nul sans que le tenseur de \bsc{Riemann} le soit.
\end{rqe}
\begin{defn}
La \textbf{tenseur de Ricci} scalaire, noté $\Rscal$, est la trace du tenseur de \bsc{Ricci} : $\Rscal=\Ric^i_{~i}$.
\end{defn}
\begin{rqe}
Le tenseur de \bsc{Ricci} définit une courbure moyenne.
\end{rqe}
\subsection{Hypersurfaces}
On se donne $\mathcal{M}$ une variété lisse de dimension $n$ (avec $n\in\mathbf{N}^*$) et $g$ une métrique sur $\mathcal{M}$. On note $\nabla$ la connexion de \bsc{Levi}-\bsc{Civita} associée à $g$.
\begin{defn}
Une \textbf{hypersurface} $\Sigma$ de $\mathcal{M}$ est une sous-variété de $\mathcal{M}$ de dimension $n-1$.
\end{defn}
\begin{rqe}
On note $\overline{g}=g_{|\Sigma}$ la métrique induite par $g$ sur $\Sigma$.\newline On l'appelle \textbf{première forme fondamentale} de $\Sigma$.
\end{rqe}
\begin{defn}
Une hypersurface $\Sigma$ de $\mathcal{M}$ est dite \textbf{orientée} s'il existe un champ vectoriel lisse $N\in\Gamma T\mathcal{M}$ tel que, pour tout $p\in\Sigma$, $N_p$ est orthogonal à $T_p\Sigma$ et $g_p(N_p,N_p)\neq 0$ (dans ce cas, on dit que $N$ oriente $\Sigma$)
\end{defn}
\begin{rqe}
Dans le cas où $g$ est un champ de produits scalaires, on peut prendre $N_p$ unitaire pour tout $p\in\Sigma$.
\end{rqe}
On note $\overline{\nabla}$ la connexion de \bsc{Levi}-\bsc{Civita} associée à $\overline{g}$.\newline Le but de la définition suivante est d'établir un lien entre $\nabla$ et $\overline{\nabla}$.
\begin{defn}
Soit $\Sigma$ une hypersurface de $\mathcal{M}$ orientée par $N$.\newline
En revenant à la preuve de l'existence de la connexion de \bsc{Levi}-\bsc{Civita}, on note que : $\forall X,Y,Z\in\Gamma T\Sigma, g(Z,\nabla_XY)=\overline{g}(Z,\overline{\nabla}_XY)$, donc, pour tout $p\in\Sigma$, pour tous $X,Y\in \Gamma T\Sigma$, il existe un unique réel $K_p(X,Y)$ tel que :
$$
\left(\nabla_XY\right)_p=\left(\overline{\nabla}_XY\right)_p+K_p(X,Y)N_p.
$$
Notons que $K_p(X,Y)=\dfrac{g_p\left(\left(\nabla_XY-\overline{\nabla}_XY\right)_p,N_p\right)}{g_p(N_p,N_p)}$, donc l'application $$K:\left(\Gamma T\Sigma\right)^2\to\mathcal{C}^\infty(\Sigma,\mathbf{R}),(X,Y)\mapsto \left(p\mapsto K_p(X,Y)\right)$$
est bien définie et est un tenseur deux fois covariant.\newline On l'appelle \textbf{seconde forme fondamentale de $\Sigma$}.
\end{defn}
\newpage
\subsection{Champs de Killing}
\begin{defn}
Deux variétés lisses munies de métriques $(\mathcal{M},g)$ et $\left(\widetilde{\mathcal{M}},\widetilde{g}\right)$ sont dites \textbf{isomorphes} s'il existe un difféomorphisme lisse $\phi:\mathcal{M}\to\widetilde{\mathcal{M}}$ tel que $\phi^*\widetilde{g}=g$.\newline
Dans le cas où $(\mathcal{M},g)=(\widetilde{\mathcal{M}},\widetilde{g})$, l'application $\phi$ est qualifiée d'\textbf{isométrie}.
\end{defn}
Par la suite, on se donne $(\mathcal{M},g)$ une variété lisse munie d'une métrique.
\begin{defn}
Un champ vectoriel lisse $X\in \Gamma T\mathcal{M}$ de flot $\phi_t$ est un \textbf{champ de Killing} si, pour tout $t$ ou cela a du sens, $\phi_t$ est une isométrie.
\end{defn}
\begin{thm}
Un champ vectoriel lisse $X\in\Gamma T\mathcal{M}$ est une isométrie si et seulement si $\mathcal{L}_Xg=0$.
\end{thm}
\begin{dem}
Le sens direct est immédiat par définition de la dérivée de \bsc{Lie}.\newline
Passons au sens réciproque. On note $\phi_t$ le flot de $X$.\newline On a, pour $s$ suffisamment proche de $0$ : $\left.\dfrac{\mathrm{d}}{\mathrm{d}t}\left(\phi_t\right)^*X\right|_{t=s}=\left.\dfrac{\mathrm{d}}{\mathrm{d}t}\left(\phi_s\right)^*(\phi_t)^*X\right|_{t=0}$, donc, comme $\left.\dfrac{\mathrm{d}}{\mathrm{d}t}(\phi_t)^*X\right|_{t=0}=0$, on a : $\left.\dfrac{\mathrm{d}}{\mathrm{d}t}\left(\phi_t\right)^*X\right|_{t=s}=0$.\newline Ainsi, $\left(\phi_s\right)^*g$ est constant égal à $\left(\phi_0\right)^*g=g$.
\end{dem}
\newpage
\subsection{Variétés riemanniennes}
\begin{defn}
Une \textbf{variété riemannienne} est un couple $(\mathcal{M},g)$ avec $\mathcal{M}$ une variété différentielle lisse et $g$ une métrique définie positive.
\end{defn}
\begin{rqe}
Autrement dit, une variété riemannienne est une variété munie de la donnée lisse d'un produit scalaire sur tout espace tangent.
\end{rqe}
L'intérêt des variétés riemanniennes est de pouvoir y définir une notion de \textbf{longueur}.
\begin{defn}
Soit $I$ un intervalle de $\mathbf{R}$ et $\gamma:I\to\mathcal{M}$ une courbe lisse.\newline La \textbf{longueur de $\gamma$}, notée $L(\gamma)$, est définie par :
$$L(\gamma)=\int_{I}\sqrt{g_{\gamma(t)}(\dot{\gamma}(t),\dot{\gamma}(t))}\mathrm{d}t.$$
\end{defn}
\begin{rqe}
La longueur de $\gamma$ est invariante par reparamétrage de $\gamma$.\newline On rappelle d'un reparamétrage de $\gamma$ est un difféomorphisme lisse croissant $\varphi:\widetilde{I}\to I$ où $\widetilde{I}$ est un intervalle de $\mathbf{R}$.\newline La reparamétrée de $(I,\gamma)$ par $\varphi$ est la courbe $\left(\widetilde{I},\widetilde{\gamma}\right)$ où $\widetilde{\gamma}=\gamma\circ\varphi$.
\end{rqe}
Intéressons-nous aux courbes de longueurs minimales. Nous verrons que toute courbe de longueur minimale est une géodésique à reparamétrage près.\newline Introduisons d'abord la notion d'abscisse curviligne.
\begin{defn}
Soit $[t_0,t_1]$ un segment de $\mathbf{R}$ et $\gamma:[t_0,t_1]\to\mathcal{M}$ lisse régulière (ie telle que $\forall t\in[t_0,t_1], g_{\gamma(t)}(\dot{\gamma}(t),\dot{\gamma}(t))\neq 0$.\newline L'application $$\displaystyle\tau:[t_0,t_1]\to[0,L(\gamma)],t\mapsto\int_{t_0}^t\sqrt{g_{\gamma(s)}(\dot{\gamma}(s),\dot{\gamma}(s))}\mathrm{d}s$$est un $\mathcal{C}^\infty$-difféomorphisme appelé \textbf{abscisse curviligne de $\gamma$}.\newline La reparamétrée $\widetilde{\gamma}$ de $\gamma$ par $\tau^{-1}$ vérifie $\forall s\in [0,L(\gamma)],g_{\widetilde{\gamma}(s)}(\dot{\widetilde{\gamma}}(s),\dot{\widetilde{\gamma}}(s))=1$ (autrement dit, son vecteur vitesse est unitaire).
\end{defn}
\begin{defn}
Une courbe lisse régulière paramétrée par abscisse curviligne $\gamma:[t_0,t_1]\to\mathcal{M}$ est dite \textbf{minimisante} si, pour toute famille lisse de courbes $(\gamma_{\varepsilon})_{\varepsilon\in[-\varepsilon_0,\varepsilon_0]}$ définies sur $[t_0,t_1]$ vérifiant \newline$\forall \varepsilon\in[-\varepsilon_0,\varepsilon_0],(\gamma_\varepsilon(t_0),\gamma_\varepsilon(t_1))=(\gamma(t_0),\gamma(t_1))$ et $\gamma_0=\gamma$, $\varepsilon\mapsto L(\gamma_\varepsilon)$, la fonction $\varepsilon\mapsto L(\gamma_\varepsilon)$ admet un minimum en $0$.
\end{defn}
\newpage
\begin{thm}
Toute courbe minimisante est une géodésique à reparamétrage près.
\end{thm}
\begin{dem}
On reprend les mêmes notations que précédemment.\newline
On a, en dérivant sous l'intégrale dans un système de coordonnées (une telle écriture est licite grâce à un argument de partitions de l'unité, qu'on verra ultérieurement) :
\begin{align*}
&\left.\dfrac{\mathrm{d}}{\mathrm{d}\varepsilon}L(\gamma_\varepsilon)\right|_{\varepsilon=0}\\&=\int_{t_0}^{t_1}\dfrac{1}{2\Vert\dot{\gamma}(t)\Vert}\left(\partial_kg_{ij}(\gamma(t))\left.\dfrac{\mathrm{d}\gamma^k_\varepsilon(t)}{\mathrm{d}\varepsilon}\right|_{\varepsilon=0}\dot{\gamma}^i(t)\dot{\gamma}^j(t)+2g_{ij}(\gamma(t))\dot{\gamma}^i(t)\left.\dfrac{\mathrm{d}\dot{\gamma}_\varepsilon^j(t)}{\mathrm{d}\varepsilon}\right|_{\varepsilon=0}\right)\mathrm{d}t
\end{align*}
où on note, pour plus de lisibilité, $\Vert\dot{\gamma}(t)\Vert=\sqrt{g_{\gamma(t)}(\dot{\gamma}(t),\dot{\gamma}(t))}$.\newline
On intègre par parties le second terme (le terme croisé est nul puisque les $\gamma_\varepsilon$ ont les mêmes extrémités) :
$$\int_{t_0}^{t_1}\dfrac{1}{\Vert\dot{\gamma}\Vert}g_{ij}\dot{\gamma}^i\left.\dfrac{\mathrm{d}\dot{\gamma_\varepsilon}^j}{\mathrm{d}\varepsilon}\right|_{\varepsilon=0}=-\int_{t_0}^{t_1}\dfrac{\mathrm{d}}{\mathrm{d}t}\left(\dfrac{1}{\Vert\dot{\gamma}\Vert}g_{ij}\dot{\gamma}^i\right)\left.\dfrac{\mathrm{d}\gamma^i_\varepsilon}{\mathrm{d}\varepsilon}\right|_{\varepsilon=0}$$
Les réels $\left.\dfrac{\mathrm{d}\gamma_\varepsilon^k(t)}{\mathrm{d}\varepsilon}\right|_{\varepsilon=0}$ pouvant être choisis arbitrairement (grâce à un argument de partitions de l'unité, qu'on verra ultérieurement), on en déduit :
$$
\dfrac{1}{2\Vert\dot{\gamma}\Vert}\partial_kg_{ij}\dot{\gamma}^i\dot{\gamma}^j=
\dfrac{\mathrm{d}}{\mathrm{d}t}\left(\dfrac{1}{\Vert\dot{\gamma}\Vert}g_{ik}\dot{\gamma}^i\right)
$$
donc :
$$
\dfrac{1}{2}\partial_kg_{ij}\dot{\gamma}^i\dot{\gamma}^j=-\dfrac{1}{\Vert\dot{\gamma}\Vert}\dfrac{\mathrm{d}\Vert\dot{\gamma}\Vert}{\mathrm{d}t}g_{ik}\dot{\gamma}^i+\dfrac{\mathrm{d}g_{ik}}{\mathrm{d}t}\dot{\gamma}^i+g_{ik}\ddot{\gamma}^i.
$$
Or $\dfrac{\mathrm{d}g_{ik}}{\mathrm{d}t}=\partial_lg_{ik}\dot{\gamma}^l$, donc, en renommant les indices muets :
$$
\dfrac{1}{2}\partial_kg_{ij}\dot{\gamma}^i\dot{\gamma}^j=-\dfrac{1}{\Vert\dot{\gamma}\Vert}\dfrac{\mathrm{d}\Vert\dot{\gamma}\Vert}{\mathrm{d}t}g_{ik}\dot{\gamma}^i+\partial_jg_{ik}\dot{\gamma}^i\dot{\gamma}^j+g_{ik}\ddot{\gamma}^i,
$$
donc, en renommant $i$ en $l$ :
$$g_{lk}\ddot{\gamma}^l-\dfrac{1}{2}\partial_kg_{lj}\dot{\gamma}^l\dot{\gamma}^j+\partial_jg_{lk}\dot{\gamma}^l\dot{\gamma}^j=\dfrac{1}{\Vert\dot{\gamma}\Vert}\dfrac{\mathrm{d}\Vert\dot{\gamma}\Vert}{\mathrm{d}t}g_{lk}\dot{\gamma}^l,$$
puis, en contractant avec $g^{ik}$ :
$$
\ddot{\gamma}^i+g^{il}\left(-\dfrac{1}{2}\partial_kg_{lj}+\partial_jg_{lk}\right)\dot{\gamma}^l\dot{\gamma}^j=\dfrac{1}{\Vert\dot{\gamma}\Vert}\dfrac{\mathrm{d}\Vert\dot{\gamma}\Vert}{\mathrm{d}t}\dot{\gamma}^i.
$$
On reconnaît les symboles de \bsc{Christoffel}, de sorte que :
$$
\ddot{\gamma}^i+\Gamma_{lj}^i\dot{\gamma}^l\dot{\gamma}^j=\dfrac{1}{\Vert\dot{\gamma}\Vert}\dfrac{\mathrm{d}\Vert\dot{\gamma}\Vert}{\mathrm{d}t}\dot{\gamma}^i.
$$
Ainsi, en reparamétrant $\gamma$ par son abscisse curviligne, le membre de droite devient nul : on retrouve l'équation des géodésiques.
\end{dem}
\begin{rqe}
Toute géodésique n'est pas minimisante. En effet, les méridiens de la sphère sont des géodésiques, mais ils ne sont clairement pas toujours minimisants.
\end{rqe}
\subsection{Variétés lorentziennes}
\begin{defn}
Une \textbf{variété lorentzienne} est un couple $(\mathcal{M},g)$ où $\mathcal{M}$ est une variété différentielle lisse et $g$ est une métrique lisse de signature $(-1,1,\ldots,1)$.
\end{defn}
Jusqu'à la fin de ce paragraphe, on se donne $(\mathcal{M},g)$ une variété lorentzienne de dimension $n$ (concrètement, en relativité générale, on considère un espace-temps de dimension $4$, et de dimension supérieure en théorie des cordes).
L'intérêt de ce genre de structure est de munir un espace-temps $\mathcal{M}$ d'une notion de causalité.
\begin{defn}
Soit $p\in\mathcal{M}$. On dit qu'un vecteur $v\in T_p\mathcal{M}$ est du genre :
\begin{itemize}
\item \textbf{temps} si $g_p(v,v)<0$ ;
\item \textbf{espace} si $g_p(v,v)>0$ ;
\item \textbf{lumière} si $g_p(v,v)=0$.
\end{itemize}
Une courbe lisse $\gamma:I\to\mathcal{M}$ (où $I$ est un intervalle de $\mathbf{R}$) est dite du genre temps (resp. espace, resp. lumière) si, pour tout $t\in I$, $\dot{\gamma}(t)$ est du genre temps (resp. espace, resp. lumière).
\end{defn}
\begin{rqe}
Les courbes du genre lumière (resp. temps) représentent les trajectoires de la lumière (resp. des particules matérielles). Par postulat, une courbe $\gamma$ représentant la trajectoire de la lumière ou d'une particule en chute libre respectent l'équation des géodésiques $\nabla_{\dot{\gamma}}\dot{\gamma}=0$ où $\nabla$ est la connexion de \bsc{Levi}-\bsc{Civita} associée à $g$.
\end{rqe}
\begin{ex}
Dans cet exemple, on prend $\mathcal{M}=\mathbf{R}^{n+1}$, muni de la métrique de \bsc{Minkowski}$$g=-\left(\mathrm{d}x^0\right)^2+\left(\mathrm{d}x^1\right)^2+\cdots+\left(\mathrm{d}x^n\right)^2$$en prenant au préalable comme système de coordonnées globales la base duale de la base canonique de $\mathbf{R}^{n+1}$.

\medskip

Les coordonnées de $g$ dans $(\mathrm{d}x^0,\ldots,\mathrm{d}x^n)$ étant constantes, on en déduit que les coefficients de \bsc{Christoffel} de la connexion de \bsc{Levi}-\bsc{Civita} sont nuls, donc les géodésiques sont des droites.
\end{ex}
\newpage
\section{Intermède : partitions de l'unité}
On donne dans cette section une preuve du théorème de partition de l'unité pour une variété différentielle, qui est particulièrement utile pour étendre des objets ou encore effectuer des recollements de façon régulière.\newline On se donne $\mathcal{M}$ une variété différentielle de classe $\mathcal{C}^k$ avec $k\in\mathbf{N}^*\cup\{\infty\}$.
\begin{defn}
Soit $(U_i)_{i\in I}$ une famille d'ouverts de $\mathcal{M}$.\newline Une \textbf{partition de l'unité associée à $(U_i)_{i\in I}$} est une famille de fonctions lisses $(\rho_i)_{i\in I}$ définies sur $\mathcal{M}$ et à valeurs dans $\mathbf{R}^+$ telle que :
\begin{itemize}
\item $\forall i\in I, \Supp(\rho_i)\subset U_i$
\item $(\Supp(\rho_i))_{i\in I}$ est localement finie (ie chaque point de $\mathcal{M}$ possède un voisinage sur lequel un nombre fini de $\rho_i$ ne sont pas identiquement nuls)
\item $\displaystyle\forall p\in\mathcal{M}, \sum\limits_{i\in I}\rho_i(p)=1$
\end{itemize}
\end{defn}
\begin{thm}
Soit $(U_i)_{i\in I}$ un recouvrement de $\mathcal{M}$ par des ouverts.\newline Il existe une partition de l'unité subordonnée à $\mathcal{M}$.
\end{thm}
\begin{dem}
Classiquement, en considérant
$$\mathbf{R}^d\to\mathbf{R},x\mapsto\left\{
    \begin{array}{ll}
        \exp\left(-\dfrac{1}{1-\Vert x\Vert^2}\right) & \mbox{si } \Vert x\Vert<1 \\
        0 & \mbox{sinon}
    \end{array}
\right.,$$
on construit une fonction positive lisse à support compact définie sur $\mathbf{R}^d$. En dilatant et en translation, pour tout point $x$ de $\mathbf{R}^d$, pour tout ouvert $U$ de $\mathbf{R}^d$ contenant $x$, il existe une fonction positive lisse à support compact inclus dans $U$.

\medskip

Comme $\mathcal{M}$ est $\sigma$-compacte, il existe un recouvrement dénombrable de $\mathcal{M}$ par des compacts $(K_n)_{n\in\mathbf{N}}$ vérifiant $\forall n\in\mathbf{N}, K_n\subset\mathring{K_{n+1}}$.\newline Pour tout $n\in\mathbf{N}$, pour tout $x\in\overline{K_{n+1}\backslash K_n}$, il existe un ouvert $V_x$ inclus dans $K_{n+2}\backslash K_{n-1}$ et dans un des $U_i$, et $f_x:V_x\to\mathbf{R}^+$ de même régularité que la variété et à support compact.\newline La famille $\left(f_x^{-1}(\mathbf{R}_*^+)\right)_{x\in \overline{K_{n+1}\backslash K_n}}$ est un recouvrement du compact $\overline{K_{n+1}\backslash K_n}$ par des ouverts. On peut en extraire un recouvrement fini $\left(f_x^{-1}(\mathbf{R}_*^+)\right)_{x\in A_n}$.\newline La famille $\displaystyle(f_x^{-1}(\mathbf{R}^+_*))_{x\in\bigcup\limits_{n\in\mathbf{N}}A_n}$ est un recouvrement de $\mathcal{M}$ localement fini tel que, pour tout $x\in\bigcup\limits_{n\in\mathbf{N}}A_n$, $f^{-1}_x(\mathbf{R}^+_*)$ est inclus dans un $U_{i(x)}$. On pose, pour tout $i\in I$ :
$$\rho_i=\dfrac{1}{\displaystyle\sum\limits_{x\in\bigcup\limits_{n\in\mathbf{N}}A_n}f_x}\sum\limits_{x\in i^{-1}(\{x\})}f_x,$$
et on vérifie que $(\rho_i)_{i\in I}$ convient.
\end{dem}
\newpage
On donne une application importante de ce résultat.
\begin{thm}[Théorème de \bsc{Whitney}]
Soit $\mathcal{M}$ une variété différentielle de classe $\mathcal{C}^k$ compacte. Il existe $N\in\mathbf{N}$ et $\varphi:\mathcal{M}\to\mathbf{R}^N$ un plongement.
\end{thm}
\begin{dem}
Comme $\mathcal{M}$ est compacte, on peut la recouvrir d'un nombre fini d'atlas $((U_i,\phi_i))_{i\in\llbracket 1,m\rrbracket}$. On note $(\rho_i)_{i\in\llbracket 1,m\rrbracket}$ une partition de l'unité associée à $(U_i)_{i\in\llbracket 1,m\rrbracket}$. En retravaillant la preuve précédente, on montre qu'on peut choisir $(\rho_i)_{i\in\llbracket 1,m\rrbracket}$ de sorte que, pour tout $i\in\llbracket 1,m\rrbracket$, $\rho_i$ est constante égale à $1$ sur un ouvert $V_i$ inclus dans $U_i$, avec $(V_i)_{i\in\llbracket 1,m\rrbracket}$ recouvrant $\mathcal{M}$. On peut étendre chaque carte $\phi_i$ à $\mathcal{M}$ tout entier en considérant $\phi_i\rho_i$. On pose ainsi :
$$\varphi:\mathcal{M}\to\mathbf{R}^n\times\cdots\times\mathbf{R}^n\times\mathbf{R}\times\cdots\times\mathbf{R},p\mapsto\left((\phi_1\rho_1)(p),\ldots,(\phi_m\rho_m)(p),\rho_1(p),\ldots,\rho_m(p)\right)$$
où $n$ est la dimension de $\mathcal{M}$. On montre facilement, en raisonnant sur chaque $V_i$, que $\varphi$ est un plongement de $\mathcal{M}$ dans $\mathbf{R}^{(n+1)m}$.
\end{dem}
\newpage
\section{Variétés orientées}
\subsection{Généralités}
On se donne $\mathcal{M}$ un espace topologique séparé et $\sigma$-compact. Toutes les cartes considérées ici sont à valeurs dans $\mathbf{R}^n$ avec $n\in\mathbf{N}$, et les changements de cartes sont supposés lisses.
\begin{defn}
Deux cartes $(U,\phi)$ et $(V,\psi)$ sur $\mathcal{M}$ ont la même orientation si :$$\Jac(\phi\circ\psi^{-1})>0.$$
\end{defn}
\begin{rqe}
L'orientation définit une relation d'équivalence sur l'ensemble des cartes sur $\mathcal{M}$.
\end{rqe}
\begin{defn}
Un atlas sur $\mathcal{M}$ \textbf{définit une orientation} si toutes ses cartes ont la même orientation. Une variété différentielle dont l'atlas définit une orientation est dite \textbf{orientée}.
\end{defn}
Par la suite, on se donne $\mathcal{M}$ une variété lisse de dimension $n$.
\begin{defn}
Une \textbf{forme volume} sur $\mathcal{M}$ est un élément non nul de $\Omega^n\mathcal{M}$.
\end{defn}

\newpage
\section{Applications physiques}
\subsection{Principe de moindre action}
Le but de ce paragraphe est de généraliser l'approche faite précédemment pour montrer que les courbes minimisantes sont des géodésiques à reparamétrage près.

\medskip

Un \textbf{système physique} est un couple $(\mathcal{M},L)$ où $\mathcal{M}$ est une variété lisse et où\newline $L:T\mathcal{M}\times\mathbf{R}\to\mathbf{R},(q,v,t)\mapsto L(q,v,t)$ de classe $\mathcal{C}^\infty$.

L'\textbf{action} $S$ est définie par :
$$S(\gamma)=\int_{t_1}^{t_2}L(\gamma(t),\dot{\gamma}(t),t)\mathrm{d}t$$
avec $\gamma:[t_1,t_2]\to\mathcal{M}$ une courbe lisse.

On pose $\mathcal{P}_{t_1,q_1}^{t_2,q_2}$ l'ensemble des courbes lisses $\gamma:[t_1,t_2]\to\mathcal{M}$ tels que $\gamma(t_1)=q_1$ et $\gamma(t_2)=q_2$.

Étant donné $\gamma\in\mathcal{P}_{t_1,q_1}^{t_2,q_2}$, une \textbf{variation de $\gamma$} est une application lisse
$$
[t_1,t_2]\times[-\varepsilon_0,\varepsilon_0]\to\mathcal{M},(t,\varepsilon)\mapsto\gamma_\varepsilon(t)
$$
telle que $\forall \varepsilon\in [-\varepsilon_0,\varepsilon_0],\gamma_\varepsilon\in\mathcal{P}_{t_1,q_1}^{t_2,q_2}$ et $\gamma_0=\gamma$.

Une chemin $\gamma\in\mathcal{P}_{t_1,q_1}^{t_2,q_2}$ est dit \textbf{stationnaire} si, pour toute variation\newline $(\gamma_\varepsilon(t))_{(t,\varepsilon)\in [t_1,t_2]\times[-\varepsilon_0,\varepsilon_0]}$ :
$$
\left.\dfrac{\mathrm{d}S(\gamma_\varepsilon)}{\mathrm{d}\varepsilon}\right|_{\varepsilon=0}=0.
$$
Cette définition est plus connue sous le nom de \textbf{principe de moindre action}.
Soit $\gamma$ un chemin stationnaire et $\gamma_\varepsilon(t)$ une variation de $\gamma$. On a, en dérivant sous l'intégrale, dans un système de coordonnées :
$$
\int_{t_1}^{t_2}\left(\dfrac{\partial L}{\partial q^i}(\gamma(t),\dot{\gamma}(t),t)\left.\dfrac{\mathrm{d}\gamma^i_\varepsilon(t)}{\mathrm{d}\varepsilon}\right|_{\varepsilon=0}+\dfrac{\partial L}{\partial v^i}(\gamma(t),\dot{\gamma}(t),t)\left.\dfrac{\mathrm{d}\dot{\gamma_\varepsilon^i}(t)}{\mathrm{d}\varepsilon}\right|_{\varepsilon=0}\right)\mathrm{d}t=0.
$$
En intégrant par parties le second membre :
$$
\int_{t_1}^{t_2}\dfrac{\partial L}{\partial v_i}(\gamma(t),\dot{\gamma}(t),t)\left.\dfrac{\mathrm{d}\dot{\gamma_\varepsilon^i}(t)}{\mathrm{d}\varepsilon}\right|_{\varepsilon=0}\mathrm{d}t=-\int_{t_1}^{t_2}\dfrac{\mathrm{d}}{\mathrm{d}t}\dfrac{\partial L}{\partial v^i}(\gamma(t),\dot{\gamma}(t),t)\left.\dfrac{\mathrm{d}\gamma_{\varepsilon}^i(t)}{\mathrm{d}\varepsilon}\right|_{\varepsilon=0}\mathrm{d}t,
$$
donc :
$$\int_{t_1}^{t_2}\left(\dfrac{\partial L}{\partial q^i}(\gamma(t),\dot{\gamma}(t),t)-\dfrac{\mathrm{d}}{\mathrm{d}t}\dfrac{\partial L}{\partial v^i}(\gamma(t),\dot{\gamma}(t),t)\right)\left.\dfrac{\mathrm{d}\gamma_\varepsilon^i(t)}{\mathrm{d}\varepsilon}\right|_{\varepsilon=0}\mathrm{d}t=0.$$
La relation précédente étant vraie pour toute variation de $\gamma$, on en déduit :
$$\forall i\in\llbracket 1,n\rrbracket, \dfrac{\partial L}{\partial q^i}(\gamma(t),\dot{\gamma}(t),t)=\dfrac{\mathrm{d}}{\mathrm{d}t}\dfrac{\partial L}{\partial v^i}(\gamma(t),\dot{\gamma}(t),t).
$$
Il s'agit des \textbf{équations d'Euler-Lagrange}.
\newpage
On suppose que $\mathcal{M}=\mathbf{R}^n$.
Il est commode de noter $\mathbf{p}=\dfrac{\partial L}{\partial\dot{\mathbf{q}}}$ l'\textbf{impulsion} associée à $L$, de sorte que l'équation d'\bsc{Euler}-\bsc{Lagrange} se réécrit :
$$
\dfrac{\mathrm{d}\mathbf{p}}{\mathrm{d}t}=\dfrac{\partial L}{\partial\mathbf{q}}.
$$
On voit clairement que $\dfrac{\partial L}{\partial\mathbf{q}}$ joue le rôle de la résultante des forces.

On suppose que le système est classique et conservatif de masse $m$, placé dans un potentiel $V$. Le lagrangien $ L(\mathbf{q},\dot{\mathbf{q}})=\dfrac{1}{2}m\dot{\mathbf{q}}^2-V(\mathbf{q})$ (qui traduit une compétition entre effets cinétiques et force) permet de retrouver le principe fondamental de la dynamique.\newline
L'équation d'\bsc{Euler}-\bsc{Lagrange} donne, pour un chemin rendant l'action stationnaire :
$$
\dfrac{\mathrm{d}\mathbf{p}}{\mathrm{d}t}=-\dfrac{\partial V}{\partial\mathbf{q}}.
$$
\newpage
\subsection{Théorème de Noether}
Soit $(\mathcal{M},L)$ un système physique avec $L$ indépendant du temps.\newline
Un \textbf{groupe de symétries} de $(\mathcal{M},L)$ est une application lisse\newline $g:\mathbf{R}\times\mathcal{M}\to\mathcal{M},(s,q)\mapsto g_s(q)$ telle que :
$$
\forall s,t\in\mathbf{R}, g_{s+t}=g_s\circ g_t\text{ et }g_0=\Id_\mathcal{M}.
$$
vérifiant : $\forall s\in\mathbf{R},\forall (q,v,t)\in T\mathcal{M}\times\mathbf{R}, L(q,v)=L(g_s(q),T_qg_s(v))$.
On se donne $\gamma$ un chemin lisse rendant l'action stationnaire. On a, pour tout $t\in\mathbf{R}$, en dérivant l'égalité précédente par rapport à $s$, dans un système de coordonnées :
$$
\dfrac{\partial L}{\partial q^i}(\gamma(t),\dot{\gamma}(t))\left.\dfrac{\mathrm{d}g_s(\gamma(t))^i}{\mathrm{d}s}\right|_{s=0}+\dfrac{\partial L}{\partial v^i}(\gamma(t),\dot{\gamma}(t))\left.\dfrac{\mathrm{d}T_{\gamma(t)}g_s(\dot{\gamma}(t))^i}{\mathrm{d}s}\right|_{s=0}=0
$$
Or :
$$\forall (q,v)\in T\mathcal{M},\forall i\in\llbracket 1,n\rrbracket,\left.\dfrac{\mathrm{d}T_qg_s(v)^i}{\mathrm{d}s}\right|_{s=0}=\left.\dfrac{\mathrm{d}}{\mathrm{d}s}\left(v^j\left.\dfrac{\partial g_s^i}{\partial q^j}\right|_{q}\right)\right|_{s=0}=v^j\left.\dfrac{\partial}{\partial q^j}\left(\left.\dfrac{\mathrm{d}g_s^i}{\mathrm{d}s}\right|_{s=0}\right)\right|_{q},$$
donc :
$$
\forall i\in\llbracket 1,n\rrbracket,\left.\dfrac{\mathrm{d}T_{\gamma(t)}g_s(\dot{\gamma}(t))^i}{\mathrm{d}s}\right|_{s=0}=\dfrac{\mathrm{d}}{\mathrm{d}t}\left(\left.\dfrac{\mathrm{d}g_s(\gamma(t))^i}{\mathrm{d}s}\right|_{s=0}\right),
$$
donc, en réinjectant dans l'équation initiale et en utilisant les équations d'\bsc{Euler}-\bsc{Lagrange} :
$$
\dfrac{\mathrm{d}I(\gamma(t),\dot{\gamma}(t))}{\mathrm{d}t}=0
$$
avec $I$ définie par : $\forall (q,v)\in T\mathcal{M},I(q,v)=\dfrac{\partial L}{\partial v^i}(q,v)\left.\dfrac{\mathrm{d}g_s(q)^i}{\mathrm{d}s}\right|_{s=0}$.\newline Ainsi, à tout groupe de symétrie d'un système physique correspond une constante du mouvement.
Nous venons d'établir le \textbf{théorème de Noether}, qui heuristiquement s'énonce ainsi :
\begin{center}
\og À toute symétrie d'un système physique correspond une quantité conservée. \fg{}
\end{center}
Voyons quelques exemples d'applications. On suppose que $\mathcal{M}=\mathbf{R}^n$.\newline On se donne un système classique conservatif de masse $m$ placé dans un potentiel $V$.
\begin{itemize}
\item On suppose que le système est invariant sous l'action de $g_s:\mathbf{q}\mapsto\mathbf{q}+s\mathbf{u}$ avec $\mathbf{u}$ un vecteur unitaire fixé. Le théorème de \bsc{Noether} donne que $\mathbf{p}\cdot\mathbf{u}$ est conservé (autrement dit, l'impulsion du système dans la direction $\mathbf{u}$ est constante).
\item On suppose que le système est décrit par trois coordonnées et est invariant sous l'action de $g_\theta:\mathbf{q}\mapsto R_\mathbf{u}(\theta)\mathbf{q}$ avec $\mathbf{u}$ un vecteur unitaire fixé, $R_\mathbf{u}(\theta)$ désignant la rotation d'axe $\mathbf{u}$ et d'angle $\theta$. Le théorème de \bsc{Noether} donne que $\mathbf{p}\cdot\left.\dfrac{\mathrm{d}R_\mathbf{u}(\theta)\mathbf{q}}{\mathrm{d}\theta}\right|_{\theta=0}$ est conservée. Or, à l'ordre $1$ en $\theta$ : $R_\mathbf{u}(\theta)\mathbf{q}=\mathbf{q}+\theta\mathbf{u}\times\mathbf{q}$, donc $\left.\dfrac{\mathrm{d}R_\mathbf{u}(\theta)\mathbf{q}}{\mathrm{d}\theta}\right|_{\theta=0}=\mathbf{u}\times\mathbf{q}$. Ainsi, la quantité $\mathbf{p}\cdot (\mathbf{u}\times\mathbf{q})=\mathbf{u}\cdot(\mathbf{p}\times\mathbf{q})$ est conservée. Il s'agit du moment cinétique du système $\mathbf{L}=\mathbf{q}\times\mathbf{p}$ dans la direction $\mathbf{u}$.
\end{itemize}
\subsection{Un bref historique de la mécanique}
D'une manière générale, une modélisation de l'univers est la donnée d'une variété différentielle et d'un ensemble de lois physiques invariantes sous l'action d'un certain groupe de symétries.

\medskip

La mécanique newtonienne repose sur une modélisation de l'univers par la variété de dimension 4 $\mathbf{R}\times E$ où $E$ est l'espace euclidien affine de dimension $3$.\newline Le groupe de symétrie des lois physiques est le groupe engendré par les translations spatio-temporelles, les isométries spatiales, et les transformations de \bsc{Galilée}.\newline Ces différentes transformations correspondent également aux changements de référentiels inertiels. L'avènement de l'électromagnétisme a remis en question cette modélisation. En effet, les équations de \bsc{Maxwell} mettent en évidence que la vitesse de la lumière dans le vide, dans tout référentiel inertiel, est la même, ce qui n'est pas compatible avec les transformations de \bsc{Galilée} (autrement dit, les équations de \bsc{Maxwell} ne sont pas invariantes par transformation de \bsc{Galilée}). Des années plus tard, \bsc{Poincaré} et \bsc{Lorentz} mettent en évidence une nouvelle classe de transformations compatibles avec les équations de \bsc{Maxwell}, les transformations de \bsc{Lorentz}.

\medskip

En remplaçant les transformations de \bsc{Galilée} par les transformations de \bsc{Lorentz}, on obtient le \textbf{groupe de Lorentz} (dont la version affine est le \textbf{groupe de Poincaré}), qui a la particularité d'être le groupe orthogonal de la forme quadratique $-t^2+x^2+y^2+z^2$, appelée métrique de \bsc{Minkowski}. Cela invite à modéliser l'univers comme une variété lorentzienne de dimension $4$ (autrement dit une variété différentielle munie d'une métrique de signature $(-+++)$, munie de la métrique de \bsc{Minkowski} $m=-\mathrm{d}t\otimes\mathrm{d}t+\mathrm{d}x\otimes\mathrm{d}x+\mathrm{d}y\otimes\mathrm{d}y+\mathrm{d}z\otimes\mathrm{d}z$. Ce dernier cadre est celui de la \textbf{relativité restreinte}, qui correspond à un univers vide, sans gravitation.

\medskip

Dans le cas d'un univers avec gravitation, on modélise l'univers par une variété lorentzienne de dimension $4$ vérifiant l'équation d'\bsc{Einstein} : $\Ric_{ij}-\dfrac{1}{2}g_{ij}\Rscal=T_{ij}$ où $T_{ij}$ est le \textbf{tenseur énergie-impulsion}, représentant le contenu en matière-énergie de l'espace-temps. On obtient ainsi la théorie de la \textbf{relativité générale}, au sein de laquelle la gravitation joue un rôle singulier : elle est modélisée comme une courbure de l'espace-temps.
\newpage
\subsection{Action de Hilbert-Einstein}
Le but de cette sous-section est de donner une preuve des équations d'\bsc{Einstein} à l'aide du principe de moindre action.\newline
La densité de lagrangien de l'univers $\mathcal{L}$ peut être décomposé en somme de deux densités : une qui correspond à la gravitation $\mathcal{L}(g)$ (qui ne dépend que du tenseur métrique), et une autre qui correspond au contenu en matière et en énergie $\mathcal{L}(g,\psi)$ où $\psi$ est un champ décrivant le contenu en matière-énergie (par exemple le champ électromagnétique). L'action totale vaut :
$$S=\int_{\mathcal{M}}\left(\mathcal{L}_G(g)+\mathcal{L}_M(g,\psi)\right)\sqrt{-\det(g_{ij})}\mathrm{d}x.$$
Cette dernière écriture peut être rendue rigoureuse au moyen de partitions de l'unité (nous ne nous attarderons pas sur ce détail ici).\newline
La quantité scalaire invariante par changement de coordonnées la plus simple à considérer est le tenseur de \bsc{Ricci} scalaire, ce qui invite à poser $\mathcal{L}_G(g)=\Rscal$.\newline La solution au problème $(g,\psi)$ doit rendre extrémale l'action totale. Autrement dit, la variation à l'ordre $1$ de l'action doit être nulle en faisant varier $g$ et $\psi$.\newline On se donne $h$ une métrique lorentzienne.\newline La dérivée de $S$ dans la direction $h$ vaut :
\begin{align*}
\delta_gS(h)=&\int_\mathcal{M}\left(\delta_g\mathcal{L}_G(h)+\delta_g\mathcal{L}_M(h)\right)\sqrt{-\det(g_{ij})}\mathrm{d}x\\&+\int_\mathcal{M}\left(\mathcal{L}_G(g)+\mathcal{L}_M(g,\psi)\right)\delta_g\left(\sqrt{-\det(g_{ij})}\right)(h)\mathrm{d}x
\end{align*}
Calculons les différents termes. On a :
$$
\delta_g\mathcal{L}_G(h)=\delta_g\left(g^{ij}\right)(h)\Ric_{ij}+g^{ij}\delta_g(\Ric_{ij})(h).
$$
Or :
\begin{align*}
\left((g+th)^{ij}\right)&=((g+th)_{ij})^{-1}=((g_{ij})+t(h_{ij}))^{-1}\\&=(g^{ij})-t(g^{ij})(g^{ij})(h_{ij})+o(t)=(g^{ij})-t(h^{ij})+o(t)
\end{align*}
donc $\delta_g\left(g^{ij}\right)(h)=-h^{ij}$.
De plus, en se plaçant en un point $p$ de $\mathcal{M}$ et en utilisant un système de coordonnées normales associé à ce point :
$$
\delta_g\left(\Ric_{ij}\right)(h)=\delta_g\left(\partial_i\Gamma^k_{jk}-\partial_j\Gamma^k_{ik}\right)(h)=\partial_i\delta_g\left(\Gamma^k_{jk}\right)(h)-\partial_j\delta_g\left(\Gamma^k_{ik}\right)(h)
$$
L'expression $\delta_g\left(\Gamma_{jk}^k\right)(h)$ peut être interprétée comme la différence infinitésimale de deux symboles de \bsc{Christoffel}, donc est un tenseur. Ainsi, l'expression précédente de $\delta_g\left(\Ric_{ij}\right)(h)$ ne dépend pas du choix du système de coordonnées.\newline
En intégrant sur $\mathcal{M}$, le terme correspondant à $\delta_g(\Ric_{ij})(h)$ est nul.
\newpage
De plus : 
$$
\delta_g\left(\sqrt{-\det(g_{ij})}\right)(h)=\dfrac{1}{2\sqrt{-\det(g_{ij})}}\delta_g\left(\det(g_{ij})\right)(h),
$$
avec $\det(g_{ij}+th_{ij})=\det(g_{ij})(1+t\Tr((g^{ij})(h_{ij})))+o(t)=\det(g_{ij})(1+tg^{kl}h_{kl})+o(t)$, donc :
$$
\delta_g\left(\sqrt{-\det(g_{ij})}\right)(h)=\dfrac{1}{2}\sqrt{-\det(g_{ij})}g^{kl}g_{ki}g_{lj}h^{ij}=\dfrac{1}{2}\sqrt{-\det(g_{ij})}g_{ij}h^{ij}.
$$
Ainsi :
$$
\delta_gS(h)=\int_\mathcal{M}\left(-\Ric_{ij}+\dfrac{1}{2}g_{ij}\Rscal+\dfrac{1}{2}g_{ij}\mathcal{L}_M-\partial_{g_{ij}}\mathcal{L}_M\right)\sqrt{-\det(g_{ij})}h^{ij}\mathrm{d}x.
$$
Cette quantité étant nulle pour tout $h$, on en déduit l'équation d'\bsc{Einstein} :
$$
\Ric_{ij}-\dfrac{1}{2}g_{ij}\Rscal=T_{ij}
$$
où $T_{ij}=\dfrac{1}{2}g_{ij}\mathcal{L}_M-\partial_{g_{ij}}\mathcal{L}_M$ est le tenseur énergie-impulsion.

\medskip

On se place dans le vide, de sorte que le tenseur énergie-impulsion est nul. En contractant, on obtient $\Rscal=0$, puis, en réinjectant dans les équations d'\bsc{Einstein} :$$\Ric_{ij}=0.$$
\end{document}
