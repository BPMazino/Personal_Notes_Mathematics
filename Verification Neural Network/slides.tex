\documentclass{beamer}
\usepackage{preamble}

%Information to be included in the title page:
\title{My TER project on Neural Network Verification}
\author{Nassim Arifette}
\institute{Université Paris-Saclay}
\date{2024}

\begin{document}

\frame{\titlepage}

\begin{frame}
\frametitle{How was my internship structured?}
    \begin{itemize}
    \item First part: Take a course on Neural Network Verification at LIX (École Polytechnique)
    \item Second part: Litterature review on the subject and reading the main paper
    \item Third part: Implementing the main algorithm in Julia
    \item Fourth part: Testing the algorithm on some benchmarks
    \end{itemize}
\end{frame}

\begin{frame}
\frametitle{What is Neural Network Verification?}
    \begin{itemize}
    \item Neural Network Verification is the process of verifying that a neural network satisfies a given property.
    \item The property can be anything, for example: "The network classifies all images of cats as cats even if they are slightly modified."
    \end{itemize}

    \begin{figure}
        \centering
        \begin{minipage}{0.40\linewidth}
            \centering
            \includegraphics[width=\linewidth]{image.png}
            %\caption{Image of cat}
            \label{fig:cat1}
        \end{minipage}
        \hfill
        \begin{minipage}{0.40\linewidth}
            \centering
            \includegraphics[width=\linewidth]{cat_dark.png}
            %\caption{Image of cat darker}
            \label{fig:cat2}
        \end{minipage}
    \end{figure}
\end{frame}

\begin{frame}
\frametitle{Formally}
    We define a neural network as a function $f: \mathbb{R}^n \rightarrow \mathbb{R}^m$.
    And we define a property as a set $X \subset \mathbb{R}^n$.
        \[X = \{x \in \mathbb{R}^n : \|x - c\| < \epsilon \} \]
    We define a verification function;
    \[f^S: \mathcal{P}( \mathbb{R}^n) \to \mathcal{P}( \mathbb{R}^m)\] such that $f^S(X) = \{y \mid x \in X, x = f(x)\}$.

    We want to verify, by computing \(f^S(X) \), if the class of the output \(r \in \R^m\) is a cat: 
    \[ f^S(X) \subseteq \{y \mid class(y) = \text{cat} \} \]


\end{frame}

\begin{frame}
    \frametitle{What I did during the course}
    \begin{itemize}
        \item Verification using intervals/boxes/hyperrectangles
        \item Verification using Zonotopes
        \item Probabilistic verification
        \item Analysis of the robustness of neural networks using Zones and Tropical Algebra
        \item  Quantified properties of neural networks,
        and neural network controlled CPS
    \end{itemize}

In this presentation I will focus on the verification using Tropical Algebra and Zones.
\end{frame}
\begin{frame}[fragile]
\frametitle{Verification methods using Zones}
    % definition of a zone

        Let \(x \in \mathbb{R}^n \). The zone domain is defined by:
        \[(\bigwedge_{i\leq i,j\leq n}^n x_i - x_j \leq c_{i,j}) \land (\bigwedge_{1\leq i\leq n} a_i \leq x_i \leq b_i)\]
        where \(a_i, b_i,  \in \mathbb{R} \) and \(c_{i,j} \in \mathbb{R} \cup \{+\infty\} \).

    In julia we can define it like this : 
\begin{lstlisting}[language= Julia]
struct Zone
    numinp::Int 
    numvars::Int
    DBM::Matrix{Float64}
end
\end{lstlisting}
And we can define \((n+1) \times (n+1) \) matrix \(C = (c_{i_j})\) called a DBM (difference bound matrix) representing the zone:
\[ \gamma(C) = \{x \in \mathbb{R}^n \mid \forall i,j \in \{0, \ldots, n\}, x_i - x_j \leq c_{i,j} \land x_0 = 0 \} \]
\end{frame}

\begin{frame}[fragile]
\frametitle{Closure of Zone}

The closed matrix \(C* \) is the smallest matrix that represents the same set of points as \(C \). It can be computed by the Floy-Warshall algorithm to update \((c_{i,j})_{0\leq i,j\leq n} \) as follows:
\begin{enumerate}
    \item Initialize \(c^*_{i,j} = c_{i,j} \) for all \(i,j \in \{0, \ldots, n\} \).
    \item For all \(i,j,k \in \{0, \ldots, n\} \),, \(c^*_{i,j} = \min(c^*_{i,j}, c^*_{i,k} + c^*_{k,j}) \).
\end{enumerate}

% code of the julia implementation of the floyd warshall algorithm
\begin{lstlisting}[language= Julia]
function zone_closure(input::Zone)
    c = input.DBM
    n = input.numvars + input.numinp + 1
    for k in 1:n
        for i in 1:n
            for j in 1:n
                c[i, j] = min(c[i, j], c[i, k] + c[k, j])
            end
        end
    end
    return Zone(input.numinp, input.numvars, c)
end
\end{lstlisting}
\end{frame}

\begin{frame}
\frametitle{Abstraction of linear maps}
    We can define the abstraction of a linear map \(f: \mathbb{R}^n \to \mathbb{R}^m \) as a matrix \(A \in \mathbb{R}^{m \times n} \) and a vector \(b \in \mathbb{R}^m \) such that:
    \[f(x) = Ax + b \]

    We can define the abstraction of a linear map as a tuple \((A, b) \) in julia:
\end{frame}

\end{document}